\chapter{Konzept}\label{ch:konzept}

\nocite{chollet_keras_2015,omalley_keras_2019,pedregosa_scikit-learn_2011,martin_abadi_tensorflow_2015}

\section{Exploration des SHL-Datensatzes}

\begin{figure}[h]
\includegraphics[width=0.5\linewidth, bb=0 0 401 386]{shl/label-quantities.pdf}
\caption{Klassenverteilungen im Hand-Datensatz der SHL-Challenge aus 2020.}\label{fig:shl-label-quantities}
\end{figure}

\begin{equation}
  \left| ACC \right| = \sqrt{ACC_x^2 + ACC_y^2 + ACC_z^2}
\end{equation}

\begin{equation}
  \left| MAG \right| = \sqrt{MAG_x^2 + MAG_y^2 + MAG_z^2}
\end{equation}

\begin{equation}
  \left| GYR \right| = \sqrt{GYR_x^2 + GYR_y^2 + GYR_z^2}
\end{equation}

\begin{figure}[h]
\includegraphics[width=\linewidth, bb=0 0 1395 386]{shl/akzelerometer-boxplot.pdf}
\caption{Statistische Verteilung der Sensorkomponenten im Akzelerometer-Signal anhand der Klassen im SHL-Datensatz.}\label{fig:shl-akzelerometer-boxplot}
\end{figure}

\begin{figure}[h]
\includegraphics[width=\linewidth, bb=0 0 1401 386]{shl/magnetometer-boxplot.pdf}
\caption{Statistische Verteilung der Sensorkomponenten im Magnetometer-Signal anhand der Klassen im SHL-Datensatz.}\label{fig:shl-magnetometer-boxplot}
\end{figure}

\begin{figure}[h]
\includegraphics[width=\linewidth, bb=0 0 1398 386]{shl/gyrosensor-boxplot.pdf}
\caption{Statistische Verteilung der Sensorkomponenten im Gyrosensor-Signal anhand der Klassen im SHL-Datensatz.}\label{fig:shl-gyrosensor-boxplot}
\end{figure}

\begin{figure}[h]
\includegraphics[width=\linewidth, bb=0 0 1057 386]{shl/sensor-histograms.pdf}
\caption{Statistische Verteilung der Signalstärken von $\left| ACC \right|$, $\left| MAG \right|$ und $\left| GYR \right|$ im SHL-Datensatz.}\label{fig:shl-sensor-histograms}
\end{figure}

\begin{figure}[h]
\includegraphics[width=\linewidth, bb=0 0 1039 443]{shl/sensor-fft.pdf}
\caption{Statistische Verteilung der Frequenzen von $\left| ACC \right|$, $\left| MAG \right|$ und $\left| GYR \right|$ anhand der Klassen im SHL-Datensatz.}\label{fig:shl-sensor-fft}
\end{figure}

% - Datenakquise
%   - Limitation auf lokale kinematische Messparameter
%   - Exploration des Sussex-Huawei-Locomotion Datensatzes
%   - Gewichtung des Datensatzes zur Vermeidung von Bias
%   - Aufteilung des Datensatzes in Train, Test und Validation
%   - Yeo-Johnson-Transformation der Eingangsdaten
%   - Konfiguration von Frequenzbandfiltern
%   - Konfiguration der Fensterlänge analog zu Anforderungen
%   - Bereitstellung der Daten durch Schnittstellen

% - Modellexploration
%   - Feature-Engineering: Shallow und Coding für DL-Features
%   - Constraint-basierte Rastersuche
%   -
