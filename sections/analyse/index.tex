\chapter{Anforderungsanalyse}\label{ch:analyse}

Wie im Grundlagenkapitel gezeigt wurde, ist die Erstellung einer Verkehrsmittelerkennung für Smartphones mit vielen verschiedenen möglichen Komponenten und Strategien verbunden. Um hieraus zu kondensieren, welche der gezeigten Ansätze mit Hinblick auf die Erstellung eines Konzeptes weiter verfeinert werden sollen, bietet sich eine differenzierte Betrachtung der Anforderungen an das spätere Konzept an. Anhand dessen lassen sich auch genau die verwandten Arbeiten identifizieren, welche für diese Arbeit einen Mehrwert bieten können.

\section{Rahmenbedingungen und Interessen}

Neben der konzeptuellen und prototypischen Realisierung der eigentlichen Verkehrsmittelerkennung müssen zunächst die Aktivitätsdaten ermittelt werden, welche als Grundlage für den Trainingsprozess und für die Evaluation dienen. Hierfür muss das Konzept eine Lösung bereitstellen, beispielsweise in Form einer eigenen Datenakquise oder durch Nutzung eines bestehenden Datensatzes. Die Qualität der akquirierten Daten ist von oberster Priorität, denn sie limitiert die spätere Generalisierungsfähigkeit des trainierten Modells. Das Konzept muss daher auch konkrete Strategien inkludieren, um die Qualität der Daten während und nach der Akquise möglichst hoch zu halten. Werden die Daten selbst aufgezeichnet, kann dies beispielsweise eine Wege- und Verkehrsmittelplanung und die Implementation einer hierfür genutzten Akquise-App mit Label-System inkludieren.

Der zentrale Teil des Konzeptes besteht nach der Konzeption der Datenakquise in der Erstellung einer Verkehrsmittelerkennung auf Smartphones. Dies erfolgt durch konzeptuelle Beschreibung und prototypische Implementation einer Software. In erster Linie dient die resultierende Software als Forschungskonzept, um die Plausibilität einer Verkehrsmittelerkennung auf Smartphones zu evaluieren. Eine Integration der Software, beispielsweise in der STADTRADELN-App, ist zunächst nicht vorgesehen. Die Interoperabilität der Verkehrsmittelerkennung mit anderen umliegenden Softwaresystemen und die Führung zur Produktreife ist somit von untergeordneter Bedeutung. Von zentraler Bedeutung ist hingegen, neben der Reproduzierbarkeit, die Möglichkeit zur Evaluation der Plausibilität anhand des Tradeoffs zwischen Ressourcenverbrauch und Ergebnisqualität. Für die Ergebnisqualität wurden im Grundlagenkapitel bereits Metriken und das Profiling als Aufzeichnungsmöglichkeit beschrieben. Das Konzept muss eine Lösung bereitstellen, wie diese Metriken mit dem Ressourcenverbrauch in Relation gesetzt werden können.

Eine weitere zentrale Rahmenbedingung resultiert aus der Überlegung, ob ein Training des Machine-Learning-Ansatzes auf dem Smartphone selbst benötigt wird. Dies ist insbesondere dann sinnvoll, wenn ein Machine-Learning-Modell bei Nutzung der dazugehörigen App individualisiert werden soll. Gleichzeitig ist das Training auf dem Smartphone mit größerem Rechen- und Energieaufwand und technologischen Anforderungen verbunden, sowie einer hieraus entstehenden Limitierung der nutzbaren Modellarchitekturen. Im Rahmen dieser Arbeit ist eine nachträgliche Anpassbarkeit oder Trainierbarkeit des Modells auf dem Smartphone wegen der fehlenden Notwendigkeit zur Individualisierung nicht vorgesehen. Mit einer Integration des Ansatzes in eine bestehende App und einer Erweiterung des dazugehörigen Adaptivitätskonzeptes kann sich diese Rahmenbedingung jedoch ändern. Darüber hinaus ist diese Rahmenbedingung davon abhängig, ob das Konzept durch ein kollaboratives In-Edge-Lernverfahren erweitert wird. Tauschen die Edge-Geräte Modellparameter aus, so ist die Anpassbarkeit des Modells auf dem Smartphone erforderlich. Die möglichen Vorteile hinter einer Erweiterung durch ein kollaboratives In-Edge-Lernverfahren, also vorrangig die Dezentralisierung und Auslagerung des Lernprozesses, ließen sich jedoch im Rahmen der Forschungsfragen dieser Arbeit nicht abschöpfen. An dieser Stelle überwiegt das Interesse einer niedrigeren Komplexität, um die Reproduzierbarkeit, Interpretierbarkeit und Aussagekraft der Evaluationsergebnisse zu bestärken.

\section{Top-Level-Architektur}

Eine mögliche Softwarearchitektur zur Realisation einer Verkehrsmittelerkennung für Smartphones ist in \Cref{fig:tla} gezeigt. Die gezeigte Softwarearchitektur ist separiert in die Verkehrsmittelerkennung selbst, sowie die notwendigen Komponenten zur Erstellung des Modells.

\begin{figure}[h]
\includegraphics[width=\linewidth, bb=0 0 713 282]{tla.pdf}
\caption[UML-Kompositionsstrukturdiagramm der Verkehrsmittelerkennung.]{UML-Kompositionsstrukturdiagramm der Verkehrsmittelerkennung.}\label{fig:tla}
\end{figure}

Bei der Abbildung wird analog zu den Rahmenbedingungen davon ausgegangen, dass die Modellerstellung auf einem externen Computer durchgeführt wird, zusammen mit einem Modellexport auf das Smartphone ohne nachträgliche Anpassbarkeit des Modells. Die zwei Hauptkomponenten sind in diesem Fall also strikt voneinander durch die Grenze zwischen Computer und Smartphone getrennt. Die Vorverarbeitung muss im zweigeteilten Ansatz möglichst auf Smartphone und Computer identisch implementiert werden, um bei denselben Datensegmenten die gleichen Inputs für das Machine-Learning-Modell zu erhalten.

