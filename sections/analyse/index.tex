\chapter{Anforderungsanalyse}\label{ch:analyse}

Wie im Grundlagenkapitel gezeigt wurde, ist die Erstellung einer Verkehrsmittelerkennung für Smartphones mit vielen verschiedenen möglichen Komponenten und Strategien verbunden. Um hieraus zu kondensieren, welche der gezeigten Ansätze mit Hinblick auf die Erstellung eines Konzeptes weiter verfeinert werden sollen, bietet sich eine differenzierte Betrachtung der Anforderungen an das spätere Konzept an. Anhand dessen lassen sich auch genau die verwandten Arbeiten identifizieren, welche für diese Arbeit einen Mehrwert bieten können.

\section{Rahmenbedingungen und Stakeholder}

Neben der konzeptuellen und prototypischen Realisierung der eigentlichen Verkehrsmittelerkennung müssen Aktivitätsdaten ermittelt werden, welche als Grundlage für den Trainingsprozess und für die Evaluation dienen. Hierfür muss das Konzept eine Lösung bereitstellen, beispielsweise in Form einer eigenen Datenakquise oder durch Nutzung eines bestehenden Datensatzes. An die akquirierten Daten sind hierbei weitere Rahmenbedingungen gekoppelt. Bei späterer Nutzung des Supervised Learning müssen die Daten zusätzlich manuell, automatisiert oder halbautomatisiert gelabelt werden. Zusätzlich zur Eignungsfeststellung der Daten anhand einer explorativen Analyse werden im Falle einer selbstständigen Akquise gegebenenfalls Testnutzer und eine eigene Akquise-Applikation benötigt. Gleichzeitig sind die experimentellen Bedingungen in dieser Weise direkt nachvollziehbar. Die Datenakquise kann in diesem Fall auch geplant werden, beispielsweise durch Erstellung eines Wegeplans mit verschiedenen genutzten Verkehrsmitteln und verschiedenen Tragepositionen des Smartphones.

Der zentrale Teil des Konzeptes besteht nach der Konzeption der Datenakquise in der Erstellung einer Verkehrsmittelerkennung auf Smartphones. Dies erfolgt durch konzeptuelle Beschreibung und prototypische Implementation einer Software. In erster Linie dient die resultierende Software als Forschungskonzept, um die Plausibilität einer Verkehrsmittelerkennung auf Smartphones zu evaluieren. Eine Integration der Software, beispielsweise in der STADTRADELN-App, ist zunächst nicht vorgesehen. Die Interoperabilität der Verkehrsmittelerkennung mit anderen umliegenden Softwaresystemen und die Führung zur Produktreife ist somit von untergeordneter Bedeutung. Von zentraler Bedeutung ist hingegen, neben der Reproduzierbarkeit, die Möglichkeit zur Evaluation der Plausibilität anhand des Tradeoffs zwischen Ressourcenverbrauch und Ergebnisqualität. Für die Ergebnisqualität wurden im Grundlagenkapitel bereits Metriken und das Profiling als Aufzeichnungsmöglichkeit beschrieben. Darüber hinaus muss das Konzept quantitative Definitionen von Metriken für den Ressourcenverbrauch bereitstellen, um den Tradeoff vergleichend zu evaluieren.

