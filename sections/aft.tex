\chapter{Einleitung}\label{ch:einleitung}\pagenumbering{arabic}

\section{Gegenstand und Motivation}\label{sec:gegenstand-und-motivation}

% Darstellung des Kontextes der Arbeit

STADTRADELN\footnote{\url{https://www.stadtradeln.de/darum-geht-es} (Abgerufen am 25.2.2021)} ist ein Projekt, bei dem Radfahrende teilnehmen können, um kollektiv mit dem Rad gefahrene Kilometer für den Klimaschutz zu sammeln. Begleitet wird das Projekt durch eine mobile App, mithilfe derer Nutzende ihre gefahrenen Kilometer tracken und sich mit anderen Teilnehmenden vergleichen können. Im Rahmen des Forschungsprojektes Movebis\footnote{\url{https://www.movebis.org/das-projekt/} (Abgerufen am 25.2.2021)} werden die hierbei erhobenen Daten ausgewertet und mithilfe von konkreten Kenngrößen visuell aufbereitet. Die aufbereiteten Daten werden der kommunalen Verkehrsplanung anschließend zur Verfügung gestellt, mit dem Ziel, die Planung der Radverkehrsinfrastruktur zu verbessern.
\\

% Fokussierung auf ein konkretes Teilgebiet des Kontextes + Motivation

\noindent Die von der STADTRADELN-App erhobenen Daten\footnote{\url{https://www.stadtradeln.de/datenschutz} (Abgerufen am 25.2.2021)} werden zu einem Server hochgeladen und von diesem für die spätere Auswertung protokolliert. Hierbei können Nutzende selbst entscheiden, zu welchen Zeitpunkten die Datenerfassung gestartet und beendet wird. Insbesondere kann es hierbei jedoch vorkommen, dass Nutzende speziell zum Anfang und Ende des Fahrradfahrens oder beim Wechsel auf ein anderes Verkehrsmittel Datenpakete aufzeichnen, bei denen kein Fahrrad gefahren wurde. Als Konsequenz hieraus ist nicht ausgeschlossen, dass für die spätere Auswertung unbrauchbare Datenpakete aufgezeichnet und zum Server übermittelt werden. Im Umfeld des Movebis-Projektes wurden daher bereits verschiedene Ansätze entwickelt, durch deren Einsatz für den Zielkontext (Fahrrad fahren) plausible von unplausiblen Datensätzen unter Berücksichtigung einer bestimmten Fehlerrate getrennt werden können \cite{matusek_anwendung_2019, stojanov_continuous_2020, werner_kontinuierliche_2020}. Bei den Ansätzen handelt es sich um Machine-Learning-Klassifikationssysteme, welche auf die Mustererkennung innerhalb der aufgezeichneten Datensätze trainiert wurden. Die entwickelten Machine-Learning-Ansätze ermöglichen eine Filtration der Datensätze zum Zeitpunkt der Auswertung, reduzieren jedoch \textit{nicht} die Aufzeichnung und den initialen Transfer der unplausiblen Datensätze vom Smartphone (hierbei betrachtet als Client) zum Server.
\\

% Erläuterung des konkreten Gegenstands der Arbeit + Motivation + Relevanz

\noindent Mithilfe eines clientseitigen Erkennungsmechanismus für die Zugehörigkeit der Datenpakete zum gewünschten Zielkontext könnten nicht zugehörige Datenpakete vom Netzwerktransfer ausgeschlossen werden, um die andernfalls genutzte Netzwerkbandbreite und Energie auf dem mobilen Endgerät zu sparen. Durch eine solche Erkennung wäre es außerdem möglich, Nutzende bei der Beendigung des Radfahrens an die noch laufende Datenaufzeichnung, beispielsweise über Notifikationen, zu erinnern. Eine ähnliche Erkennung wird bereits im Betriebssystem der Apple Watch seit 2018 (mit Release von WatchOS 5) durchgeführt, um verschiedene sportliche Aktivitäten zu erkennen und in Form von \enquote{Workout Reminders} in das Adaptionskonzept\footnote{Adaption: Die Anpassung von mobilen Applikationen als Reaktion auf einen sich ändernden Kontext} des Betriebssystems zu integrieren\footnote{\url{https://support.apple.com/en-us/HT204523} (Abgerufen am 25.2.2021)}. Außerdem stellen die beiden marktführenden mobilen Betriebssysteme Android\footnote{\url{https://developers.google.com/android/reference/com/google/android/gms/location/DetectedActivity} (Abgerufen am 3.3.2021)} und iOS\footnote{\url{https://developer.apple.com/documentation/coremotion/cmmotionactivity/1615451-cycling} (Abgerufen am 3.3.2021)} bereits abstrakte Schnittstellen zur Verfügung, mithilfe derer eine rudimentäre Aktivitätserkennung auf Grundlage von wenigen fest definierten Klassen (darunter auch Fahrrad fahren) durchgeführt werden kann.
\\

\noindent Mit der steigenden Verfügbarkeit von mobilen Machine-Learning-Frameworks wie TensorFlow Lite\footnote{\url{https://www.tensorflow.org/lite} (Abgerufen am 3.3.2021)} (iOS und Android, veröffentlicht in 2017) und CoreML\footnote{\url{https://developer.apple.com/documentation/coreml} (Abgerufen am 3.3.2021)} (iOS, veröffentlicht in 2017) und der Integration von spezialisierten Hardware-Koprozessoren für Machine-Learning in Smartphones bieten sich zunehmend mehr Möglichkeiten für die App-Entwicklung, Machine-Learning-Modelle auf Smartphones zu betreiben. Unter anderem ist es hierdurch auch möglich, Machine-Learning-Modelle auf Smartphones zu transferieren. Diese Möglichkeit bietet die Grundlage für die Problem- und Zielstellung einer Diplomarbeit, welche im Rahmen der Analyse des Forschungsthemas vorbereitet werden soll.


\section{Problem- und Zielstellung}\label{sec:problem-und-zielstellung}

In der Diplomarbeit soll untersucht werden, wie die zuvor entwickelten Machine-Learning-Ansätze aus \cite{matusek_anwendung_2019, stojanov_continuous_2020, werner_kontinuierliche_2020} anhand aktueller Machine-Learning-Frameworks auf Smartphones portiert werden können. Hierfür müssen die bestehenden Modelle analysiert und verglichen werden, um schließlich ein Portierungskonzept zu erstellen, mithilfe dessen die bestehenden Modellarchitekturen für mobile Anwendungen optimiert und transferiert werden sollen. Hierbei soll insbesondere auch untersucht werden, welche Hardware- und Softwareanforderungen an die bestehenden Modelle gekoppelt sind und welche Optimierungsmöglichkeiten existieren, um den Zeit-, Energie- und Speicherverbrauch auf Smartphones zu reduzieren. Das ermittelte Portierungskonzept soll in einer prototypischen App integriert werden. Auf dieser Grundlage sollen die portierten Modelle bezüglich deren Effizienz evaluiert werden, im Speziellen auch gegenüber den bestehenden abstrakten systemeigenen Schnittstellen. Zur differenzierten Diskussion dieses zentralen Forschungsgegenstands sollen folgende konkrete Forschungsfragen eruiert werden:

\begin{researchquestion}\label{rq1}
Aus welchen architekturellen Elementen bestehenden die existierenden Machine-Learning-Ansätze zur Klassifikation der STADTRADELN-Daten und wie können diese auf Smartphones transferiert werden?
\end{researchquestion}

\begin{researchquestion}\label{rq2}
Welche quantitativen Anforderungen stellen die bestehenden Machine-Learning-Ansätze an den Zeit-, Energie- und Speicherbedarf bei der Ausführung auf Smartphones?
\end{researchquestion}

\begin{researchquestion}\label{rq3}
Wie stehen die in \Cref{rq2} gefundenen Anforderungen an Zeit-, Energie- und Speicherbedarf mit der Qualität der Klassifikation des jeweiligen Machine-Learning-Ansatzes in Relation und welcher der bestehenden Ansätze eignet sich am besten für den Einsatz auf Smartphones?
\end{researchquestion}

\begin{researchquestion}\label{rq4}
Welche Strategien können auf die bestehenden Machine-Learning-Ansätze angewandt werden, um deren Energie- und Speicherbedarf zu optimieren, und welche quantitativen Auswirkungen hat dies gleichzeitig auf die Qualität der Klassifikation?
\end{researchquestion}

\chapter{Aufbau der Arbeit}\label{ch:aufbau-der-arbeit}

Die Analyse des Forschungsthemas gliedert sich chronologisch als Vorbereitung der Diplomarbeit ein, welche die oben genannten Forschungsfragen klären soll. Hierzu sollen bereits die relevanten Grundlagen und verwandten Arbeiten in einer möglichst detaillierten Form erarbeitet werden. Die angefertigte Ausarbeitung dient anschließend als Ausgangspunkt für die weiteren Analysen, Konzepte und evaluationsbasierten Schlussfolgerungen der Diplomarbeit.

\todo{Weiter ausformulieren}

\chapter{Grundlagen}\label{ch:grundlagen}

\todo{Einleitung der Grundlagen schreiben}

\section{Datenerfassung in der STADTRADELN-App}

\begin{figure}[H]
\includegraphics[width=\linewidth, bb=0 0 598 404]{datengrundlage.pdf}
\caption{Die bei der Nutzung der STADTRADELN-App erhobenen Daten im Überblick.}\label{fig:datengrundlage}
\end{figure}

\noindent Bei der Nutzung der STADTRADELN-App werden unterschiedliche Datensätze erhoben, verarbeitet und gespeichert. \Cref{fig:datengrundlage} zeigt die in der Datenschutzerklärung der STADTRADELN-App beschriebenen Datensatzstrukturen\footnote{\url{https://www.stadtradeln.de/datenschutz/} (Abgerufen am 3.4.2021)} im Überblick. Neben allgemeinen personenbezogenen Daten werden auch Daten für die Meldeplattform \enquote{RADar!}\footnote{\url{https://www.radar-online.net/home} (Abgerufen am 3.4.2021)} verarbeitet, bei der vor allem die Markierung von Orten für die Ausbesserung der Radverkehrsinfrastruktur im Vordergrund steht. Die im Rahmen dieser Arbeit unabhängig von den anderen Datensätzen betrachteten Aktivitätsdaten sind auf der rechten Seite in \Cref{fig:datengrundlage} gezeigt. Die Datensätze bestehen aus über die Fahrdauer kontinuierlich erfassten Messparametern:

\begin{itemize}
  \item GPS-Koordinaten des jeweiligen Smartphone-Geolokalisationssystems, bestehend aus geografischer Länge und Breite, Höhe, Datum und Zeit, sowie einer Genauigkeitsangabe und der aktuellen Geschwindigkeit
  \item Beschleunigungsdaten des Smartphone-Akzelerometers in drei Raumdimensionen
  \item Neigungswinkeldaten des Gyroskop-Sensors in Winkelauslenkungen dreier Rotationsachsen
  \item Orientierungsdaten des Magnetometer-Sensors in drei Raumdimensionen
  \item Weitere Hilfsdaten, wie eine Geräte-Identifikationsnummer und das Smartphone-Modell
\end{itemize}

\noindent Diese Messparameter sollen als Grundlage für die im Einleitungskapitel motivierte Aktivitätserkennung fungieren. Daher sollen diese nun genauer betrachtet werden. Dabei spielen insbesondere folgende Fragen eine zentrale Rolle:

\begin{enumerate}
\item Wie werden die jeweiligen Messparameter technisch erfasst und mit welchen Wertebereichen ist hierbei zu rechnen?
\item In welchen Dimensionen sind die Messparameter strukturiert und welche Auskunft geben sie?
\item Welche Probleme und Herausforderungen können bei der Erfassung der Messparameter auftreten und wie können diese gelöst werden?
\end{enumerate}

\subsection{Geolokalisationsinformationen}

Das \textit{Navigational Satellite Timing and Ranging - Global Positioning System}, kurz \textit{NAVSTAR GPS} oder auch nur \textit{GPS} ist ein System aus (Stand 9. Januar 2021) 31 Satelliten, welche die Erde in einer Höhe von ca. 20.000 km umkreisen \cite{us_space_force_gpsgov_2021}.

\subsubsection{Terminologische Einordnung}

Der Begriff GPS wird teils als Synonym für globale Satellitennavigationssysteme (engl. \textit{Global Navigation Satellite System GNSS}) verstanden. Neben dem amerikanischen GPS existieren hierbei jedoch noch weitere Systeme, wie das europäische Galileo, das russische Glonass, sowie das chinesische BeiDou \cite{olynik_temporal_2002}. Moderne Smartphone-Geolokalisationssysteme nutzen teils eine Kombination dieser globalen Satellitennavigationssysteme \todo{Beleg ergänzen}. In Abhängigkeit der genutzten Smartphone-Schnittstellen ist es also möglich, dass die in der Datenschutzerklärung der STADTRADELN-App beschriebenen GPS-Daten von verschiedenen Satellitennavigationssysteme stammen.

\subsubsection{Funktionsweise}

Die GPS-Satelliten besitzen sehr präzise Atomuhren und senden über verschiedene militärische und zivile Frequenzen kontinuierlich ihre Position und Zeit zur Erde \cite{us_space_force_global_2020}. Mithilfe von speziellen Antennen können diese Signale empfangen werden, wobei die empfangenen Signale amplifiziert und weiter verarbeitet werden müssen \cite{jan_van_sickle_gps_2021}. Als Ergebnis der internen Prozessierung des GPS-Signals kann die Distanz zum GPS-Satelliten anhand der Signallaufzeit unter Berücksichtigung verschiedener Fehlerquellen polynomiell berechnet werden \cite{olynik_temporal_2002, sameet_mangesh_deshpande_study_2004}. Somit stehen anschließend die Position des Satelliten und die berechnete Distanz zur Verfügung. Bei einem Kontakt zu mindestens 4 Satelliten kann der Ort des Empfängers (Höhe, Breitengrad, Längengrad) ermittelt werden, indem eine Intersektion von 4 Sphären durchgeführt wird, wobei das Zentrum der jeweiligen Sphäre die transmittierte Position des GPS-Satelliten ist, und der Radius der Sphäre sich durch die berechnete Distanz bestimmt \cite{zhang_senstrack_2013}. Die satellitengestützte Bestimmung ist hierbei passiv, da keine aktive Kommunikation zum Satelliten stattfindet \cite{kaplan_understanding_2005}. Smartphones nutzen zusätzlich zur satellitengestützten Positionsbestimmung Eigenschaften des WiFi- und GSM-Funknetzes, um die Präzision der Positionsbestimmung zu verbessern und den Energieverbrauch zu reduzieren \cite{zhang_senstrack_2013}. Als Ergebnis der hybriden Positionsbestimmung stellt das Smartphone-Geolokalisationssystem mehrere Messparameter zur Verfügung, darunter die Ortsinformationen (Höhe, Breitengrad, Längengrad), abgeleitete Informationen wie die Geschwindigkeit und eine Genaugkeitsangabe.

\subsubsection{Herausforderungen}

Die von den jeweiligen Satelliten (über elektromagnetische Wellen) ausgesendeten GPS-Informationen unterliegen dem physikalischen Abstandsgesetz.

\begin{equation}\label{eq:abstandsgesetz}
I_{Signal} \propto \frac{1}{d^{2}}
\end{equation}
wobei:
\begin{conditions}
  I_{Signal} & Intensität des GPS-Signals  \\
  d & Distanz zwischen GPS-Satellit und Empfänger
\end{conditions}

\noindent Das in \Cref{eq:abstandsgesetz} gezeigte Abstandsgesetz bedingt, dass wegen der begrenzten Signalemissionsstärke des Satelliten und der großen Distanz zum Empfänger die Empfindlichkeit des Empfängersystems verhältnismäßig hoch sein muss, in einer Größenordnung von -130 dBm (Dezibel Milliwatt) \cite{sameet_mangesh_deshpande_study_2004}. Die elektronische Prozessierung des Signals erfordert hierdurch eine entsprechende elektrische Leistung \cite{jan_van_sickle_gps_2021}. Die erforderliche elektrische Leistung stellt für Smartphones durch deren Limitation durch die Kapazität des Akkus ein signifikantes Problem dar \cite{zhang_senstrack_2013, oshin_improving_2012, constandache_enloc_2009, zhuang_improving_2010}. Eine energiesparendere Geolokalisation über GSM- und WiFi-Funknetze ist weniger akkurat als GPS und kann daher in Anwendungsfällen mit einer benötigten Mindestpräzision lediglich begleitend eingesetzt werden \cite{paek_energy-efficient_2010}. Aufgrund der Abhängigkeit zum Satellitensignal (respektive zu GSM- und WiFi-Funknetzen) sind weitere Probleme die Ortsabhängigkeit \cite{zhang_senstrack_2013, kaplan_understanding_2005}, die Wetterabhängigkeit \cite{paek_energy-efficient_2010}, sowie mögliche Interferenzen und andere Signaltransmissionsstörungen \cite{olynik_temporal_2002, sameet_mangesh_deshpande_study_2004}. Im Rahmen der Anwendungsentwicklung sowie der Entwicklung einer Aktivitätserkennung auf Grundlage der Geolokalisationsinformationen sind diese Probleme zu berücksichtigen, insbesondere wegen der ständig variierenden Präzision \cite{zhang_senstrack_2013}.

\subsubsection{Lösungen}

Zur Verbesserung der hybriden Geolokalisation über GPS-, GSM- und WiFi-Signale wurden verschiedene Konzepte entwickelt. Der Energieverbrauch kann signifikant reduziert werden, indem die GPS-Abtastrate gesenkt wird \cite{constandache_enloc_2009, lu_jigsaw_2010}. Hierzu können zusätzliche Informationen des Akzelerometers \cite{oshin_improving_2012} und des Orientierungssensors \cite{zhang_senstrack_2013} genutzt werden, um die GPS-Abtastrate adaptiv zu senken. Außerdem kann die erforderliche Präzision der Geolokalisation zusammen mit der primär genutzten Methode (GPS, GSM, WiFi) adaptiv angepasst werden \cite{lin_energy-accuracy_2010, zhang_senstrack_2013}. Aktuelle Schnittstellen in iOS- und Android-Smartphones greifen auf diese Prinzipien zurück und bieten dem Anwendungsentwickler Möglichkeiten, die benötigte Präzision einzustellen, um den Energieverbrauch zu reduzieren \cite{google_inc_fused_2021, apple_inc_desiredaccuracy_2021}.

\subsection{Kinematische Messparameter}

Neben den GPS-Geolokalisationsinformationen werden in der STADTRADELN-App außerdem Messparameter mithilfe von Sensoren des Smartphones aufgezeichnet. Diese Messparameter besitzen die Gemeinsamkeit, dass sie auch ohne externe Funksignale (GPS, GSM, WiFi) messbar sind und sich auf die Bewegung und Ausrichtung des Gerätes konzentrieren (Kinematik). Es handelt sich hierbei um jeweils triaxiale Messparameter in Form von Beschleunigungsdaten des Akzelerometers, Neigungsdaten des Gyrosensors, sowie Orientierungsdaten des Magnetometers.

\todo{Infografik über die Messachsen einfügen}

\subsubsection{Funktionsweise}

Das Akzelerometer des Smartphones basiert auf einem mikroelektronischen mechanischem System \textit{(MEMS)}, welches die mechanische Beschleunigung des Smartphones in vom Mikrochip interpretierbare elektrische Signale wandelt \cite{matej_andrejasic_mems_2008, constantinescu_capacitive_2013}. Hierbei wird eine sehr kleine gefederte Masse, welche sich zwischen zwei Elektroden befindet, in Bewegung versetzt. Die Bewegungsänderung bewirkt eine Spannungsänderung am Kondensator, welche vom System als Beschleunigung interpretiert werden kann \cite{dey_accelprint_2014}. Um neben der Beschleunigung auch die Winkelauslenkung (Neigung) des Gerätes zu messen, wird ein Gyrosensor eingesetzt. Gyrosensoren bestehen traditionell aus einer rotierenden Masse, welche durch eine entsprechende Lagerung in Relation zum umgebenden Objekt statisch bleibt. Dies ist physikalisch begründet in der Drehimpulserhaltung. Die Auslenkung des umgebenden Objektes in Relation zur rotierenden Masse kann anschließend elektrisch gemessen werden \cite{singh_piezoelectric_2007}. Wegen der benötigten Größe des traditionellen Apparats wurde das Wirkprinzip mikroelektronisch auf der Grundlage von sogenannten Piezo-Elementen (zum Beispiel einem Quarzkristall) implementiert. Hierbei induziert die Auslenkung des Gerätes eine Spannung über den sogenannten Piezoeffekt \cite{singh_piezoelectric_2007, ichimura_fem_2002, koitabashi_improvement_2002}. Neben dem Gyrosensor und dem Akzelerometer wird zusätzlich auch das Magnetfeld über ein Magnetometer gemessen. Es handelt sich hierbei ebenfalls um ein mikroelektronisches Bauteil, welches über ein komplexes optisches Messsystem anhand von Alkalimetalldämpfen (zum Beispiel Rubidium oder Cäsium) das lokale Magnetfeld messen kann \cite{schwindt_chip-scale_2004, dmitry_budker_alkali_2021}. Anhand der Magnetfelddaten können die triaxialen Orientierungsdaten abgeleitet werden, anhand der Ausrichtung des Erdmagnetfelds.

\subsubsection{Herausforderungen}

Ähnlich zur Geolokalisation über GPS, GSM und WiFi ist die Erfassung der lokalen kinematischen Messparameter (Beschleunigung, Winkelauslenkung, Orientierung) auch mit Problemen behaftet. Bei der Interpretation dieser Messparameter kann die Lage (Ausrichtung) des Smartphones zu Beginn und auch während der Messung in Relation zum Bezugssystem (beispielsweise in Bezug zum Erdboden) von entscheidender Bedeutung sein. So ist insbesondere bei Beschleunigungsdaten des Akzelerometers häufig die Erdschwerkraft im triaxialen Vektorsystem integriert \cite{ken_taylor_activity_2011, tundo_correcting_2013}. Bei einer Neigungsänderung des Smartphones wird somit auch immer eine Änderung des Gravitationsvektors gemessen. Dies erschwert die Interpretation der gemessenen Daten. Ebenfalls erschwerend wirken sich von Gerät zu Gerät unterschiedliche inhärente Messfehler und -parameter der Sensoren aus. Beispielsweise kann die Abtastrate der Sensoren und der genutzten Systemschnittstellen zur Aufzeichnung variieren. Bei der Messung der Akzelerometer-Daten ist mit einem Rauschen der Sensorwerte zu rechnen \cite{dey_accelprint_2014, ravi_deep_2016}. Dies ist übergreifend zurückzuführen auf unterschiedliche Imperfektionen in der Fertigung und die altersbedingte Degradation von mikroelektronischen Chips \cite{matej_andrejasic_mems_2008, constantinescu_capacitive_2013, hillman_manufacturing_2004}. \cite{dey_accelprint_2014} konnten nachweisen, dass allein anhand des gerätespezifischen Rauschens identifiziert werden kann, welches Smartphone zur Aufzeichnung der Messwerte (in diesem Fall Akzelerometer-Daten) genutzt wurde. Die diesem Ergebnis zugrundeliegenden Rauschmuster sind von hoher Relevanz für die spätere Signalprozessierung. Durch die zu antizipierende inhärente Abweichung der Sensordaten vom tatsächlichen Wert eignen sich die Beschleunigungsdaten in ihrer Reinform auch nicht für eine vollständige Bewegungsrekonstruktion und Lokalisation durch Aggregation der Bewegungsänderungen im Raum \cite{takeda_drift_2014}. Neben diesen Herausforderungen besteht mit Hinblick auf die Orientierungsdaten ein weiteres Problem. Da das Erdmagnetfeld (ca. 30 bis 60 Mikrotesla) im Vergleich zu einigen künstlichen Magnetfeldquellen wie Kühlschränken (ca. 5 Millitesla) \cite{schirmer_smartphone_2016, moreno-torres_evaluation_2013} relativ schwach ist, kann der Magnetometer-Sensor leicht gestört oder gar dekalibriert werden \cite{schirmer_smartphone_2016, zhang_preliminary_2012}, was zu signifikanten Fehlern führen kann \cite{kunze_can_2010}.

\subsubsection{Lösungen}

Zur Mitigation der Herausforderungen aus der vorigen Sektion wurden bereits verschiedene Methoden entwickelt. Zur Normalisierung der möglicherweise von Gerät zu Gerät unterschiedlichen Abtastraten und zur Synchronisation von asynchron aufgezeichneten Daten unterschiedlicher Herkunft ist ein Up- oder Downsampling mithilfe einer Interpolation der Datenpunkte möglich \cite{matusek_anwendung_2019}. Eventuell auftretendes Rauschen kann durch einen Tiefpassfilter respektive einem Hochpassfilter eliminiert werden. \cite{nutter_design_2018} nutzen einen Tiefpassfilter bei 0.3 Hz, einen Medianfilter von Länge 3, sowie einen Butterworth-Filter bei 20 Hz für die Vorverarbeitung von Akzelerometer-Daten. \cite{takeda_drift_2014} verwenden einen Butterworth-Filter für die Verarbeitung von Auslenkungsdaten des Gyrosensor. \cite{abdulla_measuring_2013} applizieren einen Tiefpassfilter bei 5 Hz auf Magnetometerdaten. Neben diesen Signalverarbeitungsmethoden ist es auch möglich, speziell bei Beschleunigungsdaten den inkludierten Gravitationsvektor zu eliminieren und mithilfe einer Rotationsmatrix das rotierbare Koordinatensystem des Smartphones in ein nicht rotierbares aufrechtes Koordinatensystem zu überführen \cite{nutter_design_2018, tundo_correcting_2013}. Bei der Wegrekonstruktion durch die Aggregation der Bewegungsänderungen lässt sich außerdem eine Drift-Korrektur anwenden, bei der Klassifikation der Sensordaten analog zur Aufgabenstellung dieser Arbeit spielt dies jedoch noch eine untergeordnete Rolle, da eine zeitliche Aggregation möglicherweise wichtige Informationen für die Klassifikation der Sensordaten eliminieren würde \cite{takeda_drift_2014}. \\

\noindent Allgemein lässt sich anhand der einschlägigen Fachliteratur beobachten, dass insbesondere Beschleunigungsdaten des Akzelerometers in vielen wissenschaftlichen Arbeiten als Grundlage für eine Aktivitätsklassifikation verwendet werden \cite{gudur_activeharnet_2019, nutter_design_2018}. Beschleunigungsdaten enthalten einen ausreichenden Informationsgehalt, um komplexe Bewegungsabläufe zu rekonstruieren \cite{cai_touchlogger_2011, marquardt_spiphone_2011, aviv_practicality_2012}. Sie können für eine Aktivitätserkennung mit Gyrosensor-Daten verknüpft werden \cite{ravi_deep_2016}. Magnetometer-Daten genügen nicht als alleinige Datengrundlage zur Aktivitätserkennung, eignen sich jedoch als Ergänzung zu Beschleunigungsdaten und Auslenkungsdaten \cite{abdulla_measuring_2013, kunze_can_2010}.

\todo{Fremdwörter erklären als Fußnote}

% Weitere Probleme: Datenschutz und -sicherheit
% -> GPS Daten sind unter Umständen sehr privat
% -> Accelerometer-Daten können aber auch zur Personenidentifikation genutzt werden [Tundo et al.]
% (Nicht behandelt in dieser Arbeit)

% Daten stellen eine Art der Kontextsensitivität dar
% Kontextsensitive Adaption: Messung, Integration, Adaption
% -> Engineering Ubiquitous Systems lesen!
% Klassifikation der gemessenen Daten fällt somit in Integration

\section{Machine Learning}

% Überleitung von Datensatz auf Problem der Künstlichen Intelligenz
% Motivation und Definition von Künstlicher Intelligenz
% Machine Learning ist ein Teil von Künstlicher Intelligenz
% -> Hierzu gehört noch mehr als Machine Learning, was?
% -> Vielleicht eignet sich hier ein Venn-Diagramm.

% Klassische Probleme der Künstlichen Intelligenz:
% - Clustering (Datenpakete zu Gruppen mit Gemeinsamkeiten zusammenfassen bei Fehlerminimierung)
% - Regression (Modellparameter bilden Verlauf eines mathematischen Zusammenhangs nach, Vorhersage von neuen Datenpunkten durch Extrapolation)
% - Generation und Transformation (Erkennung von Mustern, Replikation dieser), bsp. Deep Dream, GPT-2, ...
% - Klassifikation (Datenpakete bestimmter Klasse zuordnen)
% -> Klassifikation ist das Problem dieser Arbeit!

% Human Activity Recognition als Klassifikationsproblem
% ... (braucht noch mehr Ideen)

% ...


\todo{Konzept: Zeitfenster diskreter Länge behandeln}
\todo{Konzept: Behandlung fehlender Daten}
\todo{Konzept: Glättung der Daten}
