\section{Machine Learning}\label{sec:machine-learning}

In \Cref{sec:datenerfassung-und-vorverarbeitung} wurde detailliert beschrieben, wie sich die Aktivitätsdaten innerhalb der STADTRADELN-App strukturieren und welche Herausforderungen bei der Weiterverarbeitung der Daten zu beachten sind. In diesem Abschnitt sollen nun Methoden vorgestellt werden, mithilfe derer eine Interpretation der vorverarbeiteten Daten möglich ist.

\subsection{Klassische Probleme der künstlichen Intelligenz}

Die Klassifikation der Aktivität anhand der beschriebenen Daten ist kein triviales Problem. Neben mannigfaltigen Fehlerursachen (in \Cref{sec:datenerfassung-und-vorverarbeitung} erläutert) ist auch die Komplexität der zu interpretierenden Daten denkbar hoch. Smartphones können zum Beispiel in verschiedenen Positionen (Hand, Hosentasche, Rucksack, ...) am Körper bzw. am/im Verkehrsmittel getragen werden. Weitere Ursachen für zwischen Messungen gleicher Aktivität abweichenden Daten sind unterschiedlich beschaffene Verkehrsmittel (zum Beispiel Rennrad vs. Mountainbike), unterschiedliche anatomische Gegebenheiten und Bewegungsmuster der jeweiligen Person, sowie die Art des Untergrundes (zum Beispiel Schotterweg vs. asphaltierte Straße) oder die allgemeine Beschaffenheit des Geländes.

% Überleitung von Datensatz auf Problem der Künstlichen Intelligenz
% Motivation und Definition von Künstlicher Intelligenz
% Machine Learning ist ein Teil von Künstlicher Intelligenz
% -> Hierzu gehört noch mehr als Machine Learning, was?
% -> Vielleicht eignet sich hier ein Venn-Diagramm.

% Klassische Probleme der Künstlichen Intelligenz:
% - Clustering (Datenpakete zu Gruppen mit Gemeinsamkeiten zusammenfassen bei Fehlerminimierung)
% - Regression (Modellparameter bilden Verlauf eines mathematischen Zusammenhangs nach, Vorhersage von neuen Datenpunkten durch Extrapolation)
% - Generation und Transformation (Erkennung von Mustern, Replikation dieser), bsp. Deep Dream, GPT-2, ...
% - Klassifikation (Datenpakete bestimmter Klasse zuordnen)
% -> Klassifikation ist das Problem dieser Arbeit!

% Human Activity Recognition als Klassifikationsproblem
% ... (braucht noch mehr Ideen)

% ...

\todo{Gradientenabstiegsverfahren behandeln}
