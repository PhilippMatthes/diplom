\section{Datenerfassung und -vorverarbeitung}

\begin{figure}[H]
\includegraphics[width=\linewidth, bb=0 0 598 404]{datengrundlage.pdf}
\caption{Die bei der Nutzung der STADTRADELN-App erhobenen Daten im Überblick.}\label{fig:datengrundlage}
\end{figure}

\noindent Bei der Nutzung der STADTRADELN-App werden unterschiedliche Datensätze erhoben, verarbeitet und gespeichert. \Cref{fig:datengrundlage} zeigt die in der Datenschutzerklärung der STADTRADELN-App beschriebenen Datensatzstrukturen\footnote{\url{https://www.stadtradeln.de/datenschutz/} (Abgerufen am 3.4.2021)} im Überblick. Neben allgemeinen personenbezogenen Daten werden auch Daten für die Meldeplattform \enquote{RADar!}\footnote{\url{https://www.radar-online.net/home} (Abgerufen am 3.4.2021)} verarbeitet, bei der vor allem die Markierung von Orten für die Ausbesserung der Radverkehrsinfrastruktur im Vordergrund steht. Die im Rahmen dieser Arbeit unabhängig von den anderen Datensätzen betrachteten Aktivitätsdaten sind auf der rechten Seite in \Cref{fig:datengrundlage} gezeigt. Die Datensätze bestehen aus über die Fahrdauer kontinuierlich erfassten Messparametern:

\begin{itemize}
  \item GPS-Koordinaten des jeweiligen Smartphone-Geolokalisationssystems, bestehend aus geografischer Länge und Breite, Höhe, Datum und Zeit, sowie einer Genauigkeitsangabe und der aktuellen Geschwindigkeit
  \item Beschleunigungsdaten des Smartphone-Akzelerometers in drei Raumdimensionen
  \item Neigungswinkeldaten des Gyroskop-Sensors in Winkelauslenkungen dreier Rotationsachsen
  \item Orientierungsdaten des Magnetometer-Sensors in drei Raumdimensionen
  \item Weitere Hilfsdaten, wie eine Geräte-Identifikationsnummer und das Smartphone-Modell
\end{itemize}

\noindent Diese Messparameter sollen als Grundlage für die im Einleitungskapitel motivierte Aktivitätserkennung fungieren. Daher sollen diese nun genauer betrachtet werden. Dabei spielen insbesondere folgende Fragen eine zentrale Rolle:

\begin{enumerate}
\item Wie werden die jeweiligen Messparameter technisch erfasst und mit welchen Wertebereichen ist hierbei zu rechnen?
\item In welchen Dimensionen sind die Messparameter strukturiert und welche Auskunft geben sie?
\item Welche Probleme und Herausforderungen können bei der Erfassung der Messparameter auftreten und wie können diese gelöst werden?
\end{enumerate}

\subsection{Lokale kinematische Messparameter}\label{sec:kinematische-messparameter}

Neben den GPS-Geolokalisationsinformationen werden in der STADTRADELN-App außerdem Messparameter mithilfe von Sensoren des Smartphones aufgezeichnet.

\begin{figure}[H]
\includegraphics[width=\linewidth, bb=0 0 540 181]{kinematische-messparameter.pdf}
\caption{Übersicht über die kinematischen Messparameter der STADTRADELN-Daten.}\label{fig:kinematische-messparameter}
\end{figure}

\noindent Wie in \Cref{fig:kinematische-messparameter} gezeigt, handelt es sich hierbei um jeweils triaxiale Messparameter in Form von Beschleunigungsdaten des Akzelerometers, Neigungsdaten des Gyrosensors, sowie Orientierungsdaten des Magnetometers.

\subsubsection{Funktionsweise}

Das Akzelerometer des Smartphones basiert auf einem mikroelektronischen mechanischem System \textit{(MEMS)}, welches die mechanische Beschleunigung des Smartphones in vom Mikrochip interpretierbare elektrische Signale wandelt \cite{matej_andrejasic_mems_2008, constantinescu_capacitive_2013}. Hierbei wird eine sehr kleine gefederte Masse, welche sich zwischen zwei Elektroden befindet, in Bewegung versetzt. Die Bewegungsänderung bewirkt eine Spannungsänderung am Kondensator, welche vom System als Beschleunigung interpretiert werden kann \cite{dey_accelprint_2014}. Um neben der Beschleunigung auch die Winkelauslenkung (Neigung) des Gerätes zu messen, wird ein Gyrosensor eingesetzt. Gyrosensoren bestehen traditionell aus einer rotierenden Masse, welche durch eine entsprechende Lagerung in Relation zum umgebenden Objekt statisch bleibt. Dies ist physikalisch begründet in der Drehimpulserhaltung\footnote{Die Drehimpulserhaltung besagt, dass in einem isolierten physikalischen System der Gesamtdrehimpuls gleich bleibt. Auf eine Ausrichtungsänderung des externen Systems reagiert eine rotierende Gyroskop-Masse mit einer entgegengesetzten Ausrichtungsänderung. \url{https://de.wikipedia.org/wiki/Kreiselinstrument} (Abgerufen am 7.4.2021)}. Die Auslenkung des umgebenden Objektes in Relation zur rotierenden Masse kann anschließend elektrisch gemessen werden \cite{singh_piezoelectric_2007}. Wegen der benötigten Größe des traditionellen Apparats wurde das Wirkprinzip mikroelektronisch auf der Grundlage von sogenannten Piezo-Elementen\footnote{Piezo-Elemente (piezo altgr. für \enquote{drücken}) reagieren mit einer elektrischen Oberflächenspannung bei einer Druckausübung (Piezoeffekt). \url{https://de.wikipedia.org/wiki/Piezoelement} (Abgerufen am 7.4.2021)} (zum Beispiel einem Quarzkristall) implementiert. Hierbei induziert die Auslenkung des Gerätes eine Spannung über den sogenannten Piezoeffekt \cite{singh_piezoelectric_2007, ichimura_fem_2002, koitabashi_improvement_2002}. Neben dem Gyrosensor und dem Akzelerometer wird zusätzlich auch das Magnetfeld über ein Magnetometer gemessen. Es handelt sich hierbei ebenfalls um ein mikroelektronisches Bauteil, welches über ein komplexes optisches Messsystem anhand von Alkalimetalldämpfen (zum Beispiel Rubidium oder Cäsium) das lokale Magnetfeld messen kann \cite{schwindt_chip-scale_2004, dmitry_budker_alkali_2021}. Anhand der Magnetfelddaten können die triaxialen Orientierungsdaten abgeleitet werden, über die Ausrichtung des Erdmagnetfelds.

\subsubsection{Herausforderungen}

Bei der Interpretation der Messparameter (Beschleunigung, Winkelauslenkung, Orientierung) kann die Ausrichtung des Smartphones zu Beginn und auch während der Messung in Relation zum Bezugssystem (beispielsweise in Bezug zum Erdboden) von entscheidender Bedeutung sein. So ist insbesondere bei Beschleunigungsdaten des Akzelerometers häufig die Erdschwerkraft im triaxialen Vektorsystem integriert \cite{ken_taylor_activity_2011, tundo_correcting_2013}. Bei einer Neigungsänderung des Smartphones wird somit auch immer eine Änderung des Gravitationsvektors gemessen. Dies erschwert die Interpretation der gemessenen Daten, da die gemessenen Werte zunächst in ein von der Neigung des Gerätes unabhängiges Bezugskoordinatensystem (in der Regel bezeichnet als \enquote{Weltkoordinatensystem}) gebracht werden müssen.

\begin{figure}[H]
\includegraphics[width=\linewidth, bb=0 0 425 162 ]{desync.pdf}
\caption{Schematische Darstellung der zeitlichen Desynchronisation von Sensordaten, angelehnt an \cite{matusek_anwendung_2019}.}\label{fig:desync}
\end{figure}

\noindent Ebenfalls erschwerend wirken sich von Gerät zu Gerät unterschiedliche inhärente Messfehler und -parameter der Sensoren aus. Wie in \Cref{fig:desync} gezeigt kann beispielsweise die Abtastrate der Sensoren und der genutzten Systemschnittstellen zur Aufzeichnung variieren oder eine verzögerte Aufzeichnung auftreten \cite{matusek_anwendung_2019}. Bei der Messung der Akzelerometer-Daten ist mit einem Rauschen der Sensorwerte zu rechnen \cite{dey_accelprint_2014, ravi_deep_2016}. Dies ist übergreifend zurückzuführen auf unterschiedliche Imperfektionen in der Fertigung und die altersbedingte Degradation von mikroelektronischen Chips \cite{matej_andrejasic_mems_2008, constantinescu_capacitive_2013, hillman_manufacturing_2004}. \cite{dey_accelprint_2014} konnten nachweisen, dass allein anhand des gerätespezifischen Rauschens identifiziert werden kann, welches Smartphone zur Aufzeichnung der Messwerte (in diesem Fall Akzelerometer-Daten) genutzt wurde. Die diesem Ergebnis zugrundeliegenden Rauschmuster müssen bei der späteren Datenvorverarbeitung berücksichtigt werden, um zu vermeiden, dass das zu portierende Machine-Learning Klassifikationssystem unerwünschte Muster lernt. Durch die zu antizipierende inhärente Abweichung der Sensordaten vom tatsächlichen Wert eignen sich die kinematischen Daten auch nur begrenzt für eine Bewegungsrekonstruktion und Lokalisation durch Aggregation im Raum. Speziell die Ausrichtungsdaten des Gyrosensors sind prädestiniert für die auch als \textit{Drift} bezeichnete Fehleraggregation \cite{takeda_drift_2014}. Neben diesen Herausforderungen besteht mit Hinblick auf die Orientierungsdaten ein weiteres Problem. Da das Erdmagnetfeld (ca. $30 \mu T$ bis $60 \mu T$) im Vergleich zu einigen künstlichen Magnetfeldquellen wie Kühlschränken (ca. $5 mT$) \cite{schirmer_smartphone_2016, moreno-torres_evaluation_2013} relativ schwach ist, kann der Magnetometer-Sensor leicht gestört oder gar dekalibriert werden \cite{schirmer_smartphone_2016, zhang_preliminary_2012}, was zu signifikanten Fehlern führen kann \cite{kunze_can_2010}.

\subsubsection{Signalverarbeitungsmethoden}

Zur Mitigation der beschriebenen Probleme wurden bereits verschiedene Methoden entwickelt.

\begin{figure}[H]
\includegraphics[width=\linewidth, bb=0 0 733 340]{frequenzpassfilter.pdf}
\caption{Anwendung von Frequenzbandfiltern auf Rohsensordaten des Akzelerometers. Datenquelle: Eigenes Sample eines iPhone XS-Akzelerometers.}\label{fig:frequenzpassfilter}
\end{figure}

\noindent Eventuell auftretendes Rauschen kann durch ein Tiefpassfilter respektive Hochpassfilter eliminiert werden. Hierbei eliminiert ein Tiefpassfilter hochfrequentes Rauschen und ein Hochpassfilter niederfrequentes Rauschen (siehe \Cref{fig:frequenzpassfilter}). \cite{nutter_design_2018} nutzen mehrere Filter für die Prozessierung von Beschleunigungsdaten, darunter auch ein Butterworth-Tiefpassfilter bei $20 Hz$. Bei dem Butterworth-Tiefpassfilter handelt es sich um ein spezielles Tiefpassfilter, welches zur Eliminierung von Frequenzen über einem Schwellenwert $\omega_{0}$ und zur Reduktion der Kurvenpunkte bei Erhaltung des Kurvenverlaufs genutzt werden kann \cite{zuricher_hochschule_fur_angewandte_wissenschaften_kapitel_2009}. \footnote{\url{https://de.wikipedia.org/wiki/Butterworth-Filter} (Abgerufen am 7.4.2021)}. \cite{takeda_drift_2014} verwenden ebenfalls ein Butterworth-Filter, hierbei für die Verarbeitung von Auslenkungsdaten des Gyrosensor. \cite{abdulla_measuring_2013} applizieren ein Tiefpassfilter bei $5 Hz$ auf Magnetometerdaten.

\begin{figure}[H]
\includegraphics[width=\linewidth, bb=0 0 724 340]{glaettingsfilter.pdf}
\caption{Anwendung von Glättungsfiltern auf Rohsensordaten des Akzelerometers. Datenquelle: Eigenes Sample eines iPhone XS-Akzelerometers.}\label{fig:glaettingsfilter}
\end{figure}

\noindent Neben Hoch- und Tiefpassfiltern besteht weiterhin auch die Möglichkeit, die Signale zu glätten, um hochfrequente Signalschwankungen zu eliminieren. Hierbei kommen Glättungsfilter zum Einsatz. \cite{matusek_anwendung_2019} behandelt das in \Cref{fig:glaettingsfilter} gezeigte und unter den Namen \textit{Moving Average} bekannte Filterverfahren, bei dem die umliegenden Datenpunkte (Länge $k$) mit dem zu glättenden Datenpunkt gemittelt werden. Es existieren zahlreiche weitere Glättungsverfahren, darunter auch das von \cite{nutter_design_2018} (hier parametrisiert mit $k = 3$) genutzte Median-Filter. Wie in \Cref{fig:glaettingsfilter} sichtbar wird das Eingangssignal mit einer von der Filterlänge abhängigen Intensität geglättet. Beim Beispielsignal aus \Cref{fig:glaettingsfilter} ist klar zu erkennen, dass hierbei jedoch auch bei einer zu hohen Filterlänge $k$ wichtige Informationen verloren gehen können.

\subsubsection{Fourier-Transformation}

Ein weiteres wichtiges Signalverarbeitungsverfahren ist die Fourier-Transformation\footnote{\url{https://de.wikipedia.org/wiki/Fourier-Transformation} (Abgerufen am 8.4.2021)}. \cite{stojanov_continuous_2020} wendet dieses Signalverarbeitungsverfahren auf den Sensordaten im Rahmen der Erzeugung des Inputvektors für das Machine-Learning-Modell an.

\begin{figure}[H]
\includegraphics[width=\linewidth, bb=0 0 734 339]{fft.pdf}
\caption{Anwendung einer Fast-Fourier-Transformation auf Rohsensordaten des Akzelerometers. Datenquelle: Eigenes Sample eines iPhone XS-Akzelerometers.}\label{fig:fft}
\end{figure}

\noindent Hierbei kommt die in \Cref{fig:fft} exemplarisch gezeigte Fast-Fourier-Transformation zum Einsatz. Die Fourier-Transformation ist ein mathematisches Verfahren, um zusätzlich zum Zeitlinienverlauf der Signale deren Frequenzspektrum zu ermitteln. Anhand des ermittelten Frequenzspektrums lässt sich sofort ablesen, welche Frequenzen besonders häufig oder sehr selten in einem Frequenzspektrum vorkommen, sowie die gemessene Maximal- und Minimalfrequenz. Dies ist nicht nur relevant für die Erkennung und Elimination von möglichen Rauschfrequenzen durch die evidenzbasierte Konfiguration des $w_0$-Parameters eingesetzter Hoch- oder Tiefpassfilter, sondern dient auch als zusätzliche Informationsquelle für das Machine-Learning-Modell.

\subsubsection{Sensorfusion}

Bei einer differenzierten Betrachtung der einzelnen Sensorsysteme lässt sich anhand der einschlägigen Fachliteratur beobachten, dass insbesondere die Beschleunigungsdaten des Akzelerometers als geeignete Grundlage für eine Aktivitätsklassifikation \cite{gudur_activeharnet_2019, nutter_design_2018} verwendet werden können. \cite{cai_touchlogger_2011, marquardt_spiphone_2011, aviv_practicality_2012} zeigen, dass Beschleunigungsdaten sogar für eine Rekonstruktion von Eingaben im Smartphone geeignet sind. Da in den Daten der STADTRADELN-App jedoch auch Auslenkungs- und Orientierungsdaten vorliegen, können die Beschleunigungsdaten für eine Aktivitätserkennung in einer geeigneten Form mit diesen verknüpft werden \cite{ravi_deep_2016, abdulla_measuring_2013, kunze_can_2010}. Mithilfe einer Fusion der Sensordaten, beispielsweise auf Grundlage von Madgwick-Filtern \cite{madgwick_estimation_2011} ist dies möglich. Speziell bei Beschleunigungsdaten kann hierdurch der inkludierte Gravitationsvektor eliminiert werden \cite{nutter_design_2018, tundo_correcting_2013}. Außerdem kann die Orientierung und Auslenkung des Gerätes in Relation zum Weltkoordinatensystem rekonstruiert werden, um schließlich alle Sensordaten in dasselbe Weltkoordinatensystem zu übermitteln.

\paragraph{Funktionsweise des Madgwick-Filter:} Das von \cite{matusek_anwendung_2019}, \cite{werner_kontinuierliche_2020} und \cite{stojanov_continuous_2020} verwendete Madgwick-Filter ist ein iteratives mathematisches Verfahren, mit dem eine Schätzung über die Lage eines Smartphones (oder anderer IMU\footnote{IMU steht für \textit{Inertial Measurement Unit}, einer Kombination aus Akzelerometer und Gyrosensor}-Systeme) in Form berechnet werden kann. Die resultierende Schätzung wird in Form eines Quaternion\footnote{\url{https://en.wikipedia.org/wiki/Quaternion} (Abgerufen am 8.4.2021)} angegeben. Die konkrete mathematische Beschreibung des Verfahrens kann in \cite{nitin_j_sanket_mathematical_2021, madgwick_estimation_2011} nachgelesen werden, zusammenfassend jedoch wird die Schätzung der Lage im Raum auf Grundlage der numerischen Lösung eines mathematischen Optimierungsproblems realisiert. Durch ein Gradientenabstiegsverfahren wird kontinuierlich versucht, den Fehler der Lagevorhersage anhand der tatsächlich gemessenen Daten zu minimieren. Gradientenabstiegsverfahren werden in \Cref{sec:machine-learning} näher erläutert, da sie zu den Machine-Learning-Methoden zählen. \\

\begin{table}[H]
  \begin{tabular}{cccccccccccccccccll}
    \rot{AQUA \cite{valenti_keeping_2015}}
    & \rot{Complementary}
    & \rot{Davenport \cite{paul_b_davenport_vector_1968}}
    & \rot{EKF \cite{sabatini_kalman-filter-based_2011}}
    & \rot{FAMC \cite{liu_simplified_2018}}
    & \rot{FLAE \cite{wu_super_2018}}
    & \rot{Fourati \cite{fourati_nonlinear_2011}}
    & \rot{FQA \cite{yun_simplified_2008}}
    & \rot{Integration}
    & \rot{\textbf{Madgwick} \cite{madgwick_estimation_2011}}
    & \rot{Mahony \cite{mahony_nonlinear_2008}}
    & \rot{OLEQ \cite{zhou_optimal_2018}}
    & \rot{QUEST \cite{shuster_three-axis_1981}}
    & \rot{ROLEQ \cite{zhou_optimal_2018}}
    & \rot{SAAM \cite{wu_super_2018}}
    & \rot{Tilt \cite{sebastian_trimpe_balancing_2012}}
    & \rot{TRIAD \cite{shuster_optimization_2007}}
    & \\
    \midrule
    \omark & \cmark & \cmark & \cmark & \cmark & \cmark & \cmark & \cmark & \xmark & \cmark & \cmark & \cmark & \cmark & \cmark & \cmark & \cmark & \cmark & ACC \\
    \cmark & \cmark & \xmark & \cmark & \xmark & \xmark & \cmark & \xmark & \cmark & \cmark & \cmark & \xmark & \xmark & \xmark & \xmark & \xmark & \xmark & GYR \\
    \omark & \omark & \cmark & \cmark & \cmark & \cmark & \cmark & \omark & \xmark & \omark & \omark & \cmark & \cmark & \cmark & \cmark & \omark & \cmark & MAG \\
    \bottomrule
  \end{tabular}\label{tab:ahrs}
  \caption{Lageschätzverfahren in der Übersicht, zusammen mit deren Abhängigkeit von Akzelerometerdaten (ACC), Gyrosensordaten (GYR) und Magnetometerdaten (MAG). Entnommen aus \cite{garcia_mayitzinahrs_2021}. Legende: Benötigt (\cmark), Optional (\omark), Nicht einbezogen (\xmark).}
\end{table}

\noindent Ein System, welches einen oder eine Kombination aus solchen Algorithmen nutzt, um die Lage des Smartphones im Weltkoordinatensystem zu rekonstruieren, wird auch als \textit{Attitude and Heading Reference System (AHRS)} bezeichnet. \\

\subsubsection{Synchronisierung und Alignment}

Zur Synchronisierung der möglicherweise von Gerät zu Gerät unterschiedlichen Abtastraten oder von asynchron aufgezeichneten Daten unterschiedlicher Sensoren ist ein Up- oder Downsampling mithilfe einer Interpolation der Datenpunkte möglich \cite{matusek_anwendung_2019, werner_kontinuierliche_2020, stojanov_continuous_2020}.

\begin{figure}[H]
\includegraphics[width=\linewidth, bb=0 0 410 171]{desync-solution.pdf}
\caption{Schematische Darstellung des Alignments von desynchronisierten Datenpunkten, angelehnt an \cite{matusek_anwendung_2019}.}\label{fig:desync-solution}
\end{figure}

\noindent \cite{matusek_anwendung_2019} beschreibt in seinem Konzept, wie desynchronisierte Datenpunkte wieder synchronisiert werden können. \Cref{fig:desync-solution} zeigt das Verfahren schematisch. Die mit Zeitstempeln versehenen Datenpunkte werden in ein festes Abtastraster gebracht. Dies gelingt über zwei primäre Schritte. Am zeitlichen Rand der Messung werden zunächst Datenpunkte verworfen, zu denen kein Wertepaar aus allen drei Sensoren gebildet werden kann. Ein Wertepaar lässt sich erst dann bilden, wenn alle Sensoren mindestens einen Wert erfasst haben, sowie am Ende der Messung vice versa, wenn alle Sensoren noch weitere Daten gemessen haben. Das so gefundene Fenster wird anschließend im zweiten Schritt anhand der vorliegenden Datenpunkte pro Sensormessung interpoliert, um den diskreten Punktdatensatz in eine kontinuierliche Näherungsfunktion zu überführen. Die Abtastzeitpunkte $t_s ... t_{s+n}$ (in \Cref{fig:desync-solution} als $\oplus$ gekennzeichnet) können anschließend für alle drei Messverläufe beliebig gewählt und errechnet werden.

\begin{figure}[H]
\includegraphics[width=\linewidth, bb=0 0 505 175]{interpolation.pdf}
\caption{Schematische Darstellung verschiedener Interpolationsverfahren.}\label{fig:interpolation}
\end{figure}

\noindent Hierbei stehen verschiedene Interpolationsverfahren zur Verfügung, die in \Cref{fig:interpolation} gezeigt sind. Die linke Seite der Abbildung zeigt ein sehr einfaches Interpolationsverfahren, bei dem die Distanz auf der Zeitachse (oder die euklidische Distanz im zweidimensionalen Raum) herangezogen wird, um den nächsten Nachbarn zu bestimmen und dessen Wert für die Interpolation des Sample-Punktes zu nutzen. Komplexere Interpolationsverfahren wie die lineare Interpolation oder die Spline-Interpolation sind rechenaufwändiger, bilden jedoch den Kurvenverlauf präziser nach. Die Spline-Interpolation ist am besten dazu in der Lage, komplexe Kurvenverläufe nachzubilden, tendiert jedoch zum \enquote{Überschwingen} durch die Constraints der Spline-Gleichung, vor allem bei stark verrauschten Daten. Daher bietet es sich an, wie in \cite{matusek_anwendung_2019} eine lineare Interpolation für die Synchronisation und das Alignment der Datenpunkte zu nutzen.

\subsection{Geolokalisationsinformationen}

Neben den lokalen kinematischen Messparametern werden in der STADTRADELN auch analog zur Datenschutzerklärung GPS-Daten erfasst. Das \textit{Navigational Satellite Timing and Ranging - Global Positioning System}, kurz \textit{NAVSTAR GPS} oder auch nur \textit{GPS} ist ein System aus (Stand 9. Januar 2021) 31 Satelliten, welche die Erde in einer Höhe von ca. 20.000 km umkreisen \cite{us_space_force_gpsgov_2021}.

\subsubsection{Terminologische Einordnung}

Der Begriff GPS wird teils als Synonym für globale Satellitennavigationssysteme (engl. \textit{Global Navigation Satellite System GNSS}) verstanden. Neben dem amerikanischen GPS existieren hierbei jedoch noch weitere Systeme, wie das europäische Galileo, das russische Glonass, sowie das chinesische BeiDou \cite{olynik_temporal_2002}. Moderne Smartphone-Geolokalisationssysteme nutzen in der Regel eine Kombination dieser globalen Satellitennavigationssysteme \cite{european_gnss_service_centre_is_2021, european_gnss_service_centre_usegalileo_2021}. In Abhängigkeit der genutzten Smartphone-Schnittstellen ist es also möglich, dass die in der Datenschutzerklärung der STADTRADELN-App beschriebenen GPS-Daten von verschiedenen Satellitennavigationssysteme stammen.

\subsubsection{Funktionsweise}

Wie alle GNSS-Satelliten besitzen GPS-Satelliten sehr präzise Atomuhren und senden über verschiedene militärische und zivile Frequenzen kontinuierlich ihre Position und Zeit zur Erde \cite{us_space_force_global_2020}. Mithilfe von speziellen Antennen können diese Signale empfangen werden, wobei die empfangenen Signale amplifiziert und weiter verarbeitet werden müssen \cite{jan_van_sickle_gps_2021}. Als Ergebnis der internen Prozessierung des GPS-Signals kann die Distanz zum GPS-Satelliten anhand der Signallaufzeit unter Berücksichtigung verschiedener Fehlerquellen polynomiell berechnet werden \cite{olynik_temporal_2002, sameet_mangesh_deshpande_study_2004}. Somit stehen anschließend die Position des Satelliten und die berechnete Distanz zur Verfügung.

\begin{figure}[H]
\includegraphics[width=\linewidth, bb=0 0 565 226]{gps-funktionsweise.pdf}
\caption{Das Funktionsprinzip der GNSS-basierten Geolokalisation und zu erwartende Fehlerquellen.}\label{fig:gps-funktionsweise}
\end{figure}

\noindent \Cref{fig:gps-funktionsweise} zeigt die schematische Funktionsweise von GNSS-basierter Geolokalisation. Bei einem Kontakt zu mindestens 4 GNSS-Satelliten kann der Ort des Empfängers (Höhe, Breitengrad, Längengrad) ermittelt werden, indem eine Intersektion von 4 Sphären durchgeführt wird, wobei das Zentrum der jeweiligen Sphäre die transmittierte Position des GNSS-Satelliten ist, und der Radius der Sphäre sich durch die berechnete Distanz bestimmt \cite{zhang_senstrack_2013}. Die auch Pseudodistanz genannte Strecke wird über den transmittierten Zeitstempel des Satelliten und die lokale Zeit des Smartphones berechnet, indem die Differenz der beiden Zeitstempel mit der Lichtgeschwindigkeit und fehlerkorrigierenden Faktoren verrechnet wird\footnote{\url{https://www.e-education.psu.edu/geog862/node/1759} (Abgerufen am 7.4.2021)}. Die satellitengestützte Bestimmung ist hierbei passiv, da keine aktive Kommunikation zum Satelliten stattfindet \cite{kaplan_understanding_2005}. Smartphones nutzen zusätzlich zur satellitengestützten Positionsbestimmung Eigenschaften des WiFi- und GSM-Funknetzes, um die Präzision der Positionsbestimmung zu verbessern und den Energieverbrauch zu reduzieren \cite{zhang_senstrack_2013}. Als Ergebnis der hybriden Positionsbestimmung stellt das Smartphone-Geolokalisationssystem mehrere Messparameter zur Verfügung, darunter die Ortsinformationen (Höhe, Breitengrad, Längengrad), abgeleitete Informationen wie die Geschwindigkeit und eine Genauigkeitsangabe.

\subsubsection{Herausforderungen}

Die von den jeweiligen Satelliten (über elektromagnetische Wellen) ausgesendeten GPS-Informationen unterliegen dem physikalischen Abstandsgesetz.

\begin{equation}\label{eq:abstandsgesetz}
I_{Signal} \propto \frac{1}{d^{2}}
\end{equation}
wobei:
\begin{conditions}
  I_{Signal} & Intensität des GPS-Signals  \\
  d & Distanz zwischen GPS-Satellit und Empfänger
\end{conditions}

\noindent Das in \Cref{eq:abstandsgesetz} gezeigte Abstandsgesetz bedingt (neben weiteren signaldämpfenden Faktoren wie atmosphärischem Scattering, Streuung und Brechung an Gebäuden), dass wegen der begrenzten Signalemissionsstärke des Satelliten und der großen Distanz zum Empfänger die Empfindlichkeit des Empfängersystems verhältnismäßig hoch sein muss, in einer Größenordnung von $-130 dBm$ (Dezibel Milliwatt) \cite{sameet_mangesh_deshpande_study_2004}. Die elektronische Prozessierung des Signals erfordert hierdurch eine entsprechende elektrische Leistung \cite{jan_van_sickle_gps_2021}. Die erforderliche elektrische Leistung stellt für Smartphones durch deren Limitation durch die Kapazität des Akkus ein signifikantes Problem dar \cite{zhang_senstrack_2013, oshin_improving_2012, constandache_enloc_2009, zhuang_improving_2010}. Eine energiesparendere Geolokalisation über GSM- und WiFi-Funknetze ist weniger akkurat als GPS und kann daher in Anwendungsfällen mit einer benötigten Mindestpräzision lediglich begleitend eingesetzt werden \cite{paek_energy-efficient_2010}. Aufgrund der Abhängigkeit zum Satellitensignal (respektive zu GSM- und WiFi-Funknetzen) sind weitere Probleme die Ortsabhängigkeit \cite{zhang_senstrack_2013, kaplan_understanding_2005}, die Wetterabhängigkeit \cite{paek_energy-efficient_2010}, sowie mögliche Interferenzen und andere Signaltransmissionsstörungen \cite{olynik_temporal_2002, sameet_mangesh_deshpande_study_2004}. Bei der Berechnung der Distanz zum Smartphone müssen unter anderem iono- und troposphärische Ladungseffekte auf die Propagierungsgeschwindigkeit der elektromagnetischen Funksignale berücksichtigt werden, diese schwanken jedoch je nach Tageszeit und Sonnenaktivität\footnote{\url{https://gssc.esa.int/navipedia/index.php/Ionospheric_Delay} (Abgerufen am 7.4.2021)}. Im Rahmen der Anwendungsentwicklung sowie der Entwicklung einer Aktivitätserkennung auf Grundlage der Geolokalisationsinformationen sind diese Probleme in Form einer Korrektur fehlerhafter Datenpunkte zu berücksichtigen, insbesondere wegen der ständig variierenden Präzision \cite{zhang_senstrack_2013, matusek_anwendung_2019}.

\subsubsection{Fehlerkorrektur der GNSS-Datenpunkte}

Grundsätzlich besteht die Möglichkeit, GNSS-Datenpunkte mithilfe von vorliegendem Kartenmaterial und den darin enthaltenen Straßen zu korrigieren. Dieses Verfahren wird auch \textit{Map Matching} genannt \cite[S. 26]{reto_wick_unsicherheiten_2013}. Ein Problem des Verfahrens ist jedoch die Notwendigkeit sowie die Aktualität des Kartenmaterials. Zur Korrektur der GNSS-Datenpunkte ohne solches Kartenmaterial stellt \cite[S. 58]{matusek_anwendung_2019} ein konkretes Verfahren vor, welches auf Grundlage der in \Cref{sec:kinematische-messparameter} erläuterten kinematischen Messparameter die gemessene Bewegung im Raum einbezieht \cite{oleg_katkov_how_2018}. Die Idee des Verfahrens ist, den Fehler der GNSS-Datenpunkte über die AHRS-gestützte Rekonstruktion der Bewegung im Raum zu korrigieren. Hierzu wird im von \cite{oleg_katkov_how_2018} beschriebenen Verfahren das Madgwick-Filter genutzt, um schließlich eine Lageschätzung (und Wegschätzung) im Weltkoordinatensystem vorzunehmen. Anschließend wird ein Kalman-Filter \cite{welch_introduction_1997} angewandt, um die korrigierte GNSS-Position zu schätzen und somit mögliche Abweichungen der GNSS-Position von der tatsächlichen Position zu verringern. Bei dem Kalman-Filter handelt es sich um ein iteratives mathematisches Verfahren zur Zustandsschätzung. Hierbei wird ein aktuelles Zustandsmodell gespeichert, für den jeweils nächsten iterativen Schritt wiederverwendet und je nach auftretendem Fehler angepasst. Das Verfahren ähnelt somit dem Madgwick-Filter in der Funktionsweise. Auch hier dient der aktuelle Zustand zur Schätzung des tatsächlichen Wertes anhand des Inputs, hierbei der jeweilige GNSS-Datenpunkt und die kinematischen Messparameter. Das vorliegende Problem (die Schätzung des tatsächlichen Ortes) ist ein klassischer Anwendungsfall des Kalman-Filters, denn die präzise Messung des Ortes ist wegen des Rauschens des GNSS-Signals (oder bei Ausfall) nur mit einer geringen Wahrscheinlichkeit möglich. Stattdessen streut der gemessene Ort um den tatsächlichen Ort. Das konkrete Streuverhalten korreliert mit der Beschaffenheit der Umgebung des Smartphones, insbesondere mit umgebenden Störquellen wie Gebäuden \cite{reto_wick_unsicherheiten_2013}. Eine Standardnormalverteilung kann jedoch zur näherungsweisen Modellierung des GNSS-Streuverhaltens herangezogen werden \cite{laube_how_2011}. Das Kalman-Filter eignet sich speziell für diesen Anwendungsfall, die tatsächliche Position bei einer normalverteilt streuenden gemessenen GNSS-Position zu schätzen. Ziel ist eine korrigierte GNSS-Trajektorie, deren Abweichung von der tatsächlich zurückgelegten Strecke deutlich geringer ist.

\subsubsection{Optimierung des Energieverbrauchs}

Zur Verbesserung der hybriden Geolokalisation über GPS-, GSM- und WiFi-Signale wurden verschiedene Konzepte entwickelt. Der Energieverbrauch kann signifikant reduziert werden, indem die GPS-Abtastrate gesenkt wird \cite{constandache_enloc_2009, lu_jigsaw_2010}. Hierzu können zusätzliche Informationen des Akzelerometers \cite{oshin_improving_2012} und des Orientierungssensors \cite{zhang_senstrack_2013} genutzt werden, um die GPS-Abtastrate adaptiv zu senken. Außerdem kann die erforderliche Präzision der Geolokalisation zusammen mit der primär genutzten Methode (GPS, GSM, WiFi) adaptiv angepasst werden \cite{lin_energy-accuracy_2010, zhang_senstrack_2013}. Aktuelle Schnittstellen in iOS- und Android-Smartphones greifen auf diese Prinzipien zurück und bieten dem Anwendungsentwickler Möglichkeiten, die benötigte Präzision einzustellen, um den Energieverbrauch zu reduzieren \cite{google_inc_fused_2021, apple_inc_desiredaccuracy_2021}.

\subsection{Normalisierung und Standardisierung der Daten}

\todo{Normalisierung motivieren -> Wichtig für Lernparameter}
\todo{Unterschied Normalisierung und Standardisierung erklären}
\todo{Konkrete Wertebereiche aufstellen}


% Weitere Probleme: Datenschutz und -sicherheit
% -> GPS Daten sind unter Umständen sehr privat
% -> Accelerometer-Daten können aber auch zur Personenidentifikation genutzt werden [Tundo et al.]
% (Nicht behandelt in dieser Arbeit)

% Daten stellen eine Art der Kontextsensitivität dar
% Kontextsensitive Adaption: Messung, Integration, Adaption
% -> Engineering Ubiquitous Systems lesen!
% Klassifikation der gemessenen Daten fällt somit in Integration
