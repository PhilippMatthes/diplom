\chapter{Verwandte Arbeiten}\label{ch:verwandte-arbeiten}

In diesem Kapitel wird näher auf die wissenschaftlichen Arbeiten eingegangen, deren Forschungsthema sich mit dem dieser Arbeit stark überschneidet. Dazu wird zunächst ein taxonomischer Überblick über Kategorien gegeben, in welche sich verwandte Arbeiten eingliedern lassen. Anschließend werden die Konzepte und Ergebnisse näher diskutiert und verglichen.

\section{Forschungsüberblick}

Der Forschungsbereich Machine Learning erhält aktuell eine große Aufmerksamkeit, allein in 2020 wurden geschätzt 25800 Publikationen mit dem Begriff \enquote{Machine Learning} im Titel veröffentlicht \footnote{Quelle: Google Scholar, erweiterte Suchfunktion. Teile dieser Ergebnisse können auch wiss. Artikel, Patente oder ähnliche Veröffentlichungen ohne garantierte wiss. Qualität (z.B. durch Peer-Review) sein.}. Ein Teil dieser Arbeiten fokussiert sich auf den Fachbereich HAR, \cite{demrozi_human_2020} zählten in 2020 insgesamt 149 wissenschaftliche Publikationen, davon 53 mit Fokus auf Deep Learning (unter anderem künstliche neuronale Netzwerke) und 96 mit Fokus auf konventionellen Machine-Learning-Ansätzen wie Random Forests, Support Vector Machines, k-Means-Clustering und weiteren. Dabei sind die auf Deep Learning basierenden Ansätze in der Erkennungsqualität nur marginal überlegen, mit einer durchschnittlichen Trefferquote von $93.0\%$ gegenüber $92.2\%$. Die meisten der in \cite{demrozi_human_2020} analysierten Arbeiten nutzen Akzelerometerdaten, gefolgt von Gyrosensordaten, Magnetometerdaten und weiteren Sensoren. Um die verwandten Arbeiten aus den über 100 Publikationen (neben relevanten Metaanalysen) zu extrahieren, wurde systematisch nach bestimmten Kriterien gesucht und die Referenzen gefundener Arbeiten weiterverfolgt (\textit{Snowballing}). Insbesondere sollten die Arbeiten hierbei neben der Realisation einer Aktivitätserkennung über Machine-Learning-Modelle nach Möglichkeit auch auf Smartphones ausführbar oder für diese optimiert sein. Als weiteres Kriterium sollten die Arbeiten auch den Spezialfall der Verkehrsmittelerkennung aus dem HAR-Forschungsbereich behandeln. Dies ist jedoch optional, da sich die Arbeiten auch in der Regel flexibel auf diesen Spezialfall überführen lassen. In der nachfolgenden Tabelle ist eine Auswahl der so gefundenen Arbeiten gegenübergestellt und in verschiedene Kategorien eingeordnet. In den unteren zwei Sektionen der Tabelle befinden sich Ansätze, welche eine Verkehrsmittelklassifikation über Machine-Learning-Modelle realisieren, ohne konkreten Fokus auf eine Portierung oder die Lauffähigkeit des Ansatzes auf Edge-Geräten. Unter anderem ordnen sich hier auch die bekannten Movebis-Ansätze ein. Darüber befinden sich Ansätze, welche einen allgemeineren Klassifikationsschwerpunkt (HAR) ohne Fokus auf Verkehrsmittel haben, gleichzeitig jedoch für Smartphones ausgelegt wurden. Im obersten Teil der Tabelle sind zwei Ansätze gezeigt, welche eine Verkehrsmittelklassifikation auf Smartphones über Machine-Learning-Modelle realisieren, sowie eine Arbeit, die nochmals limitiertere Edge-Geräte (Smartwatches) betrachtet.

\begin{landscape}
  \begin{table}[]
  \resizebox{\textwidth}{!}{%
  \begin{tabular}{@{}llp{3cm}p{2cm}p{7cm}p{3cm}p{3cm}p{3cm}p{4cm}@{}}
  \toprule
    Arbeit &
    Kat. &
    Labels &
    Daten &
    Preprocessing &
    Features &
    Modell(e) &
    Regularisierung &
    Training \\ \midrule
  \cite{bhattacharya_smart_2016} &
    HAR-W &
    Mehrere Varianten &
    ACC &
    Segmentierung, Abs. Signalstärke, FFT &
    Shallow &
    RBM &
    Keine Angabe &
    Supervised Learning \\ \midrule
  \cite{hemminki_accelerometer-based_2013}  &
    VME-S &
    Train, Bus, Stationary, Metro, Tram, Car &
    ACC &
    Tiefpass, Segmentierung, Elimination der Gravitation, Berechnung der Features &
    Shallow &
    HMM und DT &
    Limitierung der Modellgröße &
    Ensemble Learning \\
  \cite{byon_real-time_2014} &
    VME-S &
    Walk, Bike, Automobile, Bus, Car &
    GNSS, ACC, MAG &
    k.A. &
    Vorverarbeitete Daten &
    FFN &
    Keine Angabe &
    Supervised Learning \\ \midrule
  \cite{mairittha_-device_2021} &
    HAR-S &
    12 Aktivitäten &
    GYR, ACC &
    Abs. Signalstärke, Segmentierung &
    Vorverarbeitete Daten &
    RNN-LSTM &
    Keine Angabe &
    Transfer Learning, On-Device Fine-Tuning \\
  \cite{mairittha_improving_2020} &
    HAR-S &
    6 Aktivitäten &
    GYR, ACC &
    Abs. Signalstärke, Segmentierung &
    Vorverarbeitete Daten &
    CNN &
    Dropout &
    Supervised Learning, On-Device Fine-Tuning \\
  \cite{mairittha_-device_2019} &
    HAR-S &
    6 Aktivitäten &
    ACC &
    Segmentierung, Feature-Berechnung &
    Shallow &
    RNN-LSTM &
    Keine Angabe &
    Supervised Learning \\
  \cite{nutter_design_2018} &
    HAR-S &
    9 Aktivitäten &
    GYR, ACC &
    Glättung, Hochpass, Tiefpass, Elimination der Gravitation &
    Non-Shallow &
    CNN &
    Dropout, Batch-Normalization &
    Transfer Learning \\
  \cite{zebin_design_2019} &
    HAR-S &
    5 Aktivitäten &
    GYR, ACC &
    Segmentierung, Windowing, Normalisierung &
    Vorverarbeitete Daten &
    CNN &
    Dropout, Batch-Normalization &
    Supervised Learning \\
  \cite{ravi_deep_2016} &
    HAR-S &
    Mehrere Varianten &
    GYR, ACC &
    Feature-Berechnung, FFT &
    Non-Shallow &
    CNN &
    Weight Decay, Momentum, Dropout &
    Supervised Learning \\ \midrule
  \cite{liang_convolutional_2017} &
    VME &
    Bike, Car, Walk, Train, Metro, Bus, Stationary &
    ACC &
    Elimination der Gravitation, Gleitfenster, Abs. Signalstärke &
    Vorverarbeitete Daten &
    CNN &
    L2-Regularisierung &
    Supervised Learning \\
  \cite{friedrich_transportation_2019} &
    VME &
    Stationary, Walk, Run, Bike, Car, Bus, Train, Metro &
    ACC, GYR, MAG, BAR &
    Datensatzgewichtung, Abs. Signalstärke, Normalisierung &
    Vorverarbeitete Daten &
    RNN-LSTM &
    Dropout &
    Supervised Learning \\ \midrule
  \cite{werner_kontinuierliche_2020} &
    VME &
    Walk, Bike, Car, Bus, Tram, Train &
    GNSS, ACC, ROT, MAG &
    Skalierung, Interpolation, AHRS, Glättung &
    Vorverarbeitete Daten &
    RNN-LSTM und Postprocessing &
    Dropout, Early Stopping &
    Supervised Learning, Validierung mith. Semi-Supervised Learning \\
  \cite{stojanov_continuous_2020} &
    VME &
    Walk, Bike, Car, Bus, Tram, Train, Boat &
    GNSS, ACC, ROT, MAG &
    AHRS, Normalisierung, Alignment, Abs. Signalstärke, Downsampling, FFT, Gleitfenster, Segmentierung &
    Shallow &
    CNN und Postprocessing &
    Dropout, Early Stopping &
    Supervised Learning \\
  \cite{matusek_anwendung_2019} &
    VME &
    Walk, Bike, Car, Bus, Tram, Train &
    GNSS, ACC, ROT, MAG &
    Interpolation und Alignment, GPS-Korrektur, Rauschfilterung über Moving Average, Zeitfenster &
    Vorverarbeitete Daten &
    FFN (Kernkonzept) &
    Early Stopping &
    Supervised Learning \\ \bottomrule
  \end{tabular}\caption{\footnotesize{Forschungsüberblick. Abkürzungen wie folgt. Kategorien: VME (Verkehrsmittelerkennung), VME-W (VME auf Wearables), VME-S (VME auf Smartphones), HAR (Human Activity Recognition), HAR-S (HAR auf Smartphones). Daten: ACC (Akzelerometer), GYR (Gyrosensor), MAG (Magnetometer), BAR (Barometer - Luftdruck). Preprocessing: FFT (Fast Fourier Transformation). Modelle: RBM (Restricted Boltzmann Machine), HMM (Hidden Markov Model), DT (Decision Tree), FFN (Feed-Forward-Network), RNN-LSTM (Recurrent Neural Network mit Long-Short-Term-Memory-Neuronen), CNN (Convolutional Neural Network).}}%
  }\label{tab:forschungsueberblick}
  \end{table}
\end{landscape}

\noindent Auf Machine-Learning basierende HAR-Ansätze wie \cites{chen_deep_2015, ravi_deep_2017, kwapisz_activity_2011, jahangiri_applying_2015, nurhanim_classification_2017, zeng_convolutional_2014, abu_alsheikh_deep_2015, inoue_deep_2018}, die keinen expliziten Fokus auf Edge-Deployment oder die Verkehrsmittelerkennung besitzen, wurden in der Tabelle nicht berücksichtigt.
