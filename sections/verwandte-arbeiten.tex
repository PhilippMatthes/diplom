\chapter{Verwandte Arbeiten}\label{ch:verwandte-arbeiten}

Für einen praxisorientierten Überblick über generelle Strategien und grundlegende Konzepte im Machine-Learning eignet sich das Sachbuch \enquote{Praxiseinstieg Machine Learning mit Scikit-Learn und TensorFlow: Konzepte, Tools und Techniken für intelligente Systeme} \cite{geron_praxiseinstieg_2018}, sowie mit Fokus auf die App-seitige Entwicklung von Machine-Learning-Konzepten speziell für iOS-Apps \enquote{Machine Learning with Swift: Artificial Intelligence for iOS} \cite{sosnovshchenko_machine_2018}. Anhand des Sachbuchs \enquote{Datenanalyse mit Python: Auswertung von Daten mit Pandas, NumPy und IPython} \cite{mckinney_datenanalyse_2018} werden konkrete datenanalytische Methoden erläutert, welche zur Voranalyse der Datensätze angewandt werden können. \cite{zhang_deep_2019} zeigen darüberhinaus verschiedene Anwendungsbereiche von Machine-Learning im Fachgebiet \textit{Mobile and Wireless Networking} und geben einen extensiven Einblick über Grundlagen und die im Machine-Learning angewandten Taxonomien und Termini.
\\

\noindent Das zu erstellende Machine-Learning-Konzept lässt sich in den Forschungsbereich der \textit{Human Activity Recognition} (Akronym HAR) einordnen, dessen Ziel die Ermittlung und der Einsatz von Algorithmen für die Erkennung von Aktivitäten einer Person ist. \cite{bulling_tutorial_2014} geben einen progressiven und forschungsorientierten Einstieg in dieses Fachgebiet. Zur Realisation von HAR wurden bereits verschiedene Machine-Learning-Ansätze konzipiert und evaluiert, unter der Anwendung von Support Vector Machines \cite{nurhanim_classification_2017}, mehrschichtigen Perzeptronen \cite{kwapisz_activity_2011}, Convolutional Neural Networks \cite{zeng_convolutional_2014, chen_deep_2015}, Recurrent Neural Networks \cite{inoue_deep_2018} und hybriden Machine-Learning-Systemen \cite{abu_alsheikh_deep_2015, ravi_deep_2017}. Zudem wurden verschiedene Studien durchgeführt, in denen konkrete Machine-Learning-Ansätze bezüglich ihrer Performance bei HAR verglichen wurden. \cite{jahangiri_applying_2015} verglichen beispielsweise Support Vector Machines und Random Forests.
\\

\noindent Da das zu erstellende Machine-Learning-Konzept die Restriktion besitzt, dass dieses auf einem Smartphone lauffähig sein soll, sind die obigen Ansätze wegen der hardware- und softwaretechnischen Limitierungen nur teilweise applizierbar. \cite[S. 13]{ota_deep_2017} separieren daher die Deployment-Modelle des Deep Learning (Teilgebiet des Machine-Learning) auf mobilen Geräten in zwei Teilbereiche:

\begin{itemize}
\item \textit{Client-Server-Deployment}, wobei die Inferenz auf dem Server-Backend geschieht, analog zum aktuellen Deployment-Konzept der STADTRADELN-App in Kooperation mit Movebis
\item \textit{Client-Only-Deployment}, wobei die Inferenz vom Client selbst übernommen wird
\end{itemize}

\noindent Die bestehenden Arbeiten, welche mithilfe von Machine-Learning-Ansätzen das Filtering der STADTRADELN-Daten im Movebis-Projekt ermöglichen, gliedern sich in das Client-Server-Deployment ein, da sie nicht in der mobilen App integriert werden, sondern auf die Translokation der Datenpakete zum Server angewiesen sind. \cite{matusek_anwendung_2019} wendet für die Klassifikation der Datenpakete ein hybrides Klassifikationskonzept aus Support Vector Machine und Random Forest an und diskutiert gleichzeitig wesentliche Schritte, welche bei der Vorverarbeitung der Datensätze zu beachten sind. \cite{stojanov_continuous_2020} sowie \cite{werner_kontinuierliche_2020} bauen auf diesem Grundkonzept auf und implementieren die Klassifikation über ein Convolutional Neural Network \cite{stojanov_continuous_2020} respektive einem Recurrent Neural Network \cite{werner_kontinuierliche_2020}.
\\

\noindent Im Unterschied zu diesen vorliegenden Konzepten liegt die Problemstellung dieser Arbeit auf dem Deployment-Modell Client-Only-Deployment. Dennoch können beide Ansätze koexistierend eingesetzt werden. \cite{ma_survey_2019} gibt einen Einblick über die möglichen Herausforderungen der Ressourcenlimitierung, hierbei mit Fokus auf IoT-Applikationen. Im Kontext von HAR auf mobilen Geräten (nach dem Prinzip des Client-Only-Deployment) existieren weitere Metaanalysen, darunter \cite{martin_activity_2013}, welche mit Bezug auf Hardwareressourcen leichtgewichtige Machine-Learning-Ansätze vergleichen. \cite{nan_deep_2019} zeigen, wie mithilfe von verschiedenen Kompressionsmethoden die Ressourcenauslastung reduziert werden kann. \cite{chen_deep_2020} diskutieren neben dem State-of-the-Art von HAR über Machine-Learning im Client-Only-Deployment auch zukünftige Herausforderungen.
\\

\noindent Es existieren bereits mehrere konkrete Machine-Learning-Ansätze für HAR, welche für eine Ausführung auf Smartphones geeignet sind. Neben der Art und Struktur der Daten unterscheiden sich diese Ansätze wesentlich in der Architektur des jeweiligen Machine-Learning-Klassifikators. Hierbei wurden neben klassischen Ansätzen wie K-Nearest-Neighbors \cite{ken_taylor_activity_2011} vor allem aktuelle Machine-Learning-Methoden aus dem Bereich Deep Learning getestet, darunter Deep Bayesian Neural Networks \cite{gudur_activeharnet_2019}, Deep Feed-Forward Neural Networks \cite{li_deep_2020}, Restricted Boltzmann Machines \cite{bhattacharya_smart_2016, radu_towards_2016}, Recurrent Neural Networks mit long short-term memory \cite{mairittha_-device_2019, mairittha_-device_2021}, Convolutional Neural Networks \cite{mairittha_improving_2020} sowie hybride Ansätze \cite{ravi_deep_2017, nutter_design_2018}.
\\

\noindent Diese Arbeiten zeigen verschiedene Formen der Datenvorverarbeitung, darunter beispielsweise die Vorverarbeitung von Sensordaten über Kurzzeit-Fourier-Transformation und Spektrogrammen für Convolutional Neural Networks \cite{ravi_deep_2016, ravi_deep_2017}, oder auch die Nutzung des Verfahrens \textit{Principal Component Analysis} \cite{nutter_design_2018}. Zur Ausgabenverarbeitung der Machine-Learning-Systeme existieren weitere verschiedene Ansätze, häufig wird eine Confusion Matrix zur Quantifizierung der Performance angewandt. \cite{huo_uncertainty_2020} zeigen eine Möglichkeit, wie unter anderem hierüber eine Vorhersage darüber getroffen werden kann, wie sicher eine Inferenz auf den tatsächlichen Wert zutrifft. Hiermit ließe sich im Rahmen des Gesamtkonzeptes dieser Arbeit der durch das Machine-Learning-Konzept realisierte Filter optional je nach Unsicherheit hinzuschalten oder abschalten.
