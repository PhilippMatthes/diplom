\chapter{Einleitung}\label{ch:einleitung}\pagenumbering{arabic}

\section{Gegenstand und Motivation}\label{sec:gegenstand-und-motivation}

% Darstellung des Kontextes der Arbeit

STADTRADELN\footnote{\url{https://www.stadtradeln.de/darum-geht-es} (Abgerufen am 25.2.2021)} ist ein Projekt, bei dem Radfahrende teilnehmen können, um kollektiv mit dem Rad gefahrene Kilometer für den Klimaschutz zu sammeln. Begleitet wird das Projekt durch eine mobile App, mithilfe derer Nutzende ihre gefahrenen Kilometer tracken und sich mit anderen Teilnehmenden vergleichen können. Im Rahmen des Forschungsprojektes Movebis\footnote{\url{https://www.movebis.org/das-projekt/} (Abgerufen am 25.2.2021)} werden die hierbei erhobenen Daten ausgewertet und mithilfe von konkreten Kenngrößen visuell aufbereitet. Die aufbereiteten Daten werden der kommunalen Verkehrsplanung anschließend zur Verfügung gestellt, mit dem Ziel, die Planung der Radverkehrsinfrastruktur zu verbessern.

% Fokussierung auf ein konkretes Teilgebiet des Kontextes + Motivation

Die von der STADTRADELN-App erhobenen Daten\footnote{\url{https://www.stadtradeln.de/datenschutz} (Abgerufen am 25.2.2021)} werden zu einem Server hochgeladen und von diesem für die spätere Auswertung protokolliert. Hierbei können Nutzende selbst entscheiden, zu welchen Zeitpunkten die Datenerfassung gestartet und beendet wird. Hierbei existiert nach aktuellem Stand kein Kontrollmechanismus, mithilfe dessem verifiziert werden kann, dass sich ein Nutzender noch mit dem Fahrrad bewegt. Daher ist nicht ausgeschlossen, dass Datenpakete außerhalb des gewünschten Zielkontextes zum Server übermittelt werden. Im Movebis-Projekt wurden daher verschiedene Machine-Learning-Modelle entwickelt, durch deren Einsatz für den Zielkontext plausible von unplausiblen Datensätzen unter Berücksichtigung einer bestimmten Fehlerrate getrennt werden können. Die entwickelten Machine-Learning-Modelle ermöglichen ein Filtering der Datensätze zum Zeitpunkt der Auswertung, reduzieren jedoch \textit{nicht} den initialen Transfer der unplausiblen Datensätze vom mobilen Endgerät zum Server.

% Erläuterung des konkreten Gegenstands der Arbeit + Motivation + Relevanz

Mithilfe eines App-seitigen Erkennungsmechanismus für die Zugehörigkeit der Datenpakete zum gewünschten Zielkontext könnten nicht zugehörige Datenpakete vom Netzwerktransfer ausgeschlossen werden, um genutzte Bandbreite und Energie auf dem mobilen Endgerät zu sparen. Durch die App-seitige Erkennung des Kontextes wäre es außerdem möglich, Nutzende bei der Beendigung des Radfahrens an die noch laufende In-App-Aktivität, beispielsweise über Notifikationen, zu erinnern. Eine solche Kontexterkennung wird beispielsweise bereits im Betriebssystem der Apple Watch seit 2018 (mit Release von WatchOS 5) durchgeführt, um sportliche Aktivitäten zu erkennen und in Form von \enquote{Workout Reminders} in das Interaktionsschema zu integrieren\footnote{\url{https://support.apple.com/en-us/HT204523} (Abgerufen am 25.2.2021)}. Die Anwendungsgebiete einer solchen App-seitigen Kontexterkennung beschränken sich somit nicht nur auf das App-seitige Filtering von Datenpaketen, sondern ermöglichen auch die Integration von weiteren Interaktionsschemata.

\section{Problem- und Zielstellung}\label{sec:problem-und-zielstellung}

% Ziele der arbeit und erkenntnisinteresse:

% - Wie sind die erhobenen Datenpakete strukturiert und mithilfe welcher attribute kann eine Kontextabbildung erfolgen?
% - Welche potentiellen Abweichungen und Fehler sind innerhalb der Sensordaten zu erwarten?
% - Welche klassifikationsmethoden gibt es und welche Kennzahlen charakterisieren sie?
% - Welche eignen sich für die implementation auf einem mobilen endgerät (identifikation von limitierenden Faktoren)?
% - Welche Auswirkungen hat die Einführung eines solchen Klassifikationsmechanismus auf die Reduktion des datenpaketbasierten Netzwerktransfers (quantitativ)?
% - Inwiefern kann der Kontext-Klassifikationsmechanismus auf andere Kontexte erweitert werden?

% Mögliche Ergebnisse skizzieren

\section{Aufbau der Arbeit}\label{sec:aufbau-der-arbeit}
