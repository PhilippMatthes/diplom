\chapter{Einleitung}\label{ch:einleitung}\pagenumbering{arabic}

\section{Gegenstand und Motivation}\label{sec:gegenstand-und-motivation}

% Darstellung des Kontextes der Arbeit

STADTRADELN\footnote{\url{https://www.stadtradeln.de/darum-geht-es} (Abgerufen am 25.2.2021)} ist ein Projekt, bei dem Radfahrende teilnehmen können, um kollektiv mit dem Rad gefahrene Kilometer für den Klimaschutz zu sammeln. Begleitet wird das Projekt durch eine mobile App, mithilfe derer Nutzende ihre gefahrenen Kilometer tracken und sich mit anderen Teilnehmenden vergleichen können. Im Rahmen des Forschungsprojektes Movebis\footnote{\url{https://www.movebis.org/das-projekt/} (Abgerufen am 25.2.2021)} werden die hierbei erhobenen Daten ausgewertet und mithilfe von konkreten Kenngrößen visuell aufbereitet. Die aufbereiteten Daten werden der kommunalen Verkehrsplanung anschließend zur Verfügung gestellt, mit dem Ziel, die Planung der Radverkehrsinfrastruktur zu verbessern.
\\

% Fokussierung auf ein konkretes Teilgebiet des Kontextes + Motivation

Die von der STADTRADELN-App erhobenen Daten\footnote{\url{https://www.stadtradeln.de/datenschutz} (Abgerufen am 25.2.2021)} werden zu einem Server hochgeladen und von diesem für die spätere Auswertung protokolliert. Hierbei können Nutzende selbst entscheiden, zu welchen Zeitpunkten die Datenerfassung gestartet und beendet wird. Insbesondere kann es hierbei jedoch vorkommen, dass Nutzende speziell zum Anfang und Ende des Fahrradfahrens oder beim Wechsel auf ein anderes Verkehrsmittel Datenpakete aufzeichnen, bei denen kein Fahrrad gefahren wurde. Als Konsequenz hieraus ist nicht ausgeschlossen, dass für die spätere Auswertung unbrauchbare Datenpakete aufgezeichnet und zum Server übermittelt werden. Im Umfeld des Movebis-Projektes wurden daher bereits verschiedene Ansätze entwickelt, durch deren Einsatz für den Zielkontext (Fahrrad fahren) plausible von unplausiblen Datensätzen unter Berücksichtigung einer bestimmten Fehlerrate getrennt werden können \cite{matusek_anwendung_2019, stojanov_continuous_2020, werner_kontinuierliche_2020}. Bei den Ansätzen handelt es sich um Machine-Learning-Klassifikationssysteme, welche auf die Mustererkennung innerhalb der aufgezeichneten Datensätze trainiert wurden. Die entwickelten Machine-Learning-Ansätze ermöglichen eine Filtration der Datensätze zum Zeitpunkt der Auswertung, reduzieren jedoch \textit{nicht} die Aufzeichnung und den initialen Transfer der unplausiblen Datensätze vom Smartphone (hierbei betrachtet als Client) zum Server.
\\

% Erläuterung des konkreten Gegenstands der Arbeit + Motivation + Relevanz

Mithilfe eines clientseitigen Erkennungsmechanismus für die Zugehörigkeit der Datenpakete zum gewünschten Zielkontext könnten nicht zugehörige Datenpakete vom Netzwerktransfer ausgeschlossen werden, um die andernfalls genutzte Netzwerkbandbreite und Energie auf dem mobilen Endgerät zu sparen. Durch eine solche Erkennung wäre es außerdem möglich, Nutzende bei der Beendigung des Radfahrens an die noch laufende Datenaufzeichnung, beispielsweise über Notifikationen, zu erinnern. Eine ähnliche Erkennung wird bereits im Betriebssystem der Apple Watch seit 2018 (mit Release von WatchOS 5) durchgeführt, um verschiedene sportliche Aktivitäten zu erkennen und in Form von \enquote{Workout Reminders} in das Adaptionskonzept\footnote{Adaption: Die Anpassung von mobilen Applikationen als Reaktion auf einen sich ändernden Kontext} des Betriebssystems zu integrieren\footnote{\url{https://support.apple.com/en-us/HT204523} (Abgerufen am 25.2.2021)}. Außerdem stellen die beiden marktführenden mobilen Betriebssysteme Android\footnote{\url{https://developers.google.com/android/reference/com/google/android/gms/location/DetectedActivity} (Abgerufen am 3.3.2021)} und iOS\footnote{\url{https://developer.apple.com/documentation/coremotion/cmmotionactivity/1615451-cycling} (Abgerufen am 3.3.2021)} bereits abstrakte Schnittstellen zur Verfügung, mithilfe derer eine rudimentäre Aktivitätserkennung auf Grundlage von wenigen fest definierten Klassen (darunter auch Fahrrad fahren) durchgeführt werden kann.
\\

Mit der steigenden Verfügbarkeit von mobilen Machine-Learning-Frameworks wie TensorFlow Lite\footnote{\url{https://www.tensorflow.org/lite} (Abgerufen am 3.3.2021)} (iOS und Android, veröffentlicht in 2017) und CoreML\footnote{\url{https://developer.apple.com/documentation/coreml} (Abgerufen am 3.3.2021)} (iOS, veröffentlicht in 2017) und der Integration von spezialisierten Hardware-Koprozessoren für Machine-Learning in Smartphones bieten sich zunehmend mehr Möglichkeiten für die App-Entwicklung, Machine-Learning-Modelle auf Smartphones zu betreiben. Unter anderem ist es hierdurch auch möglich, Machine-Learning-Modelle auf Smartphones zu transferieren. Diese Möglichkeit bietet die Grundlage für die Problem- und Zielstellung der Diplomarbeit.


\section{Problem- und Zielstellung}\label{sec:problem-und-zielstellung}

Im Rahmen der Diplomarbeit soll untersucht werden, wie die zuvor entwickelten Machine-Learning-Ansätze aus \cite{matusek_anwendung_2019, stojanov_continuous_2020, werner_kontinuierliche_2020} anhand aktueller Machine-Learning-Frameworks auf Smartphones portiert werden können. Hierfür müssen die bestehenden Modelle analysiert und verglichen werden, um schließlich ein Portierungskonzept zu erstellen, mithilfe dessen die bestehenden Modellarchitekturen für mobile Anwendungen optimiert und transferiert werden sollen. Hierbei soll insbesondere auch untersucht werden, welche Hardware- und Softwareanforderungen an die bestehenden Modelle gekoppelt sind und welche Optimierungsmöglichkeiten existieren, um den Zeit-, Energie- und Speicherverbrauch auf Smartphones zu reduzieren. Das ermittelte Portierungskonzept soll in einer prototypischen App integriert werden. Auf dieser Grundlage sollen die portierten Modelle bezüglich deren Effizienz evaluiert werden, im Speziellen auch gegenüber den bestehenden abstrakten systemeigenen Schnittstellen. Zur differenzierten Diskussion dieses zentralen Forschungsgegenstands sollen folgende konkrete Forschungsfragen eruiert werden:

\begin{researchquestion}\label{rq1}
Aus welchen architekturellen Elementen bestehenden die existierenden Machine-Learning-Ansätze zur Klassifikation der STADTRADELN-Daten und wie können diese auf Smartphones transferiert werden?
\end{researchquestion}

\begin{researchquestion}\label{rq2}
Welche quantitativen Anforderungen stellen die bestehenden Machine-Learning-Ansätze an den Zeit-, Energie- und Speicherbedarf bei der Ausführung auf Smartphones?
\end{researchquestion}

\begin{researchquestion}\label{rq3}
Wie stehen die in \Cref{rq2} gefundenen Anforderungen an Zeit-, Energie- und Speicherbedarf mit der Qualität der Klassifikation des jeweiligen Machine-Learning-Ansatzes in Relation und welcher der bestehenden Ansätze eignet sich am besten für den Einsatz auf Smartphones?
\end{researchquestion}

\begin{researchquestion}\label{rq4}
Welche Strategien können auf die bestehenden Machine-Learning-Ansätze angewandt werden, um deren Energie- und Speicherbedarf zu optimieren, und welche quantitativen Auswirkungen hat dies gleichzeitig auf die Qualität der Klassifikation?
\end{researchquestion}

\chapter{Aufbau der Arbeit}\label{ch:aufbau-der-arbeit}

Um die obigen Forschungsfragen systematisch zu diskutieren, wird die Diplomarbeit in einem didaktisch üblichen Aufbau erstellt. Vor dem Hauptteil der Arbeit folgen Deckblatt und die wichtigsten Verzeichnisse und begleitende Inhalte.

\begin{enumerate}[label=\Roman*]
  \item Deckblatt
  \item Aufgabenstellung im Originaltext
  \item Erklärung zum Urheberrecht
  \item Vorwort
  \item Inhaltsverzeichnis
  \item Tabellenverzeichnis
  \item Abbildungsverzeichnis
  \item Glossar
\end{enumerate}

\newlist{gliederung}{enumerate}{10}
\setlist[gliederung]{label*=.\arabic*}
\setlist[gliederung,1]{label=\arabic*}

\paragraph{Einleitung:} Anschließend folgt eine Einleitung in das Thema, welche Gegenstand, Motivation, Problem- und Zielstellung, sowie den Aufbau der Arbeit beinhaltet.

\begin{gliederung}
\item Einleitung
  \begin{gliederung}
  \item Gegenstand und Motivation
  \item Problem- und Zielstellung
  \item Aufbau der Arbeit
  \end{gliederung}
\end{gliederung}

\paragraph{Grundlagen:} Nach der Einleitung werden zunächst grundlegende domänenspezifische Begriffe und Zusammenhänge erläutert. Die Sektionen \enquote{Grundlagen} und \enquote{Verwandte Arbeiten} sind bereits sehr breit angelegt, um die möglichen zu diskutierenden Aspekte im didaktischen Aufbau zu zeigen. Im ersten Teil der Bearbeitung (Analyse des Forschungsthemas, siehe \Cref{ch:vorlaufiger-zeitplan}) sollen diese Punkte bewusst detailliert aufgearbeitet werden. Nach der detaillierten Aufbereitung werden diese Aspekte erneut bezüglich ihrer Relevanz evaluiert und bis zur finalen Fassung der Diplomarbeit zu einer didaktisch konsistenten Form kondensiert.

\begin{gliederung}[resume]
\item Grundlagen
  \begin{gliederung}
  \item Datengrundlage
    \begin{gliederung}
    \item Kontextsensitivität im Mobile Computing
    \item Erfassung von GPS- und Sensordaten in der STADTRADELN-App
    \item Zentrale Probleme der Datenerfassung (Messfehler und -differenzen, Energieverbrauch, Datenschutz und -sicherheit)
    \item Methoden zur Datenanalyse
    \item Methoden zur Vorverarbeitung und Aufbereitung der Daten
    \end{gliederung}
  \item Künstliche Intelligenz
    \begin{gliederung}
    \item Klassische datenorientierte Probleme (Clustering, Regression, Klassifikation und Generation)
    \item Limitierung und Praktikabilität von herkömmlichen Algorithmen
    \item Human Activity Recognition
    \end{gliederung}
  \item Machine-Learning
    \begin{gliederung}
    \item Taxonomische Einordnung und Definition
    \item Inferenz versus Training
    \item Lernmethoden
      \begin{gliederung}
      \item Supervised Learning
      \item Unsupervised Learning
      \item Reinforcement Learning
      \item Kombinierte Lernverfahren (Ensemble Learning, Transfer Learning)
      \end{gliederung}
    \item Überblick über konkrete Machine-Learning-Architekturen
      \begin{gliederung}
      \item Decision Trees und Random Forests
      \item Support Vector Machines
      \item Künstliche Neuronale Netze und Deep Learning
      \item Recurrent Neural Networks
      \item Convolutional Neural Networks
      \item Weitere Architekturen und Anwendungsgebiete im Überblick (Lineare Regression, Naive Bayes Klassifikation, K-Nearest-Neighbors und K-Means-Clustering, PCA, t-SNE, Hidden Markov Models, Restricted Boltzmann Machines, Adversarial Neural Networks, Autoencoder)
      \end{gliederung}
    \item Typische Probleme von Machine-Learning-Modellen
    \item Abhängigkeiten zur Datenvorverarbeitung
    \item Modell-Hyperparameter und deren Auswirkung
    \end{gliederung}
  \item Zusammenfassung
  \end{gliederung}
\end{gliederung}

\paragraph{Verwandte Arbeiten:} An die Betrachtung der Grundlagen schließt sich eine Diskussion verwandter Arbeiten an. Insbesondere sollen in diesem Kapitel neben der verwandten Forschungsliteratur in Form von Sachbüchern und wissenschaftlichen Papers auch die Arbeiten betrachtet werden, welche im Rahmen der Konzeption und Implementation der Machine-Learning-Ansätze zur Auswertung der Datenpakete im Movebis-Projekt erstellt wurden.

\begin{gliederung}[resume]
\item Verwandte Arbeiten
  \begin{gliederung}
  \item Einordnung und Überblick
  \item Stand der Technik
    \begin{gliederung}
    \item Datenerfassung und -vorverarbeitung
    \item Verkehrsmittelerkennung mit Machine-Learning
    \item Methoden zur Verbesserung des Lernerfolgs
    \item Methoden zur Validierung und Visualisierung des Lernerfolgs
    \item Methoden zur Modelloptimierung
      \begin{gliederung}
      \item Beschleunigung und Latenzreduktion
      \item Quantisierung
      \item Pruning
      \item Clustering
      \end{gliederung}
    \item Methoden zur Portierung von Machine-Learning-Modellen auf Smartphones
    \end{gliederung}
  \item Bestehende Ansätze des Movebis-Projektes
    \begin{gliederung}
    \item Klassifikation über Support Vector Machines und Random Forests nach \cite{matusek_anwendung_2019}
    \item Klassifikation über Recurrent Neural Networks nach \cite{werner_kontinuierliche_2020}
    \item Klassifikation über Convolutional Neural Networks nach \cite{stojanov_continuous_2020}
    \item Vergleich der Klassifikationsansätze
    \end{gliederung}
  \item Zusammenfassung
  \end{gliederung}
\end{gliederung}

\paragraph{Analyse:} Basierend auf den Grundlagen und verwandten Arbeiten soll anschließend eine Analyse durchgeführt werden. Hierbei soll vordergründig analysiert werden, inwiefern die erstellten Modelle für eine Portierung auf Smartphones geeignet sind. Außerdem sollen Analysen für die spätere Erstellung des Applikationsprototyps durchgeführt werden.

\begin{gliederung}[resume]
\item Analyse
  \begin{gliederung}
  \item Datenbestand und mögliche Fehlerquellen
  \item Portierbarkeit bestehender Ansätze des Movebis-Projektes
    \begin{gliederung}
    \item Hardware- und Softwareanforderungen
    \item Limitationen beim Training gegenüber Inferenz
    \item Analyse von möglichen Optimierungschancen
    \item Spezifikation von Integrationspunkten
    \end{gliederung}
  \item Analyse einer prototypischen App
    \begin{gliederung}
    \item Analyse von systemeigenen Schnittstellen zur Aktivitätserkennung
    \item Anforderungen an die Erfassung der Daten und Systemparameter
    \end{gliederung}
  \item Zusammenfassung
  \end{gliederung}
\end{gliederung}

\paragraph{Konzept:} Mithilfe der Analyse soll ein konkretes Konzept erarbeitet und vorgestellt werden, welches eine Klassifikation des Kontextes anhand des Beispiels der STADTRADELN-Datenpakete auf Smartphones umsetzt.

\begin{gliederung}[resume]
\item Konzept
  \begin{gliederung}
  \item Datenaggregation und -vorverarbeitung
  \item Training, Validierung und Test der Modelle
  \item Portierung der Modelle
    \begin{gliederung}
    \item Optimierung
    \item Konvertierung
    \item Export und Transfer
    \item Integration
    \item Validierung
    \end{gliederung}
  \item Evaluationskonzept
    \begin{gliederung}
    \item Entwurf einer Test-Applikation
    \item Integration der Modelle
    \item Datenerfassung und Inferenz
    \item Szenarienbasierte Testcases
    \item Statistische Evaluation von Energie- und Speicherbedarf
    \item Statistischer Vergleich mit systemeigenen Schnittstellen
    \end{gliederung}
  \item Zusammenfassung
  \end{gliederung}
\end{gliederung}

\paragraph{Implementation und Evaluation:} Die konzipierte Test-App soll anschließend implementiert und evaluiert werden.

\begin{gliederung}[resume]
\item Implementation
  \begin{gliederung}
  \item Genutzte Frameworks und Technologien
  \item Erstellung, Training, Validierung und Test der Modelle
  \item Erstellung einer prototypischen iOS-App
  \item Portierung der Modelle (Fokus auf Optimierung)
  \item Zusammenfassung
  \end{gliederung}
\item Evaluation
  \begin{gliederung}
  \item Erfasste Daten und Testcases
  \item Ergebnisse der Evaluation von Energie- und Speicherbedarf
  \item Ergebnisse des statistischen Vergleichs mit systemeigenen Schnittstellen
  \item Fehlerbetrachtung und Gefahren für die Validität
  \item Zusammenfassung
  \end{gliederung}
\end{gliederung}

\paragraph{Fazit:} Zum Abschluss der Arbeit werden die eingangs gestellten Forschungsfragen anhand des evaluierten Konzeptes beantwortet und ein Ausblick über mögliche zukünftige Forschungsbeträge gegeben.

\begin{gliederung}[resume]
\item Fazit
  \begin{gliederung}
  \item Forschungsbeitrag
  \item Ausblick
  \end{gliederung}
\end{gliederung}

Im Anhang der Arbeit befinden sich zusätzliche Dokumente und Informationen, sowie das Literaturverzeichnis und eine Danksagung.

\begin{enumerate}[label=\roman*]
\item Anhang
\item Literaturverzeichnis
\item Danksagung
\end{enumerate}

% ---------------------------------------------

\chapter{Vorläufiger Zeitplan}\label{ch:vorlaufiger-zeitplan}

Dieser Abschnitt soll einen vorläufigen Zeitplan über die Bearbeitung der Diplomarbeit geben. Für Diplomarbeiten sind an der TU Dresden 22 Wochen als Bearbeitungszeit vorgesehen, an welcher sich dieser Zeitplan orientiert. Wird somit von einem Bearbeitungsstart am 1. April 2021 ausgegangen, so ist die Bearbeitungszeit der Diplomarbeit über 22 Wochen bis zum 2. September 2021.

\begin{itemize}
\item 1. Projektbearbeitungsphase (ca. 4 Wochen): Analyse des Forschungsthemas. Diese Projektbearbeitungsphase beginnt am 1. April 2021 und läuft bis 1. Mai 2021. In dieser Phase werden verwandte Forschungsarbeiten studiert und im Rahmen der ersten Sektionen der Diplomarbeit (Einleitung, Grundlagen, Verwandte Arbeiten) diskutiert. Abschluss der Projektbearbeitungsphase ist 1-stündiges Kolloquium mit einem Vortragsteil von 40 Minuten im Rahmen des Moduls INF-D-960.
\item 2. Projektbearbeitungsphase (ca. 9 Wochen): Analyse, Konzepterarbeitung und Implementation. Diese Projektbearbeitungsphase beginnt am 1. Mai 2021 und ist angesetzt bis zum 1. Juli 2021. Abschluss der Projektbearbeitungsphase ist eine Zwischenpräsentation.
\item 3. Projektbearbeitungsphase (ca. 9 Wochen): Evaluation, Optimierung und Konsolidierung der Diplomarbeit. Diese Projektbearbeitungsphase beginnt am 1. Juli 2021 und läuft bis zum 2. September 2021. Abschluss der Projektbearbeitungsphase ist die Abgabe der Diplomarbeit und die sich daran anschließende Diplomprüfung.
\end{itemize}

\chapter{Verwandte Arbeiten}\label{ch:verwandte-arbeiten}

Dieser Abschnitt gibt einen Ausblick auf zentrale Literatur, welche im Rahmen der Diplomarbeit studiert, diskutiert und ggf. angewandt werden soll.
\\

Für einen praxisorientierten Überblick über generelle Strategien und grundlegende Konzepte im Machine-Learning eignet sich das Sachbuch \enquote{Praxiseinstieg Machine Learning mit Scikit-Learn und TensorFlow: Konzepte, Tools und Techniken für intelligente Systeme} \cite{geron_praxiseinstieg_2018}, sowie mit Fokus auf die App-seitige Entwicklung von Machine-Learning-Konzepten speziell für iOS-Apps \enquote{Machine Learning with Swift: Artificial Intelligence for iOS} \cite{sosnovshchenko_machine_2018}. Anhand des Sachbuchs \enquote{Datenanalyse mit Python: Auswertung von Daten mit Pandas, NumPy und IPython} \cite{mckinney_datenanalyse_2018} werden konkrete datenanalytische Methoden erläutert, welche zur Voranalyse der Datensätze angewandt werden können. \cite{zhang_deep_2019} zeigen darüber hinaus verschiedene Anwendungsbereiche von Machine-Learning im Fachgebiet \textit{Mobile and Wireless Networking} und geben einen extensiven Einblick über Grundlagen und die im Machine-Learning angewandten Taxonomien und Termini.
\\

Das zu erstellende Machine-Learning-Konzept lässt sich in den Forschungsbereich der \textit{Human Activity Recognition} (Akronym HAR) einordnen, dessen Ziel die Ermittlung und der Einsatz von Algorithmen für die Erkennung von Aktivitäten einer Person ist. \cite{bulling_tutorial_2014} geben einen progressiven und forschungsorientierten Einstieg in dieses Fachgebiet. Zur Realisation von HAR wurden bereits verschiedene Machine-Learning-Ansätze konzipiert und evaluiert, unter der Anwendung von Support Vector Machines \cite{nurhanim_classification_2017}, mehrschichtigen Perzeptronen \cite{kwapisz_activity_2011}, Convolutional Neural Networks \cite{zeng_convolutional_2014, chen_deep_2015}, Recurrent Neural Networks \cite{inoue_deep_2018} und hybriden Machine-Learning-Systemen \cite{abu_alsheikh_deep_2015, ravi_deep_2017}. Zudem wurden verschiedene Studien durchgeführt, in denen konkrete Machine-Learning-Ansätze bezüglich ihrer Performanz bei HAR verglichen wurden. \cite{jahangiri_applying_2015} verglichen beispielsweise Support Vector Machines und Random Forests.
\\

Da die zu portierenden Machine-Learning-Konzepte die Restriktion besitzen, dass diese auf einem Smartphone lauffähig sein sollen, sind die obigen Ansätze wegen der hardware- und softwaretechnischen Limitierungen nur teilweise applizierbar. \cite[S. 13]{ota_deep_2017} separieren daher die Deployment-Modelle des Deep Learning (Teilgebiet des Machine-Learning) auf mobilen Geräten in zwei Teilbereiche:

\begin{itemize}
\item \textit{Client-Server-Deployment}, wobei die Inferenz auf dem Server-Backend geschieht, analog zum aktuellen Deployment-Konzept der STADTRADELN-App in Kooperation mit Movebis
\item \textit{Client-Only-Deployment}, wobei die Inferenz vom Client selbst übernommen wird
\end{itemize}

Die bestehenden Arbeiten, welche mithilfe von Machine-Learning-Ansätzen das Filtern der STADTRADELN-\allowbreak Daten im Movebis-Projekt ermöglichen, gliedern sich in das Client-Server-Deployment ein, da sie nicht in der mobilen App integriert werden, sondern auf die Translokation der Datenpakete zum Server angewiesen sind. \cite{matusek_anwendung_2019} wendet für die Klassifikation der Datenpakete ein hybrides Klassifikationskonzept aus Support Vector Machine und Random Forest an und diskutiert gleichzeitig wesentliche Schritte, welche bei der Vorverarbeitung der Datensätze zu beachten sind. \cite{stojanov_continuous_2020} sowie \cite{werner_kontinuierliche_2020} bauen auf diesem Grundkonzept auf und implementieren die Klassifikation über ein Convolutional Neural Network \cite{stojanov_continuous_2020} respektive einem Recurrent Neural Network \cite{werner_kontinuierliche_2020}.
\\

In Ergänzung zu diesen vorliegenden Konzepten liegt die Problemstellung dieser Arbeit auf der Portierung des Deployment-Modells auf das Client-Only-Deployment von Smartphones. Als Edge-Geräte sind Smartphones vor allem durch die auf dem Gerät verfügbaren Ressourcen limitiert, als zentrale Aspekte stehen sich hierdurch die Optimalität (Qualität, Präzision der Inferenz) und die Effizienz gegenüber \cite{zou_edge_2019, ma_survey_2019, deng_model_2020}. Dies erfordert neue Strategien aus dem Forschungsbereich \textit{Edge AI}, um Machine-Learning-Konzepte für Edge-Geräte zu entwickeln und anzupassen. Eine besondere Herausforderung und damit auch ein zentrales Thema im Forschungsbereich Edge AI stellt das Anlernen von Machine-Learning-Modellen dar, da dieser Prozess im Vergleich zur Inferenz in der Regel eine rechenaufwändige mathematische Optimierung und Anpassung der Modellparameter erfordert.

\begin{figure}[H]
\includegraphics[width=0.75\linewidth, bb=0 0 367 304]{six-level-rating-ei.pdf}
\caption{6-stufiger taxonomischer Überblick über Lern- und Inferenzverfahren im Edge Computing nach \cite{zhou_edge_2019}.}\label{fig:six-level-rating-ei}
\end{figure}

\Cref{fig:six-level-rating-ei} zeigt einen 6-stufigen taxonomischen Überblick über die verschiedene Möglichkeiten für das Training von Machine-Learning-Modellen im Edge Computing in Kooperation mit Cloud-Servern nach \cite{zhou_edge_2019}. Besonders in den Stufen \textit{All-In-Edge} und \textit{Cloud-Edge Co-Training} sind spezielle Lernkonzepte wie beispielsweise das in \cite{wang_-edge_2019} gezeigte \textit{Federated Learning} notwendig, mithilfe derer die Modellparameter über das Netzwerk zwischen Edge-Geräten und Cloud-Servern ausgetauscht werden können. Das in dieser Arbeit zu erstellende Portierungskonzept soll sich jedoch analog zu \Cref{sec:problem-und-zielstellung} auf die darunter liegende Stufe \textit{On-Device Intelligence} im Rahmen des Client-Only-Deployments beschränken, bei der das vorliegende Machine-Learning-Konzept vortrainiert wird und nur die Inferenz auf dem Gerät stattfindet.
\\

Hierbei ist vor allem (als Teil der Portierung) die Optimierung der vorliegenden Machine-Learning-Konzepte von hoher Relevanz, um eine hohe Effizienz und Optimalität der Inferenz zu gewährleisten. Vor diesem Hintergrund wurden bereits verschiedene Analysen und Konzepte über Optimierungsmethoden für Machine-Learning-Modelle im Edge Computing erstellt. \cite{deng_model_2020} und \cite{choudhary_comprehensive_2020} vermitteln einen detaillierten Überblick über den Stand der Forschung in der Modelloptimierung. \cite{nan_deep_2019} und \cite{dai_toward_2021} analysieren verschiedene Optimierungsmethoden für mobile Plattformen unter anderem bezüglich ihrer konkreten Auswirkungen auf die Modellpräzision, die Inferenzzeit, die CPU-Last und die damit verbundenen thermischen Auswirkungen. Außerdem existieren bereits konkrete Ansätze aus dem Bereich Edge Computing, welche verschiedene Modelloptimierungen implementieren und evaluieren \cite{wang_context-aware_2020, wang_icing-edgenet_2021, tu_deep_2019, schneider_q-eegnet_2020}.
\\

Im Forschungsbereich Human Activity Recognition existieren bereits mehrere konkrete Machine-Learning-Ansätze, welche für eine Ausführung auf Smartphones (als Edge-Geräte) geeignet sind \cite{martin_activity_2013, chen_deep_2020}. Neben der Art und Struktur der Daten unterscheiden sich diese Ansätze wesentlich in der Architektur des jeweiligen Machine-Learning-Klassifikators. Hierbei wurden neben klassischen Ansätzen wie K-Nearest-Neighbors \cite{ken_taylor_activity_2011} vor allem aktuelle Machine-Learning-Methoden aus dem Bereich Deep Learning getestet, darunter Deep Bayesian Neural Networks \cite{gudur_activeharnet_2019}, Deep Feed-Forward Neural Networks \cite{li_deep_2020}, Restricted Boltzmann Machines \cite{bhattacharya_smart_2016, radu_towards_2016}, Recurrent Neural Networks mit LSTM \cite{mairittha_-device_2019, mairittha_-device_2021}, Convolutional Neural Networks \cite{mairittha_improving_2020} sowie hybride Ansätze \cite{ravi_deep_2017, nutter_design_2018}.
\\

Diese Arbeiten zeigen verschiedene Formen der Datenvorverarbeitung, darunter beispielsweise die Vorverarbeitung von Sensordaten über Kurzzeit-Fourier-Transformation und Spektrogrammen für Convolutional Neural Networks \cite{ravi_deep_2016, ravi_deep_2017}, oder auch die Nutzung des Verfahrens \textit{Principal Component Analysis} zur Dimensionalitätsreduktion der Input-Vektoren eines Machine-Learning-Systems \cite{nutter_design_2018}. Zur Ausgabenverarbeitung der Machine-Learning-Systeme existieren weitere verschiedene Ansätze, häufig wird eine Confusion Matrix zur Quantifizierung der Performanz angewandt. \cite{huo_uncertainty_2020} zeigen eine Möglichkeit, wie unter anderem hierüber eine Vorhersage darüber getroffen werden kann, wie sicher eine Inferenz auf den tatsächlichen Wert zutrifft. Dieses Verfahren ist insbesondere bei der Regelung der Häufigkeit der Inferenz betrachtenswert, besitzt jedoch auch Potenzial für eine Anwendung im Bereich der kontextsensitiven Adaption. Ein ähnlicher Konfidenzwert wird auch vom \texttt{CMMotionActivity} Framework im iOS-Betriebssystem bereitgestellt\footnote{\url{https://developer.apple.com/documentation/coremotion/cmmotionactivity/1615433-confidence} (Abgerufen am 5.3.2021)}.
\\

\chapter{Zusammenfassung}

In diesem Exposé wurden die zentralen Aspekte hinter einer möglichen Diplomarbeit über die \textit{Erstellung und Evaluation eines Portierungskonzeptes für Machine-Learning-Ansätze zur Aktivitätsklassifikation auf Smartphones} illustriert. Neben der Motivation der Diplomarbeit wurde eine konkrete Problem- und Zielstellung erarbeitet, insbesondere die konstituierenden Forschungsfragen.
\\

Anschließend wurde eine konkrete Gliederung erstellt, anhand derer die Diplomarbeit in einem didaktisch üblichen Aufbau abgehandelt werden soll. Hierbei werden zunächst die Motivation, Problem- und Zielstellung analog zu diesem Exposé erörtert, um hierauf aufbauend Grundkonzepte und verwandte Arbeiten zu diskutieren. Ziel ist die Analyse und Erstellung des Konzeptes, um dieses anhand einer prototypischen Implementation zu evaluieren.
\\

Der zeitliche Rahmen der Diplomarbeit gliedert sich in 3 Projektphasen, wobei die erste Projektphase von einer verstärkten Recherche der Literatur als Grundlage für die spätere Konzepterstellung geprägt ist. Neben Literatur, welche ein allgemeines Verständnis für Machine-Learning-Systeme bereitstellt, sollen auch konkrete Forschungsansätze zu zentralen Themen der Diplomarbeit analysiert werden. Dazu wurde die Diplomarbeit unter anderem in die Forschungsgebiete \textit{Human Activity Recognition} und \textit{Edge AI} eingeordnet und in Relation zu verwandten Arbeiten in diesen Forschungsgebieten gesetzt. Neben den 3 bestehenden Machine-Learning-Ansätzen aus dem Movebis-Projekt konnten hierbei bereits mehrere verwandte Konzepte identifiziert werden, welche bei der \textit{Analyse des Forschungsthemas} studiert und diskutiert werden können.
