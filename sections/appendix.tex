\tocless\chapter{Movebis-Modellarchitekturen}\label{appendix:a}

\begin{figure}[h]
  \centering
  \begin{subfigure}[t]{.25\textwidth}
    \centering
    \includegraphics[width=\linewidth, bb=0 0 368 491]{matusek-model.pdf}
    \caption{FFN aus \cite[S. 80]{matusek_anwendung_2019}.}
    \label{fig:matusek-ffn}
  \end{subfigure}%
  \begin{subfigure}[t]{.25\textwidth}
    \centering
    \includegraphics[width=\linewidth, bb=0 0 443 1139]{stojanov-model.pdf}
    \caption{\acrshort{cnn} aus \cite[S. 76]{stojanov_continuous_2020}.}
    \label{fig:stojanov-cnn}
  \end{subfigure}
  \begin{subfigure}[t]{.49\textwidth}
    \centering
    \includegraphics[width=\linewidth, bb=0 0 957 1031]{werner-model.pdf}
    \caption{\acrshort{rnn} aus \cite[S. 24]{werner_kontinuierliche_2020}.}
    \label{fig:werner-rnn}
  \end{subfigure}
  \caption[Konkrete Konfigurationen der Machine-Learning-Architekturen aus den Movebis-Ansätzen zur Analyse der Modellgrößen.]{Konkrete Konfigurationen der Machine-Learning-Architekturen aus den Movebis-Ansätzen zur Analyse der Modellgrößen. Das dreigeteilte \acrshort{rnn}-Netzwerk aus \cite{werner_kontinuierliche_2020} wird an der oberen Seite entlang der Zeitreihe eingebunden. \cite{stojanov_continuous_2020} stapelt die erstellten Feature-Fenster in drei Dimensionen (Zeitstempel, Anzahl Features, gestapelte Fenster) für den Input in das \acrshort{cnn}.}
  \label{fig:ml-movebis}
\end{figure}
