\chapter{Einleitung}\label{ch:einleitung}\pagenumbering{arabic}

\section{Gegenstand und Motivation}\label{sec:gegenstand-und-motivation}

% Darstellung des Kontextes der Arbeit

STADTRADELN\footnote{\url{https://www.stadtradeln.de/darum-geht-es} (Abgerufen am 25.2.2021)} ist ein Projekt, bei dem Radfahrende teilnehmen können, um kollektiv mit dem Rad gefahrene Kilometer für den Klimaschutz zu sammeln. Begleitet wird das Projekt durch eine mobile App, mithilfe derer Nutzende ihre gefahrenen Kilometer tracken und sich mit anderen Teilnehmenden vergleichen können. Im Rahmen des Forschungsprojektes Movebis\footnote{\url{https://www.movebis.org/das-projekt/} (Abgerufen am 25.2.2021)} werden die hierbei erhobenen Daten ausgewertet und mithilfe von konkreten Kenngrößen visuell aufbereitet. Die aufbereiteten Daten werden der kommunalen Verkehrsplanung anschließend zur Verfügung gestellt, mit dem Ziel, die Planung der Radverkehrsinfrastruktur zu verbessern.

% Fokussierung auf ein konkretes Teilgebiet des Kontextes + Motivation

Die von der STADTRADELN-App erhobenen Daten\footnote{\url{https://www.stadtradeln.de/datenschutz} (Abgerufen am 25.2.2021)} werden zu einem Server hochgeladen und von diesem für die spätere Auswertung protokolliert. Hierbei können Nutzende selbst entscheiden, zu welchen Zeitpunkten die Datenerfassung gestartet und beendet wird. Insbesondere kann es hierbei jedoch vorkommen, dass Nutzende speziell zum Anfang und Ende des Fahrradfahrens oder beim Wechsel auf ein anderes Verkehrsmittel Datenpakete aufzeichnen, bei denen kein Fahrrad gefahren wurde. Als Konsequenz hieraus ist nicht ausgeschlossen, dass für die spätere Auswertung unbrauchbare Datenpakete aufgezeichnet und zum Server übermittelt werden. Im wissenschaftlichen Umfeld des Movebis-Projektes wurden daher bereits verschiedene Ansätze entwickelt, durch deren Einsatz für den Zielkontext (Fahrrad fahren) plausible von unplausiblen Datensätzen unter Berücksichtigung einer bestimmten Fehlerrate getrennt werden können \cite{matusek_anwendung_2019, stojanov_continuous_2020, werner_kontinuierliche_2020}. Bei den Ansätzen handelt es sich um Machine-Learning-Klassifikationssysteme, welche auf die Mustererkennung innerhalb der aufgezeichneten Datensätze trainiert wurden. Die entwickelten Machine-Learning-Ansätze ermöglichen eine Filtration der Datensätze zum Zeitpunkt der Auswertung, reduzieren jedoch \textit{nicht} die Aufzeichnung und den initialen Transfer der unplausiblen Datensätze vom Smartphone (hierbei betrachtet als Client) zum Server.

% Erläuterung des konkreten Gegenstands der Arbeit + Motivation + Relevanz

Mithilfe eines clientseitigen Erkennungsmechanismus für die Zugehörigkeit der Datenpakete zum gewünschten Zielkontext könnten nicht zugehörige Datenpakete vom Netzwerktransfer ausgeschlossen werden, um die andernfalls genutzte Netzwerkbandbreite und Energie auf dem mobilen Endgerät zu sparen. Durch eine solche Erkennung wäre es außerdem im Rahmen eines Adaptivitätskonzeptes möglich, Nutzende bei der Beendigung des Radfahrens an die noch laufende Datenaufzeichnung zu erinnern. Die beiden marktführenden mobilen Betriebssysteme Android\footnote{\url{https://developers.google.com/android/reference/com/google/android/gms/location/DetectedActivity} (Abgerufen am 3.3.2021)} und iOS\footnote{\url{https://developer.apple.com/documentation/coremotion/cmmotionactivity/1615451-cycling} (Abgerufen am 3.3.2021)} stellen bereits abstrakte Schnittstellen zur Verfügung, mithilfe derer eine rudimentäre Aktivitätserkennung auf Grundlage von wenigen fest definierten Klassen (darunter auch Fahrrad fahren) durchgeführt werden kann. Mit der steigenden Verfügbarkeit von mobilen Machine-Learning-Frameworks und der Integration von spezialisierten Hardware-Koprozessoren für Machine-Learning in Smartphones bieten sich jedoch zunehmend mehr Möglichkeiten auch für die App-Entwicklung, genau für den Anwendungsfall zugeschnittene Machine-Learning-Modelle auf Smartphones zu transferieren und zu betreiben. Diese Möglichkeit bietet die Grundlage für die Problem- und Zielstellung.

\section{Problem- und Zielstellung}\label{sec:problem-und-zielstellung}

In dieser Arbeit soll untersucht werden, wie die Kernkonzepte der zuvor entwickelten Machine-Learning-Ansätze aus \cite{matusek_anwendung_2019, stojanov_continuous_2020, werner_kontinuierliche_2020} anhand aktueller Frameworks und Strategien auf Smartphones portiert werden können. Hierfür müssen die bestehenden Konzepte neben verwandten Forschungsarbeiten analysiert und mit Hinblick auf die Portierbarkeit und die systemlimitierenden Faktoren wie Verarbeitungszeit, Energie- und Speicherbedarf verglichen werden. Die hieraus hervorgegangene beste Lösung soll anschließend im Rahmen eines Portierungskonzeptes auf Smartphones transferiert werden, um hierauf basierend eine Analyse des Tradeoffs zwischen Ressourcenbedarf und Ergebnisqualität durchzuführen. Hierbei soll insbesondere auch untersucht werden, welche Hardware- und Softwareanforderungen an die bestehenden Modelle gekoppelt sind und wie die Modelle für den Einsatz auf Smartphones optimiert werden können. Das ermittelte Portierungskonzept soll in einer prototypischen App integriert werden. Auf dieser Grundlage sollen die portierten Modelle bezüglich des Tradeoffs evaluiert werden. Zur differenzierten Diskussion dieses zentralen Forschungsgegenstands sollen folgende konkrete Forschungsfragen eruiert werden:

\begin{researchquestion}\label{rq1}
Aus welchen architekturellen Elementen bestehen die existierenden Machine-Learning-Ansätze zur Klassifikation der STADTRADELN-Daten und wie können diese auf Smartphones transferiert werden?
\end{researchquestion}

\begin{researchquestion}\label{rq2}
Welche quantitativen Anforderungen stellen die bestehenden Machine-Learning-Ansätze an den Zeit-, Energie- und Speicherbedarf bei der Ausführung auf Smartphones?
\end{researchquestion}

\begin{researchquestion}\label{rq3}
Wie stehen die in \Cref{rq2} gefundenen Anforderungen an Zeit-, Energie- und Speicherbedarf mit der Ergebnisqualität der Klassifikation des jeweiligen Machine-Learning-Ansatzes in Relation und welcher der bestehenden Ansätze eignet sich am besten für den Einsatz auf Smartphones?
\end{researchquestion}

\begin{researchquestion}\label{rq4}
Welche Strategien können auf die bestehenden Machine-Learning-Ansätze appliziert werden, um deren Energie- und Speicherbedarf zu optimieren, und welche quantitativen Auswirkungen hat dies gleichzeitig auf die Qualität der Klassifikation?
\end{researchquestion}

\section{Aufbau der Arbeit}\label{sec:aufbau-der-arbeit}

Der Aufbau dieser Arbeit ist didaktisch typisch, chronologisch und nach Abstraktionslevel strukturiert. Zunächst werden die abstrakten Grundlagen behandelt und motiviert, darunter elementare Faktoren, Methoden und Probleme bei der Verkehrsmittelerkennung und deren Portierung auf Smartphones. Für ein besseres Verständnis der Forschungsdomäne werden Grundbegriffe eingeführt und taxonomisch illustriert. Die Grundlagen dieser Arbeit gliedern sich in die Datenerfassung, die Datenverarbeitung und Integration, sowie die Portierung.

Anschließend werden verwandte Arbeiten diskutiert. Hierzu wird zunächst ein Forschungsüberblick gegeben, um die Konzepte und Ergebnisse der Arbeiten nachfolgend genau zu betrachten und gegenüberzustellen. Hierbei wird auch illustriert, welche generellen Herangehensweisen bei einer Verkehrsmittelerkennung von den verwandten Arbeiten verfolgt wurden. Zusätzlich soll darauf eingegangen werden, warum die gefundenen Arbeiten die zentralen Fragen dieser Arbeit nicht hinreichend lösen konnten.

Die beschriebenen Grundlagen und verwandten Arbeiten fließen in die darauf folgende Erstellung des Konzeptes dieser Arbeit ein. Zunächst müssen jedoch die konkreten Anforderungen des Konzeptes geklärt werden. Dies inkludiert eine Analyse der Anwendungsfälle und der Portierung mit Hinblick auf die spätere Evaluation. Auf dieser Basis dieser Analyse kann das Konzept im Nachfolgenden erstellt werden. Die Konzeption der Verkehrsmittelerkennung basiert auf einer empirischen Auswahl konkreter Elemente aus den betrachteten verwandten Arbeiten.

Die Evaluation des Konzeptes und des Tradeoffs wird im Rahmen einer prototypischen Implementation mit Definition konkreter Frameworks und Technologien konzipiert und durchgeführt. Es werden experimentelle Daten erhoben, mit dem Ziel, die Ergebnisqualität und den Ressourcenbedarf des Konzeptes unter Einsatz von Optimierungsmethoden zu bewerten. Die Studie der so gefundenen Ergebnisse erfolgt auf Grundlage von quantitativen Metriken. Abschließend wird unter Betrachtung der offenen Punkte und des wissenschaftlichen Ausblicks diskutiert, inwiefern die eingangs formulierten Forschungsfragen beantwortet wurden.
