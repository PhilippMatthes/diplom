\documentclass[ngerman]{tudscrreprt}
\ifpdftex{
\usepackage[T1]{fontenc}

\usepackage{draftwatermark}
\SetWatermarkText{Entwurf}
\SetWatermarkScale{1}
\SetWatermarkColor[gray]{0.9}

\RequirePackage[ngerman=ngerman-x-latest]{hyphsubst}
\usepackage{babel}
\usepackage{isodate}
\usepackage{tudscrsupervisor}
\usepackage{enumitem}\setlist{noitemsep}
\begin{document}
\faculty{Fakultät Informatik}
\institute{Institut für Systemarchitektur}\chair{Professur für Rechnernetze}
\title{%
Erstellung und Evaluation eines Machine-Learning-Konzeptes für die Klassifikation genutzter Verkehrsmittel anhand von Kontextinformationen in Smartphones
}
\thesis{diploma}\graduation[Dipl.-Inf.]{Diplom-Informatiker}
\author{%
Philipp Matthes\matriculationnumber{4605459}%
\dateofbirth{12.3.1997}\placeofbirth{Chemnitz}%
}\matriculationyear{2016}\issuedate{1.1.1970}\duedate{1.1.1970}
\supervisor{Dr. Thomas Springer}\professor{Prof. Dr. rer. nat. habil. Dr. h. c. Alexander Schill}
\taskform[pagestyle=empty]{%
In der STADTRADELN\footnote{\url{https://www.stadtradeln.de}}-App werden GPS- und Sensordaten von Radfahrenden gespeichert und für die Weiterverarbeitung an einen Server übermittelt. Die protokollierten Datensätze werden unter anderem durch das Forschungsprojekt Movebis\footnote{\url{https://www.movebis.org}} genutzt, mit dem Ziel, die Planung der Radverkehrsinfrastruktur zu verbessern. Bei Analysen der Datensätze konnte festgestellt werden, dass diese auch Datenpakete mit für Radfahrer untypischen Parameterwerten beinhalten. Daher wurden im Movebis-Projekt Machine-Learning-Ansätze entwickelt, um unplausible Datenpakete zu erkennen und von der Visualisierung auszuschließen. Dennoch werden auch die unplausiblen Daten vom Smartphone aufgezeichnet und zunächst an den Server übermittelt, wobei wegen der aktiven Internet- und GPS-Nutzung sowohl Bandbreite, als auch Energie verbraucht wird.
Ziel dieser Arbeit ist es, ein auf Smartphones lauffähiges Machine-Learning-Konzept für die automatisierte Klassifikation des genutzten Verkehrsmittels, vorrangig des Fahrradfahrens, auf Grundlage der Kontextinformationen aus den STADTRADELN-Datensätzen zu entwickeln. Dazu müssen die STADTRADELN-Datensätze systematisch bezüglich ihrer Charakteristika analysiert werden. Außerdem müssen konkrete Machine-Learning-Ansätze mit Hinblick auf die Limitationen von Smartphones diskutiert werden. Aufbauend auf dieser Grundlage soll eine prototypische App implementiert werden, anhand derer das Konzept evaluiert werden soll.
}{%
\begin{itemize}
\item Recherche und Vergleich von Machine-Learning-Methoden für Smartphones mit Hinblick auf deren Limitationen
\item Anforderungsanalyse und Konzeption eines Machine-Learning-Konzeptes für die Klassifikation der Datenpakete
\item Prototypische Implementation der Datenerfassung und des klassifizierenden Machine-Learning-Konzeptes in einer App
\item Evaluation der Konzepte anhand der prototypischen Implementation
\end{itemize}
}
\end{document}
