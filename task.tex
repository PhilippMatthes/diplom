\documentclass[ngerman]{tudscrreprt}
\ifpdftex{
\usepackage[T1]{fontenc}

\usepackage{draftwatermark}
\SetWatermarkText{Entwurf}
\SetWatermarkScale{1}
\SetWatermarkColor[gray]{0.9}

\RequirePackage[ngerman=ngerman-x-latest]{hyphsubst}
\usepackage{babel}
\usepackage{isodate}
\usepackage{tudscrsupervisor}
\usepackage{enumitem}\setlist{noitemsep}

\usepackage[
  style=alphabetic,
  backend=biber,
  url=false,
  doi=false,
  isbn=false,
  hyperref,
]{biblatex}
% configure the location of the biblatex file
\addbibresource{bibliography.bib}
\AtEveryBibitem{%
  \clearfield{note}%
}
\AtEveryBibitem{\clearfield{url}}
\AtEveryBibitem{\clearfield{number}}
\AtEveryBibitem{\clearfield{abstract}}
\AtEveryBibitem{\clearfield{pages}}
\AtEveryBibitem{\clearlist{language}}

\begin{document}
\faculty{Fakultät Informatik}
\institute{Institut für Systemarchitektur}\chair{Professur für Rechnernetze}
\title{%
Erstellung und Evaluation eines Portierungskonzeptes für Machine-Learning-Ansätze zur Aktivitätsklassifikation auf Smartphones
}
\thesis{diploma}\graduation[Dipl.-Inf.]{Diplom-Informatiker}
\author{%
Philipp Matthes\matriculationnumber{4605459}%
\dateofbirth{12.3.1997}\placeofbirth{Chemnitz}%
}\matriculationyear{2016}\issuedate{1.1.1970}\duedate{1.1.1970}
\supervisor{Dr.-Ing. Thomas Springer}\professor{Prof. Dr. rer. nat. habil. Dr. h. c. Alexander Schill}
\taskform[pagestyle=empty]{%
In der STADTRADELN\footnote{\url{https://www.stadtradeln.de}}-App werden GPS- und Sensordaten von Radfahrenden gespeichert und für die Weiterverarbeitung an einen Server übermittelt. Die protokollierten Datensätze werden unter anderem durch das Forschungsprojekt Movebis\footnote{\url{https://www.movebis.org}} genutzt, mit dem Ziel, die Planung der Radverkehrsinfrastruktur zu verbessern. Bei Analysen der Datensätze konnte festgestellt werden, dass diese auch Datenpakete mit für Radfahrer untypischen Parameterwerten beinhalten. Daher wurden im Movebis-Projekt Machine-Learning-Ansätze\footnote{\fullcite{matusek_anwendung_2019}}\footnote{\fullcite{stojanov_continuous_2020}}\footnote{\fullcite{werner_kontinuierliche_2020}} entwickelt, um unplausible Datenpakete zu erkennen und von der Visualisierung auszuschließen. Dennoch werden auch die unplausiblen Daten vom Smartphone aufgezeichnet und zunächst an den Server übermittelt, wobei wegen der aktiven Internet- und GPS-Nutzung sowohl Bandbreite, als auch Energie verbraucht werden.
Ziel dieser Arbeit ist es, anhand der bestehenden Machine-Learning-Ansätze zu untersuchen, inwiefern diese auf Smartphones portiert werden können. Hierzu müssen die bestehenden Ansätze analysiert werden, um darauf aufbauend ein Portierungskonzept zu entwickeln. Teil dessen ist eine Recherche von Problemen und Lösungsstrategien aus verwandten Arbeiten aktueller Forschung. Das Portierungskonzept soll anschließend anhand einer prototypischen Implementation im Rahmen einer Smartphone-App evaluiert und mit den bestehenden Machine-Learning-Ansätzen verglichen werden.
}{%
}
\end{document}
