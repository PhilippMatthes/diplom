% the following command is only required if the thesis is written in german
\RequirePackage[ngerman=ngerman-x-latest]{hyphsubst}

\documentclass[
  ngerman, % change to ngerman for german theses
  symmetric, % use two-side for booklike layouts
  cdfont=off, % modify this to use the TUD font (Open Sans)
  headings=optiontohead,
  numbers=noenddot % remove trailing dots in chapter/section/... enumeration
]{tudscrreprt}
% use a custom serif font
\usepackage[bitstream-charter]{mathdesign}
% make chapter, section, ... use serif font
\addtokomafont{disposition}{\rmfamily}

\usepackage[T1]{fontenc}
\usepackage[utf8]{inputenc}
\usepackage[
  ngerman % change to ngerman for german theses
]{babel}
\usepackage{isodate}
\usepackage{pdfpages}
\usepackage{listings}
\usepackage[toc, page]{appendix}

\usepackage{hyphenat}

\hyphenation{STADT-RA-DELN}

\emergencystretch 3em

\usepackage[
  style=alphabetic,
  backend=biber,
  url=false,
  doi=false,
  isbn=false,
  hyperref,
]{biblatex}
% configure the location of the biblatex file
\addbibresource{bibliography.bib}
\AtEveryBibitem{%
  \clearfield{note}%
}

% make all links clickable but hide ugly boxes
\usepackage[hidelinks]{hyperref}
\usepackage{amsmath}
% automatically insert Fig. X in the text with \cref{..}
\usepackage[capitalise,nameinlink,noabbrev]{cleveref}

\usepackage[colorinlistoftodos,prependcaption,textsize=tiny]{todonotes}

\usepackage{graphicx}
\graphicspath{ {./images/} }

\usepackage{svg}

% if you need mathy stuff
\newtheorem{lem}{Lemma}
\crefname{lem}{Lemma}{Lemmas}
\newtheorem{thm}{Theorem}
\crefname{thm}{Theorem}{Theorems}
\newtheorem{defs}{Definition}
\crefname{defs}{Def.}{Defs.}

\usepackage{blindtext}

%\usepackage{tudscrsupervisor} % if you want to copy the sources of the task description into the thesis

\usepackage{csquotes}

\usepackage{caption}
\captionsetup{font=normalfont,labelfont=normalfont,labelsep=space}
\usepackage{floatrow}
\floatsetup{font=normalfont}
\floatsetup[table]{style=plaintop}
\captionsetup{singlelinecheck=off,format=hang,justification=raggedright}
\DeclareCaptionSubType[alph]{figure}
\DeclareCaptionSubType[alph]{table}
\captionsetup[subfloat]{labelformat=brace,list=off}

\usepackage{booktabs}
\usepackage{array}
\usepackage{tabularx}
\usepackage{tabulary}
\usepackage{tabu}
\usepackage{longtable}
\usepackage{multirow}

\usepackage{quoting}

\usepackage[babel]{microtype}

\usepackage{xfrac}

\usepackage{enumitem}
\setlist[itemize]{noitemsep}

\usepackage{ellipsis}
\let\ellipsispunctuation\relax

\usepackage{pifont}
\newcommand{\cmark}{\ding{51}}
\newcommand{\xmark}{\ding{55}}
\newcommand{\omark}{\ding{108}}

\input{lst.tex} % code styles (listings)

% use this custom theorem for research questions
\newtheorem{researchquestion}{Forschungsfrage}
\crefname{researchquestion}{Forschungsfrage}{Forschungsfragen}

% use this custom environment for equations
\newenvironment{conditions}
  {\par\vspace{\abovedisplayskip}\noindent\begin{tabular}{>{$}l<{$} @{${}={}$} l}}
  {\end{tabular}\par\vspace{\belowdisplayskip}}

\usepackage{float}

% configure the name of your appendix
\renewcommand\appendixtocname{Anhang}
\renewcommand\appendixpagename{Anhang}

% use \tocless before a chapter/section/... in the
% appendix to hide it from the toc
\newcommand{\nocontentsline}[3]{}
\newcommand{\tocless}[2]{
  \bgroup\let\addcontentsline=\nocontentsline#1{#2}\egroup
}

\usepackage[autooneside=false,automark,headsepline]{scrlayer-scrpage}
\renewcommand*{\headfont}{\normalfont}
\clearpairofpagestyles
\ihead{\leftmark}
\ohead{\ifstr{\leftmark}{\rightmark}{}{\rightmark}}
\cfoot*{\pagemark}

\usepackage{booktabs}
\usepackage{xparse}
\NewDocumentCommand{\rot}{O{45} O{1em} m}{\makebox[#2][l]{\rotatebox{#1}{#3}}}%

\setcounter{tocdepth}{3}
\setcounter{secnumdepth}{3}

\begin{document}

  % use uppercase roman letters for all pages until the introduction
  % this way, it is easier to identify how many pages the thesis has
  \pagenumbering{Roman}

  \faculty{Fakultät Informatik}
  \department{}
  \institute{Institut für Systemarchitektur}
  \chair{Professur für Rechnernetze}
  \title{%
    Erstellung und Evaluation eines Portierungskonzeptes für Machine-Learning-Ansätze zur Aktivitätsklassifikation auf Smartphones
  }
  \subtitle{%
    INF-D-960 Analyse eines Forschungsthemas
  }

  % \thesis{diploma} % the type of thesis you want to write

  \author{Philipp Matthes}
  \matriculationnumber{4605459}
  \matriculationyear{2016}
  \dateofbirth{12.3.1997}
  \placeofbirth{Chemnitz}

  \course{Diplom Informatik (PO 2010)}

  \supervisor{%
    Dr.-Ing. Thomas Springer
  }
  \professor{Prof. Dr. rer. nat. habil. Dr. h. c. Alexander Schill}
  % \date{7.3.2021} % the date of submission
  \maketitle

  \newpage

  \includepdf[pages=-]{task.pdf}

  % for the order of the following sections please refer to
  % the recommendations for thesis structuring

  \confirmation

  % TODO: Vorwort (DA)

  \tableofcontents

  \listoftables
  \addcontentsline{toc}{chapter}{\listtablename}

  \listoffigures
  \addcontentsline{toc}{chapter}{\listfigurename}

  % this is where your thesis lives

  \chapter{Einleitung}\label{ch:einleitung}\pagenumbering{arabic}

\section{Gegenstand und Motivation}\label{sec:gegenstand-und-motivation}

% Darstellung des Kontextes der Arbeit

STADTRADELN\footnote{\url{https://www.stadtradeln.de/darum-geht-es} (Abgerufen am 25.2.2021)} ist ein Projekt, bei dem Radfahrende teilnehmen können, um kollektiv mit dem Rad gefahrene Kilometer für den Klimaschutz zu sammeln. Begleitet wird das Projekt durch eine mobile App, mithilfe derer Nutzende ihre gefahrenen Kilometer tracken und sich mit anderen Teilnehmenden vergleichen können. Im Rahmen des Forschungsprojektes Movebis\footnote{\url{https://www.movebis.org/das-projekt/} (Abgerufen am 25.2.2021)} werden die hierbei erhobenen Daten ausgewertet und mithilfe von konkreten Kenngrößen visuell aufbereitet. Die aufbereiteten Daten werden der kommunalen Verkehrsplanung anschließend zur Verfügung gestellt, mit dem Ziel, die Planung der Radverkehrsinfrastruktur zu verbessern.
\\

% Fokussierung auf ein konkretes Teilgebiet des Kontextes + Motivation

\noindent Die von der STADTRADELN-App erhobenen Daten\footnote{\url{https://www.stadtradeln.de/datenschutz} (Abgerufen am 25.2.2021)} werden zu einem Server hochgeladen und von diesem für die spätere Auswertung protokolliert. Hierbei können Nutzende selbst entscheiden, zu welchen Zeitpunkten die Datenerfassung gestartet und beendet wird. Es existiert jedoch nach aktuellem Stand kein Kontrollmechanismus für die App, mithilfe dessem verifiziert werden kann, dass sich ein Nutzender noch mit dem Fahrrad bewegt. Daher ist nicht ausgeschlossen, dass Datenpakete außerhalb des gewünschten Zielkontextes (Fahrrad fahren) zum Server übermittelt werden. Im Umfeld des Movebis-Projektes wurden daher bereits verschiedene Machine-Learning-Ansätze entwickelt, durch deren Einsatz für den Zielkontext plausible von unplausiblen Datensätzen unter Berücksichtigung einer bestimmten Fehlerrate getrennt werden können \cite{matusek_anwendung_2019, stojanov_continuous_2020, werner_kontinuierliche_2020}. Die entwickelten Machine-Learning-Ansätze ermöglichen ein Filtering der Datensätze zum Zeitpunkt der Auswertung, reduzieren jedoch \textit{nicht} den initialen Transfer der unplausiblen Datensätze vom mobilen Endgerät zum Server.
\\
% Erläuterung des konkreten Gegenstands der Arbeit + Motivation + Relevanz

\noindent Mithilfe eines App-seitigen Erkennungsmechanismus für die Zugehörigkeit der Datenpakete zum gewünschten Zielkontext könnten nicht zugehörige Datenpakete vom Netzwerktransfer ausgeschlossen werden, um die genutzte Bandbreite und Energie auf dem mobilen Endgerät zu sparen. Durch die App-seitige Klassifikation des Kontextes wäre es außerdem möglich, Nutzende bei der Beendigung des Radfahrens an die noch laufende In-App-Aktivität, beispielsweise über Notifikationen, zu erinnern. Eine solche Kontextklassifikation wird beispielsweise bereits im Betriebssystem der Apple Watch seit 2018 (mit Release von WatchOS 5) durchgeführt, um sportliche Aktivitäten zu erkennen und in Form von \enquote{Workout Reminders} in das Interaktionsschema zu integrieren\footnote{\url{https://support.apple.com/en-us/HT204523} (Abgerufen am 25.2.2021)}. Die Anwendungsgebiete einer solchen App-seitigen Kontextklassifikation beschränken sich somit nicht nur auf das App-seitige Filtering von Datenpaketen, sondern ermöglichen auch die Integration von weiteren Interaktionsschemata.

\section{Problem- und Zielstellung}\label{sec:problem-und-zielstellung}

Für die Realisation einer solchen Kontextklassifikation (über die Abbildung der GPS- und Sensordaten auf den Kontext) soll ein Machine-Learning-Konzept erstellt werden, anhand dessen diskutiert und evaluiert werden soll, inwiefern dieses für ein App-seitiges Filtering von Datenpaketen und andere Interaktionsschemata auf dem Smartphone genutzt werden kann. Zur differenzierten Diskussion dieses zentralen Forschungsgegenstands sollen folgende konkrete Forschungsfragen eruiert werden:

\begin{researchquestion}\label{rq1}
Wie sind die im Rahmen der STADTRADELN-App erhobenen Datenpakete strukturiert und welche Attribute eignen sich am besten für eine Kontextklassifikation?
\end{researchquestion}

\begin{researchquestion}\label{rq2}
Welche Abweichungen und Fehler sind innerhalb der erfassten Sensordaten zu erwarten und welche Charakteristika oder Muster können zu einer Kontextklassifikation herangezogen werden?
\end{researchquestion}

\begin{researchquestion}\label{rq3}
Welche hardware- und softwaretechnisch limitierenden Faktoren beschränken die Selektion von konkreten Machine-Learning-Methoden für den Einsatz auf Smartphones gegenüber dem Einsatz in der Datenauswertung?
\end{researchquestion}

\begin{researchquestion}\label{rq4}
Welche Machine-Learning-Methoden können unter Betrachtung von \Cref{rq3} für die Kontextklassifikation genutzt werden?
\end{researchquestion}

\begin{researchquestion}\label{rq5}
Wie können die gewählten Machine-Learning-Methoden im Rahmen eines Gesamtkonzeptes mit Fokus auf die Verkehrsmittelerkennung angewandt und implementiert werden?
\end{researchquestion}

\begin{researchquestion}\label{rq6}
Welche quantitativen Effekte hat der Einsatz eines Konzeptes analog zu \Cref{rq5} auf die Reduktion der Übertragung von unplausiblen Datenpaketen und inwiefern kann das Konzept auf andere Kontexte erweitert werden?
\end{researchquestion}

\paragraph{Erkenntnisgewinn und Relevanz der obigen Forschungsfragen:} Anhand von \Cref{rq1} und \Cref{rq2} soll ein detailliertes Verständnis für die Struktur und Eigenschaften der STADTRADELN-Daten als Grundlage für die Kontextklassifikation erarbeitet werden. Durch \Cref{rq3} soll ein systematischer Überblick gegeben werden, welche zusätzlichen Herausforderungen App-seitiges (engl. \enquote{on-device}) Machine-Learning einer Wiederverwendung der bestehenden Auswertungssoftware aus dem Movebis-Projekt entgegenstellt. Anschließend soll im Rahmen von \Cref{rq4} geklärt werden, welche konkreten Machine-Learning-Methoden sich prinzipiell für eine App-seitige Kontextklassifikation eignen, um für die Beantwortung von \Cref{rq5} eine empirische Auswahl aus den eruierten Machine-Learning-Methoden in einer Software-Architektur zu kombinieren und zu implementieren. Mithilfe der implementierten Software-Architektur soll \Cref{rq6} untersucht und evaluiert werden, um perspektivisch einen möglichen Einsatz in der STADTRADELN-App oder Apps mit ähnlichem Anwendungsgebiet zu beurteilen.

\section{Aufbau der Arbeit}\label{sec:aufbau-der-arbeit}

\paragraph{Grundlagen:} Zu Beginn der Arbeit werden zunächst grundlegende domänenspezifische Begriffe und Zusammenhänge erläutert. Hierzu wird ein taxonomischer Überblick über Klassifikationssysteme gegeben, sowie welche grundlegenden Ideen diesen zugrundeliegen. Anschließend werden die Grundkonzepte des Machine-Learnings in diese Domäne eingeordnet und wichtige Anwendungsgebiete, Architekturen, Lernmodelle, Test- und Validierungsstrategien und hierbei auftretende Herausforderungen diskutiert. Auf Grundlage dessen werden verschiedene Frameworks gegenübergestellt, mit deren Hilfe Machine-Learning-Ansätze auf Smartphones portiert oder implementiert werden können. Um das Grundlagen-Kapitel zu konkludieren, werden die STADTRADELN-Datenpakete bezüglich ihrer summativen und individuellen Attribute betrachtet.

\paragraph{Verwandte Arbeiten:} An die Betrachtung der Grundlagen schließt sich eine Diskussion verwandter Arbeiten an, insbesondere sollen hierbei neben der verwandten Fachliteratur auch die Arbeiten betrachtet werden, welche im Rahmen der Konzeption und Implementation der Machine-Learning-Ansätze zur Auswertung der Datenpakete im Movebis-Projekt erstellt wurden. Im dazugehörigen Kapitel sollen hierzu verschiedene Forschungsansätze zunächst im Überblick beschrieben werden, um schließlich selektiv einige Forschungsansätze näher, auch anhand der Eignung für eine Anwendung im Rahmen dieser Arbeit, zu diskutieren.

\paragraph{Analyse:} Basierend auf den Grundlagen und verwandten Arbeiten soll anschließend eine zielorientierte Analyse geeigneter Machine-Learning-Methoden durchgeführt werden, deren zentrale Bestandteile auch die Analyse der Datengrundlage und die Ermittlung der Klassifikationsanforderungen sind. Außerdem sollen die analysierten Machine-Learning-Methoden bezüglich einer Eignung für die Implementation auf Smartphones bewertet werden, indem konkrete hardware- und softwaretechnische Limitationen analysiert werden.

\paragraph{Konzeption:} Mithilfe der analysierten Machine-Learning-Ansätze soll ein konkretes Konzept erarbeitet und vorgestellt werden, welches eine Klassifikation des Smartphone-Kontextes anhand des Beispiels der STADTRADELN-Datenpakete umsetzt. In diesem Rahmen soll zu Visualisierungs-, Evaluations- und Testzwecken eine App konzipiert werden, in welcher Datenpakete analog zur STADTRADELN-App erhoben werden können.

\paragraph{Implementation und Evaluation:} Die konzipierte App soll anschließend implementiert werden, wobei der entwickelte Machine-Learning-Ansatz analog zum zuvor entwickelten Konzept mit in die App integriert werden soll. Hierbei sollen zusätzliche Interaktionsschemata aufgezeigt werden, welche über das Paket-Filtering genutzt werden können, um die Menge der abgeschickten Datenpakete außerhalb des Zielkontextes weiterhin zu reduzieren. Um den implementierten Ansatz nachfolgend systematisch und reproduzierbar zu evaluieren, sollen im Rahmen eines Evaluationskonzeptes konkrete Testumgebungen und -parameter festgelegt werden. Mithilfe dieser Testbedingungen soll eine quantitative Analyse der Klassifikation durchgeführt werden, auf deren Grundlage schließlich beurteilt werden soll, inwiefern sich der entwickelte Machine-Learning-Ansatz für die Anwendung des App-seitigen Paket-Filterings eignet.

  \chapter{Grundlagen}\label{ch:grundlagen}

\begin{figure}[h]
\includegraphics[width=\linewidth, bb=0 0 631 204]{grundlagen.pdf}
\caption{Systematischer Überblick über Methoden zur Verkehrsmittelerkennung. Darstellung angelehnt an \cite[S. 16]{demrozi_human_2020}.}\label{fig:grundlagen}
\end{figure}

In diesem Kapitel sollen die zentralen Motivationen und Ideen hinter einer Verkehrsmittelerkennung auf Smartphones diskutiert werden. Wie in \Cref{fig:grundlagen} gezeigt, ist eine Verkehrsmittelerkennung auf Smartphones als abstraktes System grob in vier Schritte auftrennbar. Zunächst werden die Kontextdaten der aktuellen Aktivität durch das Smartphone aufgenommen, um dann vorverarbeitet zu werden und als Eingabe für eine Mustererkennung zu dienen. Diese Mustererkennung muss ggf. trainiert und anschließend evaluiert werden. Ausgabe der Mustererkennung soll das genutzte Verkehrsmittel sein. Bei der Portierung wird die Musterkennung zusammen mit den notwendigen Vorverarbeitungsmethoden auf das Smartphone transferiert. dockerDas Grundlagenkapitel untergliedert sich analog zur Übersicht in aufeinander aufbauende Sektionen. Zunächst wird eruiert, wie die zugrundeliegenden Daten erfasst werden, welche Probleme hieraus entstehen und wie die Daten für eine Klassifikation vorbereitet werden können. Anschließend werden heuristische und auf Machine-Learning basierende Ansätze diskutiert, welche zur Mustererkennung genutzt werden können. Zum Schluss wird gezeigt, welche Konzepte und Ideen hinter einer Portierung von Machine-Learning-Ansätzen auf Smartphones stehen.

\section{Datenerfassung und -verarbeitung}\label{sec:datenerfassung-und-verarbeitung}

% \begin{figure}[H]
% \includegraphics[width=\linewidth, bb=0 0 598 404]{datengrundlage.pdf}
% \caption{Die bei der Nutzung der STADTRADELN-App erhobenen Daten im Überblick.}\label{fig:datengrundlage}
% \end{figure}
%
% \Cref{fig:datengrundlage} zeigt die in der Datenschutzerklärung der STADTRADELN-App beschriebenen Datensatzstrukturen\footnote{\url{https://www.stadtradeln.de/datenschutz/} (Abgerufen am 3.4.2021)} im Überblick.

Bei der Nutzung der STADTRADELN-App werden unterschiedliche Datensätze erhoben, verarbeitet und gespeichert\footnote{\url{https://www.stadtradeln.de/datenschutz/} (Abgerufen am 3.4.2021)}. Neben allgemeinen personenbezogenen Daten werden auch Daten für die Meldeplattform \enquote{RADar!}\footnote{\url{https://www.radar-online.net/home} (Abgerufen am 3.4.2021)} verarbeitet, bei der vor allem die Markierung von Orten für die Ausbesserung der Radverkehrsinfrastruktur im Vordergrund steht. Die im Rahmen dieser Arbeit unabhängig von anderen Datensätzen betrachteten \textit{Aktivitätsdaten} sind die folgenden:

\begin{itemize}
  \item GPS-Koordinaten des jeweiligen Smartphone-Geolokalisationssystems, bestehend aus geografischer Länge und Breite, Höhe, Datum und Zeit, sowie einer Genauigkeitsangabe und der aktuellen Geschwindigkeit
  \item Beschleunigungsdaten des Smartphone-Akzelerometers in drei Raumdimensionen
  \item Neigungswinkeldaten des Gyroskop-Sensors in Winkelauslenkungen dreier Rotationsachsen
  \item Orientierungsdaten des Magnetometer-Sensors in drei Raumdimensionen
  \item Weitere Hilfsdaten, wie eine Geräte-Identifikationsnummer und das Smartphone-Modell
\end{itemize}

\noindent Diese Messparameter können als Grundlage für eine im Einleitungskapitel motivierte Aktivitätserkennung fungieren. Daher sollen diese nun genauer betrachtet werden. Dabei spielen insbesondere folgende Fragen eine zentrale Rolle:

\begin{enumerate}
\item Wie werden die jeweiligen Messparameter technisch erfasst und mit welchen Wertebereichen ist hierbei zu rechnen?
\item In welchen Dimensionen sind die Messparameter strukturiert und welche Auskunft geben sie?
\item Welche Probleme und Herausforderungen können bei der Erfassung der Messparameter auftreten und wie können diese gelöst werden?
\end{enumerate}

\noindent Diese Fragen sollen in den folgenden Sektionen diskutiert werden.

\subsection{Lokale kinematische Messparameter}\label{sec:kinematische-messparameter}

Neben GPS-Geolokalisationsinformationen werden in der STADTRADELN-App Aktivitätsdaten mithilfe von Sensoren des Smartphones aufgezeichnet.

\begin{figure}[H]
\includegraphics[width=\linewidth, bb=0 0 540 181]{kinematische-messparameter.pdf}
\caption{Übersicht über die kinematischen Messparameter der STADTRADELN-Daten.}\label{fig:kinematische-messparameter}
\end{figure}

\noindent Wie in \Cref{fig:kinematische-messparameter} gezeigt, handelt es sich hierbei um jeweils triaxiale Messparameter in Form von Beschleunigungsdaten des Akzelerometers, Neigungsdaten des Gyrosensors, sowie Orientierungsdaten des Magnetometers.

\subsubsection{Funktionsweise}

Das Akzelerometer des Smartphones basiert auf einem mikroelektronischen mechanischem System \textit{(MEMS)}, welches die mechanische Beschleunigung des Smartphones in vom Mikrochip interpretierbare elektrische Signale wandelt, durch die Auslenkung sehr kleiner gefederter Massen \cite{matej_andrejasic_mems_2008, constantinescu_capacitive_2013, dey_accelprint_2014}. Um neben der Beschleunigung auch die Winkelauslenkung (Neigung) des Gerätes zu messen, wird ein piezoelektrischer Gyrosensor eingesetzt \cite{singh_piezoelectric_2007}. Hierbei induziert die Auslenkung des Gerätes eine Spannung über den sogenannten Piezoeffekt \cite{singh_piezoelectric_2007, ichimura_fem_2002, koitabashi_improvement_2002}. Neben dem Gyrosensor und dem Akzelerometer wird zusätzlich auch das Magnetfeld über ein Magnetometer gemessen, dabei handelt es sich um ein mikroelektronisches optisches Messsystem auf Grundlage von Alkalimetalldämpfen \cite{schwindt_chip-scale_2004, dmitry_budker_alkali_2021}. Anhand der Magnetfelddaten können die triaxialen Orientierungsdaten abgeleitet werden, über die Ausrichtung des Erdmagnetfelds.

\subsubsection{Herausforderungen}

Bei Beschleunigungsdaten des Akzelerometers ist häufig die Erdschwerkraft im triaxialen Vektorsystem integriert \cite{ken_taylor_activity_2011, tundo_correcting_2013}. Bei einer Neigungsänderung des Smartphones wird somit auch immer eine Änderung des Gravitationsvektors gemessen. Dies erschwert die Interpretation der gemessenen Daten, da die gemessenen Werte zunächst in ein von der Neigung des Gerätes unabhängiges Bezugskoordinatensystem, das sogenannte Weltkoordinatensystem, gebracht werden müssen.

\begin{figure}[H]
\includegraphics[width=\linewidth, bb=0 0 425 162 ]{desync.pdf}
\caption{Schematische Darstellung der zeitlichen Desynchronisation von Sensordaten, angelehnt an \cite{matusek_anwendung_2019}.}\label{fig:desync}
\end{figure}

\noindent Ebenfalls erschwerend wirken sich von Gerät zu Gerät unterschiedliche inhärente Messfehler und -parameter der Sensoren aus. Wie in \Cref{fig:desync} gezeigt kann beispielsweise die Abtastrate der Sensoren und der genutzten Systemschnittstellen zur Aufzeichnung variieren oder eine verzögerte Aufzeichnung auftreten \cite{matusek_anwendung_2019}. Bei der Messung der Akzelerometer-Daten ist mit einem signifikanten Rauschen der Sensorwerte zu rechnen \cite{dey_accelprint_2014, ravi_deep_2016}. Dies ist übergreifend zurückzuführen auf unterschiedliche Imperfektionen in der Fertigung und die altersbedingte Degradation von mikroelektronischen Chips \cite{matej_andrejasic_mems_2008, constantinescu_capacitive_2013, hillman_manufacturing_2004}. Eine Aktivitätserkennung sollte möglichst nicht durch diese Rauschmuster beeinflusst werden \cite{dey_accelprint_2014}. Durch die typische inhärente Abweichung der Sensordaten vom tatsächlichen Wert eignen sich die kinematischen Daten auch nur begrenzt für eine Bewegungsrekonstruktion und Lokalisation durch zeitliche Aggregation im Raum. Je länger die Aktivitätsdaten, speziell Ausrichtungsdaten, aggregiert werden, desto größer ist die beobachtete Gesamtabweichung (\textit{Drift}) \cite{takeda_drift_2014}. Neben diesen Herausforderungen besteht mit Hinblick auf die Orientierungsdaten ein weiteres Problem. Da das Erdmagnetfeld (ca. $30 \mu T$ bis $60 \mu T$) im Vergleich zu einigen künstlichen Magnetfeldquellen wie Kühlschränken (ca. $5 mT$) relativ schwach ist, kann das Magnetometer leicht gestört oder sogar dekalibriert werden \cite{moreno-torres_evaluation_2013, schirmer_smartphone_2016, zhang_preliminary_2012}, was zu signifikanten Fehlern führen kann \cite{kunze_can_2010}.

\subsubsection{Signalverarbeitungsmethoden}

Die beschriebenen Sensorfehler können offensichtlich ein Problem für die Interpretation der Signale darstellen. Es folgen Ideen zur Mitigation.

\begin{figure}[H]
\includegraphics[width=\linewidth, bb=0 0 733 284]{frequenzpassfilter.pdf}
\caption{Anwendung von Frequenzbandfiltern auf Rohsensordaten des Akzelerometers. Datenquelle: Eigenes Sample eines iPhone XS-Akzelerometers.}\label{fig:frequenzpassfilter}
\end{figure}

\noindent Eventuell auftretendes Rauschen kann durch Tiefpassfilter und Hochpassfilter selektiv eliminiert werden. Hierbei eliminiert ein Tiefpassfilter hochfrequentes Rauschen und ein Hochpassfilter niederfrequentes Rauschen (siehe \Cref{fig:frequenzpassfilter}). Bei dem Butterworth-Tiefpassfilter\footnote{\url{https://de.wikipedia.org/wiki/Butterworth-Filter} (Abgerufen am 7.4.2021)} handelt es sich um ein spezielles Tiefpassfilter, welches zur Eliminierung von Frequenzen über einem Schwellenwert $\omega_{0}$ und zur Reduktion der Kurvenpunkte bei Erhaltung des Kurvenverlaufs genutzt werden kann \cite{zuricher_hochschule_fur_angewandte_wissenschaften_kapitel_2009}. Die Auswahl des Filters setzt die Evaluation potenzieller Rauschmuster, zum Beispiel durch explorative Analyse des Datensatzes unter Betrachtung des Anwendungsfalls, voraus \cite{nutter_design_2018,takeda_drift_2014,abdulla_measuring_2013}.

\begin{figure}[H]
\includegraphics[width=\linewidth, bb=0 0 724 284]{glaettingsfilter.pdf}
\caption{Anwendung von Glättungsfiltern auf Rohsensordaten des Akzelerometers. Datenquelle: Eigenes Sample eines iPhone XS-Akzelerometers.}\label{fig:glaettingsfilter}
\end{figure}

\noindent Neben Hoch- und Tiefpassfiltern besteht weiterhin auch die Möglichkeit, die Signale zu glätten, um hochfrequente Signalschwankungen und -spitzen zu eliminieren. Hierbei können verschiedene Glättungsfilter zum Einsatz kommen. \cite{matusek_anwendung_2019} behandelt beispielsweise das in \Cref{fig:glaettingsfilter} gezeigte und unter den Namen \textit{Moving Average} bekannte Filterverfahren, bei dem die umliegenden Datenpunkte (Länge $k$) mit dem zu glättenden Datenpunkt gemittelt werden. Es existieren zahlreiche weitere Glättungsverfahren, darunter auch das von \cite{nutter_design_2018} (hier parametrisiert mit $k = 3$) genutzte Median-Filter. Wie in \Cref{fig:glaettingsfilter} sichtbar, wird das Eingangssignal mit einer von der Filterlänge abhängigen Intensität geglättet. Beim Beispielsignal aus \Cref{fig:glaettingsfilter} ist klar zu erkennen, dass hierbei jedoch auch bei einer zu hohen Filterlänge $k$ wichtige Informationen verloren gehen können. Die empirische Parametrisierung ist also auch hier von zentraler Bedeutung für die spätere Ergebnisqualität der weiteren Signalverarbeitung.

\subsubsection{Fourier-Transformation}

Ein weiteres Signalverarbeitungsverfahren ist die \textit{Fourier-Transformation}\footnote{\url{https://de.wikipedia.org/wiki/Fourier-Transformation} (Abgerufen am 8.4.2021)}.

\begin{figure}[H]
\includegraphics[width=\linewidth, bb=0 0 734 339]{fft.pdf}
\caption{Anwendung einer Fast-Fourier-Transformation auf Rohsensordaten des Akzelerometers. Datenquelle: Eigenes Sample eines iPhone XS-Akzelerometers.}\label{fig:fft}
\end{figure}

\noindent Eine weit verbreitete Implementationsvariante ist die in \Cref{fig:fft} exemplarisch gezeigte \textit{Fast-Fourier-Transformation}. Die Fourier-Transformation ist ein mathematisches Verfahren, um zusätzlich zum Zeitlinienverlauf der Signale deren Frequenzspektrum zu ermitteln. Anhand des ermittelten Frequenzspektrums lässt sich sofort ablesen, welche Frequenzen besonders häufig oder sehr selten in einem Frequenzspektrum vorkommen, sowie die gemessene Maximal- und Minimalfrequenz. Dies ist nicht nur relevant für die explorative Datenanalyse, sondern kann auch als zusätzliche Informationsquelle für das Machine-Learning-Modell dienen.

\subsubsection{Sensorfusion}

Die Sensordaten (insbesondere Beschleunigungsdaten) können für eine Weiterverarbeitung einzeln betrachtet werden \cite{gudur_activeharnet_2019, nutter_design_2018}, dies genügt bereits für die Rekonstruktion von komplexen Bewegungsabläufen \cite{cai_touchlogger_2011, marquardt_spiphone_2011, aviv_practicality_2012}. Alternativ können die Sensordaten miteinander \textit{fusioniert} werden \cite{ravi_deep_2016, abdulla_measuring_2013, kunze_can_2010}. Mithilfe einer Fusion der Sensordaten über Filter ist dies möglich \cite{madgwick_estimation_2011}. Speziell bei Beschleunigungsdaten kann hierdurch zum Beispiel auch der inkludierte Gravitationsvektor eliminiert werden \cite{nutter_design_2018, tundo_correcting_2013}. Außerdem kann die Orientierung und Auslenkung des Gerätes in Relation zum Weltkoordinatensystem rekonstruiert werden.

\paragraph{Funktionsweise:} Ziel ist die Schätzung der Lage des Geräts im Weltkoordinatensystem. Die resultierende Schätzung kann zum Beispiel in Form eines Quaternion\footnote{\url{https://en.wikipedia.org/wiki/Quaternion} (Abgerufen am 8.4.2021)} angegeben werden. Für die Schätzung existieren verschiedene mathematische Verfahren. \\

\begin{table}[H]
  \begin{tabular}{cccccccccccccccccll}
    \rot{AQUA \cite{valenti_keeping_2015}}
    & \rot{Complementary}
    & \rot{Davenport \cite{paul_b_davenport_vector_1968}}
    & \rot{EKF \cite{sabatini_kalman-filter-based_2011}}
    & \rot{FAMC \cite{liu_simplified_2018}}
    & \rot{FLAE \cite{wu_super_2018}}
    & \rot{Fourati \cite{fourati_nonlinear_2011}}
    & \rot{FQA \cite{yun_simplified_2008}}
    & \rot{Integration}
    & \rot{\textbf{Madgwick} \cite{madgwick_estimation_2011}}
    & \rot{Mahony \cite{mahony_nonlinear_2008}}
    & \rot{OLEQ \cite{zhou_optimal_2018}}
    & \rot{QUEST \cite{shuster_three-axis_1981}}
    & \rot{ROLEQ \cite{zhou_optimal_2018}}
    & \rot{SAAM \cite{wu_super_2018}}
    & \rot{Tilt \cite{sebastian_trimpe_balancing_2012}}
    & \rot{TRIAD \cite{shuster_optimization_2007}}
    & \\
    \midrule
    \omark & \cmark & \cmark & \cmark & \cmark & \cmark & \cmark & \cmark & \xmark & \cmark & \cmark & \cmark & \cmark & \cmark & \cmark & \cmark & \cmark & ACC \\
    \cmark & \cmark & \xmark & \cmark & \xmark & \xmark & \cmark & \xmark & \cmark & \cmark & \cmark & \xmark & \xmark & \xmark & \xmark & \xmark & \xmark & GYR \\
    \omark & \omark & \cmark & \cmark & \cmark & \cmark & \cmark & \omark & \xmark & \omark & \omark & \cmark & \cmark & \cmark & \cmark & \omark & \cmark & MAG \\
    \bottomrule
  \end{tabular}\label{tab:ahrs}
  \caption{Lageschätzverfahren in der Übersicht, zusammen mit deren Abhängigkeit von Akzelerometerdaten (ACC), Gyrosensordaten (GYR) und Magnetometerdaten (MAG). Entnommen aus \cite{garcia_mayitzinahrs_2021}. Legende: Benötigt (\cmark), Optional (\omark), Nicht einbezogen (\xmark).}
\end{table}

\noindent Ein System, welches einen oder eine Kombination aus solchen Algorithmen nutzt, um die Lage des Gerätes im Weltkoordinatensystem zu rekonstruieren, wird auch als \textit{Attitude and Heading Reference System (AHRS)} bezeichnet.\\

\subsubsection{Synchronisierung und Alignment}

Zur Synchronisierung der möglicherweise von Gerät zu Gerät unterschiedlichen Abtastraten oder von asynchron aufgezeichneten Daten unterschiedlicher Sensoren ist ein Up- oder Downsampling mithilfe einer Interpolation der Datenpunkte möglich \cite{matusek_anwendung_2019, werner_kontinuierliche_2020, stojanov_continuous_2020}.

\begin{figure}[H]
\includegraphics[width=0.75\linewidth, bb=0 0 410 171]{desync-solution.pdf}
\caption{Schematische Darstellung des Alignments von desynchronisierten Datenpunkten, angelehnt an \cite{matusek_anwendung_2019}.}\label{fig:desync-solution}
\end{figure}

\noindent \Cref{fig:desync-solution} zeigt das Verfahren schematisch. Die mit Zeitstempeln versehenen Datenpunkte werden in ein festes Abtastraster gebracht. Dies gelingt über zwei primäre Schritte. Am zeitlichen Rand der Messung werden zunächst Datenpunkte verworfen, zu denen kein Wertepaar aus allen drei Sensoren gebildet werden kann. Ein Wertepaar lässt sich erst dann bilden, wenn alle Sensoren mindestens einen Wert erfasst haben, am Ende der Messung vice versa. Das so gefundene Fenster wird anschließend im zweiten Schritt anhand der vorliegenden Datenpunkte pro Sensormessung interpoliert, um den diskreten Punktdatensatz in eine kontinuierliche Näherungsfunktion zu überführen. Die Abtastzeitpunkte $t_s ... t_{s+n}$ (in \Cref{fig:desync-solution} als $\otimes$ gekennzeichnet) können anschließend für alle drei Messverläufe beliebig gewählt und errechnet werden.

\begin{figure}[H]
\includegraphics[width=\linewidth, bb=0 0 505 175]{interpolation.pdf}
\caption{Schematische Darstellung verschiedener Interpolationsverfahren.}\label{fig:interpolation}
\end{figure}

\noindent Hierbei stehen verschiedene Interpolationsverfahren zur Verfügung, die in \Cref{fig:interpolation} vergleichend gegenübergestellt sind. Je mehr variable Parameter die Interpolation einbezieht, desto enger kann sie den Kurvenverlauf nachbilden, desto größer ist aber auch das Risiko für ungewolltes Überschwingen und der benötigte Rechenaufwand.

\subsection{Geolokalisationsinformationen}

Neben den lokalen kinematischen Messparametern können mit Smartphones auch GPS-Daten erfasst werden.

\subsubsection{Terminologische Einordnung}

Mit GPS ist das \textit{Navigational Satellite Timing and Ranging - Global Positioning System}, kurz \textit{NAVSTAR GPS} gemeint \cite{us_space_force_gpsgov_2021}. Der Begriff GPS wird teils als Synonym für globale Satellitennavigationssysteme (engl. \textit{Global Navigation Satellite System GNSS}) verstanden. Neben dem amerikanischen GPS existieren hierbei jedoch noch weitere Systeme, wie das europäische Galileo, das russische Glonass, sowie das chinesische BeiDou \cite{olynik_temporal_2002}. Moderne Smartphone-Geolokalisationssysteme nutzen in der Regel eine Kombination dieser globalen Satellitennavigationssysteme \cite{european_gnss_service_centre_is_2021, european_gnss_service_centre_usegalileo_2021}. Daher wird in den nachfolgenden Sektionen zwischen GPS (nur NAVSTAR GPS) und GNSS (kombiniert) unterschieden.

\subsubsection{Funktionsweise und Herausforderungen}

\begin{figure}[H]
\includegraphics[width=\linewidth, bb=0 0 565 226]{gps-funktionsweise.pdf}
\caption{Das Funktionsprinzip der GNSS-basierten Geolokalisation und zu erwartende Fehlerquellen.}\label{fig:gps-funktionsweise}
\end{figure}

GNSS-Satelliten senden kontinuierlich Funksignale zur Erde \cite{us_space_force_global_2020}. Mithilfe von speziellen Antennen können diese Signale empfangen und zur Weiterverarbeitung amplifiziert werden \cite{jan_van_sickle_gps_2021}. Als Ergebnis der internen Prozessierung des GPS-Signals kann die Distanz zum GPS-Satelliten anhand der Signallaufzeit unter Berücksichtigung verschiedener Fehlerquellen\footnote{\url{https://www.e-education.psu.edu/geog862/node/1759} (Abgerufen am 7.4.2021)} polynomiell berechnet werden \cite{olynik_temporal_2002, sameet_mangesh_deshpande_study_2004}. Somit stehen anschließend die Position des Satelliten und die berechnete Distanz zur Verfügung. Anhand dessen kann die Position des Gerätes trianguliert werden \cite{zhang_senstrack_2013}. Die Positionsbestimmung ist also passiv, es findet keine aktive Kommunikation mit dem Satelliten statt \cite{kaplan_understanding_2005}. Trotzdem ist der Energieverbrauch in Relation zu anderen Verbrauchern im Smartphone für die GNSS-Geolokalisation sehr hoch. Smartphones nutzen daher zusätzlich zur satellitengestützten Positionsbestimmung Eigenschaften des WiFi- und GSM-Funknetzes, um den Energieverbrauch zu reduzieren \cite{zhang_senstrack_2013}. \\

\noindent Das Abstandsgesetz $I_{Signal} \propto d^{-2}$ bedingt neben weiteren signaldämpfenden Faktoren wie atmosphärischem Scattering, oder Streuung und Brechung an Gebäuden, dass wegen der begrenzten Signalemissionsstärke des Satelliten und der großen Distanz zum Empfänger die Empfindlichkeit des Empfängersystems verhältnismäßig hoch sein muss, in einer Größenordnung von $-130 dBm$ (Dezibel Milliwatt) \cite{sameet_mangesh_deshpande_study_2004}. Die elektronische Prozessierung des Signals erfordert hierdurch eine entsprechende elektrische Leistung \cite{jan_van_sickle_gps_2021}. Die erforderliche elektrische Leistung stellt für Smartphones durch deren Limitation durch die Kapazität des Akkus ein signifikantes Problem dar \cite{zhang_senstrack_2013, oshin_improving_2012, constandache_enloc_2009, zhuang_improving_2010}. Eine energiesparendere Geolokalisation nur über GSM- und WiFi-Funknetze ist weniger akkurat als GNSS und kann daher in Anwendungsfällen mit einer benötigten Mindestpräzision lediglich begleitend eingesetzt werden \cite{paek_energy-efficient_2010}. Aufgrund der Abhängigkeit von Funksignalen sind weitere Probleme die Ortsabhängigkeit \cite{zhang_senstrack_2013, kaplan_understanding_2005}, die Wetterabhängigkeit \cite{paek_energy-efficient_2010}, sowie mögliche Interferenzen und andere Signaltransmissionsstörungen \cite{olynik_temporal_2002, sameet_mangesh_deshpande_study_2004}. Bei der Berechnung der Distanz zum Smartphone müssen unter anderem iono- und troposphärische Ladungseffekte auf die Propagierungsgeschwindigkeit der elektromagnetischen Funksignale berücksichtigt werden, diese schwanken jedoch je nach Tageszeit und Sonnenaktivität\footnote{\url{https://gssc.esa.int/navipedia/index.php/Ionospheric_Delay} (Abgerufen am 7.4.2021)}. Im Rahmen der Anwendungsentwicklung sowie der Entwicklung einer Aktivitätserkennung auf Grundlage der Geolokalisationsinformationen sind diese Probleme in Form einer Korrektur fehlerhafter Datenpunkte zu berücksichtigen, insbesondere wegen der ständig variierenden Präzision \cite{zhang_senstrack_2013, matusek_anwendung_2019}.

\subsubsection{Fehlerkorrektur der GNSS-Datenpunkte}

Grundsätzlich besteht die Möglichkeit, GNSS-Datenpunkte mithilfe von vorliegendem Kartenmaterial und den darin enthaltenen Straßen zu korrigieren. Dieses Verfahren wird auch \textit{Map Matching} genannt \cite[S. 26]{reto_wick_unsicherheiten_2013}. GNSS-Datenpunkte können aber auch einfacher über Filter korrigiert werden \cite{oleg_katkov_how_2018}. Die Idee des Verfahrens ist, den Fehler der GNSS-Datenpunkte über die AHRS-gestützte Rekonstruktion der Bewegung im Raum und die darauf basierende Lageschätzung zu korrigieren, beispielsweise über Kalman-Filter \cite{oleg_katkov_how_2018}. Das Verfahren ähnelt dem Madgwick-Filter in der Funktionsweise. Auch hier dient der aktuelle Zustand zur Schätzung des tatsächlichen Wertes anhand des Inputs, hierbei der jeweilige GNSS-Datenpunkt und die kinematischen Messparameter. Das vorliegende Problem (die Schätzung des tatsächlichen Ortes) ist ein klassischer Anwendungsfall des Kalman-Filters, denn die präzise Messung des Ortes ist wegen der inhärenten Fehler unwahrscheinlich, je nach Definition des Toleranzbereiches. Der gemessene Ort streut um den tatsächlichen Ort. Das konkrete Streuverhalten korreliert mit der Beschaffenheit der Umgebung des Smartphones, insbesondere mit umgebenden Störquellen wie Gebäuden \cite{reto_wick_unsicherheiten_2013}. Eine Standardnormalverteilung kann jedoch zur näherungsweisen Modellierung des GNSS-Streuverhaltens herangezogen werden \cite{laube_how_2011}. Dies bietet die Grundlage für die Filter-basierte Korrektur. Filter wie das Kalman-Filter eignen sich speziell für diesen Anwendungsfall, die tatsächliche Position bei einer normalverteilt streuenden GNSS-Position zu schätzen.

\subsubsection{Optimierung des Energieverbrauchs}

Zur Verbesserung der hybriden Geolokalisation über GPS-, GSM- und WiFi-Signale wurden verschiedene Konzepte entwickelt. Der Energieverbrauch kann signifikant reduziert werden, indem die GPS-Abtastrate gesenkt wird \cite{constandache_enloc_2009, lu_jigsaw_2010}. Hierzu können zusätzliche Informationen des Akzelerometers \cite{oshin_improving_2012} und des Orientierungssensors \cite{zhang_senstrack_2013} genutzt werden, um die GPS-Abtastrate adaptiv zu senken. Außerdem kann die erforderliche Präzision der Geolokalisation zusammen mit der primär genutzten Methode (GPS, GSM, WiFi) adaptiv angepasst werden \cite{lin_energy-accuracy_2010, zhang_senstrack_2013}. Aktuelle Schnittstellen in iOS- und Android-Smartphones greifen auf diese Prinzipien zurück und bieten dem Anwendungsentwickler Möglichkeiten, die benötigte Präzision einzustellen, um den Energieverbrauch zu reduzieren \cite{google_inc_fused_2021, apple_inc_desiredaccuracy_2021}.

\subsection{Normalisierung und Standardisierung der Daten}

Die in den vorigen Sektionen erläuterten Daten besitzen teils stark voneinander unterschiedliche Wertebereiche.

\begin{table}[H]
  \begin{tabular}{ll}
    Sensor & Wertebereich \\
    \midrule
    Akzelerometer & Vielfache der Erdgravitation, negativ und positiv, Einheit: $g = \frac{m}{9.81 s^2}$ \\
    Gyrosensor & Winkelauslenkungen, negativ und positiv, in $rad$ \\
    Magnetometer & Magnetfeldstärke, negativ und positiv, in $\mu T$ \\
    GNSS & Longitude und Latitude, in absoluten Gradzahlen, sowie Geschwindigkeit in $\frac{km}{h}$ \\
    \bottomrule
  \end{tabular}\label{tab:wertebereiche}
  \caption{Zu erwartende Wertebereiche der verfügbaren Sensorwerte.}
\end{table}

\noindent Die hohe Variabilität der zu erwartenden Maxima und Minima dieser Wertebereiche stellt ein Problem für die meisten Machine-Learning-Systeme (siehe \Cref{sec:machine-learning}) dar, in Abhängigkeit von deren Architektur \cite[S. 66]{geron_praxiseinstieg_2018}. Daher ist es üblich, zur Vorverarbeitung der Daten neben den bereits vorgestellten Methoden auch eine Skalierung durchzuführen. Für die Skalierung haben sich zwei mathematische Methoden etabliert. Die Normalisierung (auch als Min-Max-Skalierung bezeichnet) skaliert die Messwerte zwischen dem Maximum und dem Minimum des Wertebereichs.

\begin{equation}\label{eq:minmax}
  norm(d) = \frac{ d - min(D) }{ max(D) - min(D) }
\end{equation}
wobei:
\begin{conditions}
  d \in D \subseteq \mathbf{R}^n & Der gewählte Datenpunkt aus der Datenpunktmenge $D$
\end{conditions}

\noindent Die Skalierung des Datenpunktes $d$ mithilfe der Normalisierung impliziert $0 \leq norm(d) \leq 1$, jedoch nur, wenn $d \in D$. Sollte $d \notin D$ gelten, dann kann $norm(d)$ auch außerhalb $[0, 1]$ liegen, wenn $d \leq min(D)$ oder $d \geq max(D)$. Die Normalisierung ist außerdem nicht robust gegenüber fehlerhaften Einzelwerten. Zur Illustration kann das Beispiel $D = \{1.25, 1.05, 0.93, 120.5\}$ betrachtet werden, bei dem $120.5 \in D$ einen Messfehler darstellt. Infolge des einzelnen fehlerhaften Datenpunktes werden alle korrekten Datenpunkte $d_k \in \{1.25, 1.05, 0.93\} \subseteq D$ mit $norm(d_k) \approx 0$ skaliert. Ein gegenüber solchen Abweichungen robusteres Vefahren ist die Standardisierung.

\begin{equation}\label{eq:standardisierung}
  std(d) = \frac{ d - \mu_D }{ \sigma_D } \\
\end{equation}
wobei:
\begin{conditions}
  d \in D \subseteq \mathbf{R}^n & Der gewählte Datenpunkt aus der Datenpunktmenge $D$ \\
  \mu_D & Das arithmetische Mittel von $D$ \\
  \sigma_D & Die Standardabweichung von $D$
\end{conditions}

\noindent Bei der Standardisierung der Datenpunktmenge $D$ wird dem Datenpunkt $d$ zunächst das arithmetische Mittel $\mu_D$ abgezogen. Ist $d < \mu_D$, dann gilt $d - \mu_D < 0$. Umgekehrt gilt $d - \mu_D > 0$ für $d > \mu_D$. Somit werden die Werte innerhalb der Datenpunktmenge gleichmäßig um den Nullpunkt verteilt. Anschließend wird durch die Standardabweichung $\sigma_D$ dividiert. Gilt $\sigma_D < 1$, so werden die standardisierten Werte vergrößert. Ist $\sigma_D > 1$, führt dies zu einer Verkleinerung der standardisierten Werte. Somit werden die Werte verschiedener Wertebereiche in annähernd denselben Zahlenraum überführt. Gleichzeitig wird der relative Abstand der fehlerhaften Datenpunkte zu korrekten Datenpunkten erhalten. Somit lässt sich ein einheitliches Wertebereichsfenster, beispielsweise $W_F = [F_{min} = -1, F_{max} = 1]$ definieren, außerhalb dessen unplausible Werte verworfen werden können. Alternativ können Werte außerhalb dieses Bereichs mit der Funktion $clamp(d) = \max(F_{min}, \min(d, F_{max}))$ in $W_F$ abgebildet werden.

\section{Machine Learning}\label{sec:machine-learning}

In \Cref{sec:datenerfassung-und-vorverarbeitung} wurde detailliert beschrieben, wie sich die Aktivitätsdaten innerhalb der STADTRADELN-App strukturieren und welche Herausforderungen bei der Weiterverarbeitung der Daten zu beachten sind. In diesem Abschnitt sollen nun Methoden vorgestellt werden, mithilfe derer eine Interpretation der vorverarbeiteten Daten möglich ist.

\subsection{Verkehrsmittelerkennung}

Die Erkennung des Verkehrsmittels anhand der beschriebenen Daten, wie sie \cite{matusek_anwendung_2019}, \cite{werner_kontinuierliche_2020} und \cite{stojanov_continuous_2020} umgesetzt haben, lässt sich als ein Problem der \textit{Human Activity Recognition (HAR)} interpretieren. Hierbei liegen Aktivitätsdaten einer jeweiligen Person vor, anhand dessen die ausgeführte Aktivität klassifiziert werden soll. Die formelle Definition der Verkehrsmittelerkennung ist wie folgt. Sei $D = \{d_0, \dots, d_n\} \subseteq \mathbf{R}^n$ die chronologisch sortierte Menge aller vorverarbeiteten und anhand einer Interpolation auf dieselbe Abtastrate synchronisierten Datenpunkte. Wähle ein zu klassifizierendes Segment $S \subseteq D$ der Länge $n_S$. Dann ist die Abbildung $VME(S): D^{n_S} \rightarrow L$ zu finden, wobei $L$ die Menge der möglichen \textit{Labels} von Verkehrsmitteln wie \textit{Fahrrad}, \textit{Bahn}, \textit{Bus}, \textit{zu Fuß} und weiteren ist. Durch die Aufteilung von $D$ in Segmente fester Länge können anschließend mithilfe der Abbildung $VME(S)$ entlang $D$ die genutzten Verkehrsmittel rekonstruiert werden.

\subsubsection{Segmentierung}

In der obigen Problemdarstellung wurde bereits vorweggenommen, dass die Datenpunkte gleichmäßig in Segmente unterteilt werden sollen, die sich auch überlappen können. Dieses Verfahren wird als Gleitfenstermethode bezeichnet. Über die zeitliche Kombination von mehreren aufeinanderfolgenden Datenpunkten (auch als Zeitlinie bezeichnet) können wertvolle Informationen erhalten werden, wie sich zeitlich wiederholende Bewegungsabläufe \cite[S. 4]{banos_window_2014}. Gleichzeitig sollen Wechsel zwischen den Verkehrsmitteln innerhalb der Datenpunktmenge erkannt werden, was mit einer ganzheitlichen Klassifikation der Datenpunktmenge nicht möglich wäre. \\

\todo{Abbildung der Segmente + Datenpunktvektoren einfügen}

\noindent Die Segmentlänge $n_S$ ist hierbei für die Berechnung der Erkennung von entscheidender Bedeutung. \cite{matusek_anwendung_2019} und \cite{werner_kontinuierliche_2020} betrachtet Segmente von $10s$ bis $180s$ Länge. Je kleiner $n_S$ gewählt wird, desto höher ist die zeitliche Granularität der Erkennung. Gleichzeitig lässt sich vermuten, dass die Präzision der Klassifikation bei kürzeren Segmentlängen geringer ist, wie die experimentellen Daten aus \cite{matusek_anwendung_2019} suggerieren. Wie in \Cref{sec:datenerfassung-und-vorverarbeitung} gezeigt, ist es durch Interpolation der verschiedenen Spuren (GNSS, Akzelerometer, Gyrosensor, Magnetometer) möglich, diese im Rahmen der Synchronisation zu beliebigen Raten abzutasten. Bei näherer Betrachtung fällt hierbei auf, dass \cite{matusek_anwendung_2019} und \cite{werner_kontinuierliche_2020} Datenpunkte mit $1Hz$ (Abtastrate des GNSS-Systems) abtasten, die mit deutlich höheren Frequenzen im dreistelligen Hertz-Bereich aufgenommenen Beschleunigungs-, Auslenkungs- und Orientierungsdaten also stark reduziert werden (Downsampling). Vermutlich bietet dies eine Möglichkeit zur Verbesserung der Konzepte aus \cite{matusek_anwendung_2019} und \cite{werner_kontinuierliche_2020}, unter der Berücksichtigung, dass mit $1Hz$ Abtastrate lediglich Bewegungsabläufe mit Frequenzen $f \leq 0.5Hz$ nach dem Nyquist-Shannon-Abtasttheorem\footnote{\url{https://de.wikipedia.org/wiki/Nyquist-Shannon-Abtasttheorem} (Abgerufen am 14.4.2021)} erhalten bleiben, Frequenzen $f > 0.5Hz$ tragen zum Rauschen der abgetasteten Daten bei. Alternativ wäre daher auch ein Upsampling der korrigierten GNSS-Daten auf eine höhere Frequenz von ca. $100 Hz$ möglich. \cite{banos_window_2014} konnten zeigen, dass sich unter dieser Voraussetzung insbesondere kürzere Segmente von $0.25s$ bis $2s$ Länge für die Aktivitätserkennung eignen, gegenüber den von \cite{matusek_anwendung_2019} und \cite{werner_kontinuierliche_2020} verwendeten Segmentlängen von $10s$ bis $180s$.

\subsubsection{Feature-Engineering}

In den vorigen Sektionen wurde bereits diskutiert, wie Daten entsprechender Smartphone-Sensorsysteme akquiriert werden können, um diese schließlich bezüglich ihrer Fehlercharakteristika vorzuverarbeiten, mithilfe von Methoden zur Filtrierung, Korrektur und Standardisierung. Hieraus werden, wie in der vorigen Sektion erläutert, Segmente der Länge $n_S$ gebildet. Zur besseren Verständlichkeit wurde bisher davon ausgegangen, dass diese Segmente analog zur Abbildungsdefinition $VME(S): D^{n_S} \rightarrow L$ direkt als Input für das Machine-Learning-Modell dienen, um schließlich die Abbildung auf die Label-Menge $L$ zu realisieren. Je nach Architektur des Modells ist es jedoch üblich, einen weiteren Zwischenschritt zu integrieren, der als \textit{Feature-Extraktion} bezeichnet wird. Hierbei werden die im Segment enthaltenen Datenpunkte $d \in D$ auf einen \textit{Feature-Space} $F \in \mathbf{R}^{n_F}$ abgebildet, welcher dann nachfolgend als Input für das Machine-Learning-Modell dient.

\begin{equation}\label{eq:feature-extraktion}
d_{t=0} =
%
  \begin{bmatrix}
  ACC(t=0)_x \\ \vdots \\ MAG(t=0)_z
  \end{bmatrix}
%
   \xrightarrow{\text{Feature-Extraktion}}
%
  \begin{bmatrix}
  | ACC(t=0) | \\ \vdots \\ | MAG(t=0) | \\
  \vdots \\ FEATURE_{n_F}(S, t=0)
  \end{bmatrix}
%
\end{equation}
wobei:
\begin{conditions}
  d_{t=0} & Datenpunkt an der Stelle $t=0$ des Segments $S$
\end{conditions}

\noindent \Cref{eq:feature-extraktion} zeigt das Schema einer solchen Feature-Extraktion. Die bestehenden Daten können zusammengefasst werden, wie beispielsweise $|ACC(t=0)|$ die absolute Länge des Beschleunigungsvektors zusammenfasst. Dies stellt bereits ein simples Feature dar. Gleichzeitig können neue Informationen in Relation zum Segment ergänzt werden, in \Cref{eq:feature-extraktion} als $FEATURE_{n_F}(S, t=0)$ symbolisiert. Der Zweck dieser Abbildung ist die semantische Aufbereitung der Datenpunkte, mit dem Ziel einer Verbesserung der Erkennung. Welche Features hierbei konkret berechnet werden sollen, wird im Prozess des \textit{Feature-Engineering} bestimmt \cite{matusek_anwendung_2019}. Für den Kontext der Verkehrsmittelerkennung lassen sich prinzipiell zwei Gruppen von Features unterscheiden.

\paragraph{Shallow Features \cite{ravi_deep_2017}:} Diese Features werden direkt aus dem vorliegenden Segment abgeleitet und sind typischerweise für alle Datenpunktvektoren eines Segments identisch. Häufig verwendete shallow Features sind der Durchschnitt und die Standardabweichung bestimmter Werte eines Segments \cite{chen_deep_2015, banos_window_2014, kwapisz_activity_2011, jahangiri_applying_2015, nurhanim_classification_2017}. Diese Features beziehen sich direkt auf die Zeitlinie. Sie werden daher auch als \textit{Time Domain Features} eingeordnet \cite{chen_deep_2015}. Alternativ ist es auch üblich, sogenannte \textit{Frequency Domain Features} durch eine vorige Fourier-Transformation des Segments zu berechnen \cite{zeng_convolutional_2014, chen_deep_2015}.

\paragraph{Non-Shallow Features \cite{ravi_deep_2017}:} Während shallow Features verhältnismäßig einfach zu implementieren sind, können sie wiederum möglicherweise keine komplexen Zusammenhänge innerhalb des Segments repräsentieren \cite{abu_alsheikh_deep_2015}. Bei \textit{Non-Shallow Features}, auch bezeichnet als \textit{Data-Driven Features} oder \textit{Deep Learning Features} \cite{abu_alsheikh_deep_2015}, erfolgt die Berechnung durch Zwischenschaltung eines Machine-Learning-Modells, welches bestimmte Eingangswerte des Segments in einen Feature-Vektor überführt (Enkodierung). \cite{ravi_deep_2017} überführen beispielsweise die Segmentdaten in ein Spektrogramm, um dieses über ein Machine-Learning-Modell zu enkodieren. \cite{zeng_convolutional_2014} verwenden sogar mehrere Machine-Learning-Modelle verschiedener Art. \\

\noindent Die obigen Features lassen sich hierbei auch miteinander kombinieren, um weitere Features zu erzeugen. Für die auf Machine-Learning basierte Aktivitätserkennung nutzen \cite{kwapisz_activity_2011} beispielsweise insgesamt 43 verschiedene erzeugte Features. Auch hier ist jedoch wieder zwischen dem potenziellen Informationszugewinn und der erhöhten Dimensionalität des Feature-Vektors und der damit verbundenen Berechnungskomplexität der Aktivitätserkennung abzuwägen. Auch bei den berechneten Features ist zu beachten, dass diese je nach Architektur des Machine-Learning-Modells normalisiert oder standardisiert werden sollten (siehe \Cref{sec:datenerfassung-und-vorverarbeitung}).

\subsubsection{Einordnung in klassische Probleme des Machine Learnings}

Neben mannigfaltigen Fehlerursachen (in \Cref{sec:datenerfassung-und-vorverarbeitung} erläutert) ist auch die kontextbedingte Komplexität der ermittelten Features denkbar hoch. Smartphones können zum Beispiel in verschiedenen Positionen (Hand, Hosentasche, Rucksack, ...) am Körper bzw. am/im Verkehrsmittel getragen werden. Weitere Ursachen für zwischen Messungen gleicher Aktivität abweichenden Daten sind unterschiedlich gebaute Verkehrsmittel (zum Beispiel Rennrad vs. Mountainbike), unterschiedliche anatomische Gegebenheiten und Bewegungsmuster der jeweiligen Person, sowie die Art des Untergrundes (zum Beispiel Schotterweg vs. asphaltierte Straße) oder die allgemeine Beschaffenheit des Geländes. Daher besteht ein weiteres zentrales Problem in der Realisation der Abbildung $VME(S): D^{n_S} \rightarrow L$. Eine einfache zustandslose Implementation dieser Abbildung wäre denkbar. Hierfür könnten beispielsweise Geschwindigkeitsbereiche auf $L$ abgebildet werden, wie $\varnothing v \geq 40\frac{km}{h} \rightarrow$ \enquote{Auto fahren} oder $\varnothing v \leq 5\frac{km}{h} \rightarrow$ \enquote{zu Fuß}. Bei der Betrachtung dieses Ansatzes lassen sich jedoch schnell Fehlerszenarien konstruieren, in welchen diese Abbildung fehlschlägt, beispielsweise eine dichte Verkehrslage mit ständigem Abbremsen und Beschleunigen, oder eine besonders schnelle Fahrradfahrt bergab. Bei der Behandlung dieser Fehlerszenarien werden bereits komplexere Überlegungen notwendig, welche möglicherweise neue Fehlerszenarien erzeugen. Je näher die zu erreichende Präzision des Systems an $100\%$ liegen soll, desto komplexer würde ein solches System vermutlich werden. Für solche Probleme, die auch durch hochkomplexe algorithmische Lösungen nicht hinreichend gelöst werden können, haben sich verschiedene Methoden aus dem Forschungsspektrum der künstlichen Intelligenz etabliert.

\begin{itemize}
\item Beim \textbf{Clustering} steht die Partitionierung einer Menge von Werten in geeignete Untergruppen im Vordergrund. Verfahren für das Clustering sind beispielsweise k-Means-Clustering \cite{likas_global_2003}, t-SNE \cite{van_der_maaten_laurens_and_hinton_geoffrey_visualizing_2008} oder PCA \cite{ringner_what_2008}.
\item Die \textbf{Regression} behandelt Probleme, deren Hauptgegenstand die Nachbildung eines Kausalzusammenhangs oder einer Korrelation in einer Menge von Werten ist. Im Zentrum steht hierbei das Ziel, die Abweichung der Regression von den Daten zu minimieren.
\item Bei der \textbf{Generation} sollen neue Daten anhand eines Vorbildes erzeugt werden. Ziel ist es, bei einer Eingabe von bestimmten Daten möglichst plausible weitere Daten zu erzeugen. Klassische Anwendungsbereiche umfassen KI-Systeme für die Chat-basierte Texterzeugung \cite{brown_language_2020} oder auch das Supersampling von Bildern \cite{burgess_rtx_2020, park_semantic_2019}.
\item Möglich ist auch die \textbf{Transformation}, wenn statt ähnlichen Daten (wie bei der Generation) andersartige Daten erzeugt werden sollen. Ein charakteristischer Anwendungsbereich ist die Realisation von \textit{Text-to-Speech}- oder \textit{Speech-to-Text}-Systemen \cite{wang_tacotron_2017}.
\item Im Rahmen der \textbf{Klassifikation} werden Eingabedaten in bestimmte Klassen eingeteilt, bei einer Maximierung der Anzahl korrekter Einordnungen.
\end{itemize}

\noindent Zusammenfassend handelt es sich hierbei um Verfahren, bei denen mathematische Modelle anhand vorliegender Daten angelernt werden (Induktion, Training), um schließlich zur Lösung des vorliegenden Problems verwendet zu werden (Deduktion, Inferenz). Daher werden diese Verfahren auch als \textit{Machine-Learning}-Methoden bezeichnet. Statt algorithmische Verfahren zu entwickeln, welche ein komplexes Problem (wie oben illustriert) möglicherweise nur begrenzt lösen können, ist also die Grundidee der Machine-Learning-Methoden die Repräsentation einer Abbildung von \textit{Input} zu \textit{Output} über ein angelerntes mathematisches Modell. \cite{matusek_anwendung_2019}, \cite{werner_kontinuierliche_2020} und \cite{stojanov_continuous_2020} konnten zeigen, dass solche Machine-Learning-Methoden zur Implementation der eingangs gezeigten Abbildung $VME(S): D^{n_S} \rightarrow L$ geeignet sind. Die Realisation der Abbildung von Input zu Output geschieht bei Machine-Learning-Modellen über ein mehr oder weniger komplexes System von mathematischen Parametern und Funktionen. Die Berechnung der Ausgabe (Inferenz) eines Machine-Learning-Modells lässt sich somit einfach realisieren, indem die Eingabewerte mithilfe der Parameter und Funktionen zum Ausgang überführt werden. Zentrale Probleme bestehen jedoch darin, eine geeignete Architektur für die Parameter und Funktionen zu wählen, sowie das so erstellte Modell zu trainieren. Diese zwei Probleme sollen in den folgenden Sektionen näher erläutert werden.

\subsection{Training von Machine-Learning-Modellen}

\todo{In dieser Sektion fehlen Einzelnachweise. Diese müssen noch ergänzt werden.}

In Abhängigkeit von der zu lösenden Aufgabe bieten sich verschiedene Lernverfahren an. Ziel ist es hierbei jedoch stets, das Machine-Learning-Modell so anhand der Eingangsdaten anzupassen, dass die Aufgabe innerhalb des Systems bestmöglich erfüllt werden kann. Typische Lernverfahren sind das \textit{Reinforcement Learning}, das \textit{Supervised Learning} und das \textit{Unsupervised Learning}. Diese Verfahren sollen nun vorgestellt werden.

\subsubsection{Reinforcement Learning}

Beim Reinforcement Learning werden auf Grundlage der Entscheidungen eines Machine-Learning-Modells \enquote{Belohnungen} und \enquote{Bestrafungen} definiert. Ziel ist die Maximierung eines Scores, der sich aus den gesammelten Belohnungen und Bestrafungen bildet. Dazu werden mehrere Modelle gebildet, welche sich in ihrer Parametrisierung leicht unterscheiden. Die Modelle werden getestet (meist durch Simulation) und anhand der definierten Belohnungen und Bestrafungen bewertet. Die besten Modelle werden ausgewählt und bieten die Grundlage für die nächste Generation. Hierbei wird angenommen, dass die Modelle analog zur Evolutionstheorie durch Zufall gewünschte Strategien durch Mutation hinzu erlernen. Für Klassifikationssysteme wie eine Verkehrsmittelerkennung ist dieses Lernverfahren jedoch eher unüblich.

\subsubsection{Unsupervised Learning}

Beim Unsupervised Learning wird ein Machine-Learning-Modell auf Daten trainiert, mit dem Ziel, innerhalb der Daten Muster zu erkennen bzw. diese zu reproduzieren. Das Training erfolgt ohne bekannte Labels für die Daten. Das Modell wird iterativ so angepasst, dass es eine abstrakte interne Repräsentation der Daten erlernt. Das Verfahren wird daher typischerweise für die Komprimierung von Daten (\textit{Sparse Coding}) oder die Segmentierung von Daten (durch Unterteilung der Daten in Gruppen mit ähnlichen Mustern) angewandt. Das Modell entwickelt hierbei durch Anpassung der Modellparameter Vermutungen über die Muster eines Datensatzes. Die Modellparameter werden anhand der beobachteten Daten iterativ so angepasst, dass das Modell die Daten möglichst präzise nachbildet oder segmentiert.

\subsubsection{Supervised Learning}

Ähnlich zum Unsupervised Learning wird beim Training von Machine-Learning-Modellen durch Supervised Learning eine zu minimierenden Kostenfunktion definiert. Eine typische Kostenfunktion ist der \textit{Root-Mean-Square Error}\footnote{\url{https://en.wikipedia.org/wiki/Root-mean-square_deviation} (Abgerufen am 6.5.2021)} (RMSE). Die Kostenfunktion soll beschreiben, wie stark das aktuelle Modell (anhand dessen Vorhersage) von dem Zielwert abweicht. Die Kostenfunktion wird minimiert, indem die mathematischen Parameter des Machine-Learning-Modells iterativ in Abhängigkeit von ihrem Beitrag zum Fehler angepasst werden.

\subsubsection{Erweiterte Lernverfahren}

Die oben erläuterten Lernverfahren bilden die drei wesentlichen Herangehensweisen beim Training von Machine-Learning-Modellen. Je nach der zu erfüllenden Aufgabe und den vorliegenden Daten ist eine geeignete Auswahl dieser drei Grundverfahren zum Training des Machine-Learning-Modells zu treffen. Für eine Verkehrsmittelerkennung bietet sich in erster Linie das Supervised Learning an, welches auch in \cite{matusek_anwendung_2019}, \cite{werner_kontinuierliche_2020} und \cite{stojanov_continuous_2020} zum Einsatz kommt. Hierfür müssen Datensätze im Voraus händisch mit den zu erkennenden Verkehrsmittel-Klassen ausgestattet werden. Für die zahlreichen bestehenden Datensätze aus dem Movebis-Projekt ist dies jedoch unter Umständen nicht möglich. Hierbei gilt grundsätzlich, dass die Menge der Daten beim Supervised Learning je nach Komplexität und Architektur des Machine-Learning-Modells ausreichend umfangreich sein muss. Um die benötigte Menge an gelabelten Daten zu reduzieren, kann ein erweitertes Lernverfahren eingesetzt werden, das \textit{Semi-Supervised Learning}. Bei diesem Lernverfahren wird zunächst Unsupervised Learning auf den nicht gelabelten Daten durchgeführt, um das Machine-Learning-Modell auf möglichen Mustern innerhalb der Daten vorzutrainieren. Das Machine-Learning-Modell soll so bereits ein abstraktes Grundverständnis der Eingaben erhalten, um anschließend auf den händisch erstellten Daten durch Supervised Learning auf die zu erkennenden Klassen \enquote{umtrainiert} zu werden. Das \enquote{Umtrainieren} von Machine-Learning-Modellen wird im breiteren Sinn auch als \textit{Transfer Learning} bezeichnet.

\subsubsection{Gradientenabstiegsverfahren}

Je nach Art des Machine-Learning-Modells und des Lernverfahrens existieren zahlreiche Möglichkeiten der iterativen Anpassung, um das Modell anzulernen. Von zentraler Bedeutung für Architekturen, wie sie in \cite{matusek_anwendung_2019}, \cite{werner_kontinuierliche_2020} und \cite{stojanov_continuous_2020} verwendet werden, sind Gradientenabstiegsverfahren zur Minimierung einer Kostenfunktion, vor allem beim Supervised Learning. Die Minimierung einer Kostenfunktion ist ein mathematisches Optimierungsproblem. Ziel ist es, die Parameter des Machine-Learning-Modells so anzupassen, dass die Kosten minimal werden. Bei einem kleinen Zustandsraum (aufgespannt durch die Modellparameter) ist dies durch einfache analytische Verfahren realisierbar. Bei vielen Tausenden oder Millionen von Modellparametern eignen sich wegen der zunehmenden Komplexität der Berechnung sogenannte Gradientenabstiegsverfahren. Diesen zugrundeliegend ist die Idee, dass sich bei der numerischen Variation der einzelnen Modellparameter die Kosten vergrößern oder verkleinern. Dieser lineare Zusammenhang kann als Gradient betrachtet werden. Beim Gradientenabstiegsverfahren werden die Modellparameter iterativ so angepasst, dass sich zur Verringerung der Kosten entlang des steilsten Gefälles am Gradient nach unten bewegt wird.

\todo{Symbolbild einfügen}

\noindent Die Intensität (Schrittweite auf dem Gradienten), mit der die Modellparameter bei jedem neuen Datenpunkt angepasst werden, wird als \textit{Learning Rate} bezeichnet. Die Learning Rate ist ein wichtiger Hyperparameter von Machine-Learning-Systemen. Ein zentrales Problem von Gradientenabstiegsverfahren ist, dass Gradienten neben einem zu erreichenden globalem Minimum auch zahlreiche lokale Minima besitzen können. Ist die Learning Rate zu klein, so lernt das Modell verhältnismäßig langsam und läuft gleichzeitig Gefahr, in ein lokales Minimum des Gradienten zu fallen (Underfitting). Ist die Learning Rate jedoch zu hoch, besteht das Risiko, dass aus Minima \enquote{herausgesprungen} wird und das Modell nicht mehr ausreichend auf neuen Daten generalisiert (Overfitting). Zentraler Teil des Trainings von Machine-Learning-Modellen ist es daher, das Lernverhalten zu beobachten und ggf. ein \textit{Hyperparameter Tuning} durchzuführen. Diese Betrachtung ist jedoch vereinfacht und dient nur zur Veranschaulichung des Funktionsprinzips. Bei erweiterten Optimierungsverfahren wie dem \textit{ADAM-Optimizer} wird die Learning Rate beispielsweise pro Modellparameter individuell angepasst.

\subsection{Modellvalidierung}

Da Machine-Learning-Modelle durch das Erlernen einer abstrakten internen Repräsentation der Abbildung von Eingaben auf Ausgaben trainiert werden, ist die Erklärung und Validierung von deren Funktion ein nichttriviales Problem, mit dem sich unter anderem der Forschungsbereich \textit{Explainable AI} (XAI) auseinandersetzt. Durch direkte Beobachtung der Modellparameter ist es in Abhängigkeit der Modellkomplexität schwer, zu determinieren, ob ein Machine-Learning-Modell erwünschte oder unerwünschte Muster lernt. Um ein mögliches Overfitting oder Underfitting zu erkennen, muss das Modell während des Trainings kontinuierlich mithilfe von konkreten Metriken validiert werden. Lassen diese Metriken erkennen, dass das Modell ein solches unerwünschtes Verhalten zeigt, können die relevanten Hyperparameter angepasst werden. Zeigt sich beim Training, dass dessen Fortführung zu einem unerwünschten Rückschritt führen würde, kann das Training vorzeitig beendet werden. Dies ist auch bekannt unter dem Fachbegriff \textit{Early Stopping}.

% \subsubsection{Kreuzvalidierung}

% \subsubsection{Metriken}

% \todo{FN, FP, Recall, Precision, F1-Score, Activation Matrix, Confusion Matrix}

% \subsection{Modellarchitekturen}

% \subsubsection{Decision Trees und Random Forests}

% \subsubsection{Support Vector Machines}

% \subsubsection{Feed-Forward Neural Networks}

% \subsubsectino{Recurrent Neural Networks}

% \subsubsection{Convolutional Neural Networks}


\section{Portierung}\label{sec:portierung}

In den vorigen Abschnitten wurde diskutiert, wie Daten vorverarbeitet werden können, um anschließend durch Machine-Learning-Modelle klassifiziert zu werden. In \cite{matusek_anwendung_2019}, \cite{werner_kontinuierliche_2020} und \cite{stojanov_continuous_2020} nutzen hierfür künstliche neuronale Netzwerke mit verschiedenen Architekturen (FFN, \acrshort{rnn}, \acrshort{cnn}). Dieser Abschnitt soll diskutieren, wie die Konzepte nun für den Einsatz auf Smartphones portiert werden können und welche Faktoren hierbei zu beachten sind.

\subsection{Deployment-Modelle im Edge Computing}

Für einen Einsatz auf Smartphones müssen Machine-Learning-Modelle nicht zwingend auch auf diesem über eine App mitgeliefert oder installiert (engl. \textit{deploy}) werden. \cite[S. 13]{ota_deep_2017} separieren beispielsweise in zwei Deployment-Modelle.

\begin{itemize}
\item \textit{Client-Server-Deployment}, wobei die Inferenz des Machine-Learning-Modells auf dem Server-Backend geschieht.
\item \textit{Client-Only-Deployment}, wobei die Inferenz vom Smartphone selbst übernommen wird, das Machine-Learning-Modell also mit in der jeweiligen App integriert ist.
\end{itemize}

\cite{matusek_anwendung_2019}, \cite{werner_kontinuierliche_2020} und \cite{stojanov_continuous_2020}, welche mithilfe von künstlichen neuronalen Netzwerken das Filtern der STADTRADELN-\allowbreak Daten im Movebis-Projekt ermöglichen, gliedern sich analog zur obigen Taxonomie in das Client-Server-Deployment ein, wobei jedoch nur eine zeitlich entkoppelte unidirektionale Kommunikation stattfindet. Die Smartphone-App erhält kein direktes Feedback zum Ergebnis der serverseitigen Klassifikation und besitzt so auch keine Möglichkeit zur Adaption auf wichtige Ereignisse wie einen Wechsel des Verkehrsmittels. Vor diesem Hintergrund könnte es allgemeiner gefasst also vorteilhaft sein, die Inferenz vom Server so nah wie möglich an den Client (Smartphone) zu verschieben, unter anderem auch um die Netzwerklast zu verringern (durch Abschalten der Aufzeichnung infolge eines Verkehrsmittelwechsels) und die sonst serverseitig für Inferenz und Speicherung der Daten benötigten Ressourcen auf die Nutzer des Systems aufzuteilen. Dies konstituiert die zentralen Motivationen hinter \textit{Edge Computing}, bei dem Edge-Geräte (Smartphones, Router, IoT-Geräte im Allgemeinen) gezielt Aufwände innerhalb eines Netzwerkes übernehmen und untereinander orchestrieren. Als Teilbereich des Edge Computing beschäftigt sich der Forschungsbereich \textit{Edge AI} vor allem damit, wie Edge-Geräte durch den Einsatz von künstlicher Intelligenz effizienter und proaktiv auf dynamische Veränderungen des Kontexts reagieren können. Die Portierung eines Machine-Learning-Systems auf Smartphones, wie es im Movebis-Projekt noch serverseitig zum Einsatz kommt, kann somit als Spezialfall dieses Forschungsbereichs gewertet werden. Als Edge-Geräte sind Smartphones vor allem durch die auf dem Gerät verfügbaren Ressourcen limitiert, als zentrale Aspekte stehen sich hierdurch die Performanz und die Effizienz des Machine-Learning-Modells gegenüber \cite{zou_edge_2019, ma_survey_2019, deng_model_2020}.

\begin{figure}[h]
\includegraphics[width=0.5\linewidth, bb=0 0 367 304]{six-level-rating-ei.pdf}
\caption[6-stufiger taxonomischer Überblick über Lern- und Inferenzverfahren im Edge Computing]{6-stufiger taxonomischer Überblick über Lern- und Inferenzverfahren im Edge Computing nach \cite{zhou_edge_2019}.}\label{fig:six-level-rating-ei}
\end{figure}

\Cref{fig:six-level-rating-ei} zeigt einen 6-stufigen taxonomischen Überblick über die verschiedene Möglichkeiten für das Training von Machine-Learning-Modellen im Edge Computing in Kooperation mit Servern nach \cite{zhou_edge_2019}. Wird das Machine-Learning-Modell in der Cloud ausgeführt und trainiert, stehen dem Modell im Umkehrschluss auch typischerweise deutlich mehr Ressourcen zur Verfügung. In diesem Fall können die Server mit geeigneter Hardware (z.B. hochparallelisierbare Grafikkarten\footnote{\url{https://www.tensorflow.org/install/gpu} (Abgerufen am 19.5.2021).}, fester Stromanschluss, hohe Bandbreite) für das Training und die Inferenz ausgestattet werden. Je näher an die taxonomische Stufe \textit{All-In-Edge} herangetreten wird, desto weniger dieser Ressourcen stehen direkt zur Verfügung. Es werden spezielle kollaborative Lernkonzepte wie beispielsweise das in \cite{wang_-edge_2019} gezeigte \textit{Federated Learning} notwendig, mithilfe derer die Modellparameter über das Netzwerk zwischen Edge-Geräten und Cloud-Servern ausgetauscht werden können, um die Effizienz zu verbessern und die individuelle Last auf dem Edge-Gerät zu verringern.

\subsection{Zentrale Probleme der Portierung}

Das Smartphone-Ökosystem stellt einige Herausforderungen an die Portierung von Machine-Learning-Modellen, unter anderem auch bedingt durch die große Variation an Smartphones und den dazugehörigen Hardwarevoraussetzungen und Betriebssystemen. Die wichtigsten Problemfaktoren sollen in diesem Abschnitt näher erläutert werden, um anschließend konkrete Lösungsmethoden vorzustellen.

\subsubsection{Hardware-Limitationen}

Als Edge-Geräte sind Smartphones hardwaretechnisch in erster Linie durch die zur Verfügung stehende Netzwerkbandbreite, den Energieverbrauch und den verfügbaren Speicher (Arbeitsspeicher, Persistenter Speicher) limitiert. Die Netzwerkbandbreite ist insbesondere dann für das Machine-Learning-System von Bedeutung, wenn dessen Lernparameter über das Netzwerk zwischen den Edge-Geräten ausgetauscht werden müssen (wie beim Federated Learning) oder das Modell in der Cloud ausgeführt wird. Für die Ausführung eines Machine-Learning-Modells direkt auf dem Smartphone ist die Netzwerkbandbreite nur von sekundärer Relevanz. Primär ist hier der Energieverbrauch und der Speicherverbrauch des Modells von Bedeutung. Künstliche neuronale Netzwerke erreichen je nach Architektur schnell hunderte Millionen von Lernparametern und Speicherdimensionen von hunderten Megabytes \cite[S. 313, 314]{sosnovshchenko_machine_2018}, in für Smartphones unpraktikablen Extremfällen wie \textit{GPT-3} sind es 175 Mrd. Lernparameter und 700 Gigabytes\footnote{\url{https://lambdalabs.com/blog/demystifying-gpt-3/} (Abgerufen am 21.5.2021)}. Wird ein Machine-Learning-Modell wie ein künstliches neuronales Netzwerk mit 100 Megabytes in eine App integriert, so erhöht dies auch wiederum die Größe der App und Nutzer sind möglicherweise nicht mehr bereit oder besitzen ausreichend Speicher, um sich diese zu installieren. Die Nutzererfahrung wird also verschlechtert, dies ist mit dem qualitativen Zugewinn durch Einsatz des Machine-Learning-Modells abzuwägen. Bestenfalls wird jedoch die Größe des Modells weitestgehend bei der Entwicklung reduziert. Ist die App heruntergeladen, kann die Ausführung von künstlichen neuronalen Netzwerken effizient parallelisiert werden, unter der Verwendung von speziellen Vektor-Koprozessoren oder Grafikchips. Dennoch stellt die Inferenz und insbesondere das Training einen hohen Rechenaufwand dar, der den Akku innerhalb kürzester Zeit verbrauchen kann \cite[S. 311]{sosnovshchenko_machine_2018}. Darüber hinaus stehen möglicherweise nicht auf allen Geräten dedizierte Hardwarekomponenten zur Verfügung, unter Umständen wird die CPU genutzt. Hierdurch steigt die Dauer, sowie die Energie- und Speicherintensität des Betriebs durch die geringe Parallelisierung. Die mathematische Optimierung, welche zum Training des Modells notwendig ist, stellt nochmals einen höheren Rechenaufwand dar. Daher ist es auch unüblich, größere Modelle auf Smartphones zu trainieren.

\subsubsection{Software-Limitationen}

In den vorigen Abschnitten wurde gezeigt, dass Machine-Learning-Modelle in der Cloud ausgeführt und trainiert werden können, oder dies auch vollständig auf dem Smartphone durchgeführt werden kann, mit den entsprechenden Hardware-Limitationen. Auch die Software-Limitationen hängen direkt von der Wahl des Deployment-Modells ab.

Die Portierung eines Machine-Learning-Modells ist an einen technologischen Flaschenhals gebunden, der durch die Heterogenität der serverseitigen Implementationen und deren clientseitiges Pendant entsteht. Grundsätzlich besteht zu Beginn der Portierung als zentraler Teil des Lösungskonzepts daher nach aktuellem Stand immer die Frage, ob das Modell mithilfe von Frameworks implementiert und über deren Funktionalitäten portiert werden soll, oder selbst mit hardwarenahen Schnittstellen reimplementiert wird. Die Nutzung von Frameworks ist typischerweise vom Standpunkt des Entwicklers die effizienteste Lösung, Modelle lassen sich mit Frameworks in wenigen Codezeilen implementieren. Dies ist jedoch auch mit Nachteilen verbunden. Es besteht das Risiko, dass durch die hohe Abstraktion der Frameworks bestimmte Details übersehen werden, die bei einer manuellen Implementation aktiv codiert werden müssten, wie beispielsweise Aktivierungsfunktionen von Neuronen. Je nach Framework sind die resultierenden Modelle darüberhinaus auf dem Smartphone unter Umständen nicht mehr trainierbar oder nur als \textit{Black Box}\footnote{Black Box bezeichnet in diesem Fall, dass das Machine-Learning-Modell bei der Framework-gestützten Portierung fest definierte Schnittstellen erhält und die internen Funktionsweisen nach außen versteckt werden.} verfügbar. Diese Nachteile sollten bei der Auswahl der Technologien und der Portierung berücksichtigt werden.

Neben dieser Problematik müssen App-Entwickler auch die Gegebenheiten des Betriebssystems und der App-Nutzung mit einbeziehen. Während Cloud-basierte Machine-Learning-Modelle dediziert und kontinuierlich für die vorgesehene Aufgabe ausgeführt werden können, ist die Ausführung von Machine-Learning-Modellen auf Smartphones an häufige Unterbrechungen durch den Nutzer (z.B. Wechsel in andere App) gebunden. Gleichzeitig kann die App permanent in den Hintergrund gelangen, entweder durch Wechsel der App oder durch Aktivierung des Stand-By-Modus. Die Verfügbarkeit der Rechenressourcen kann also zu jedem Zeitpunkt entzogen werden, langfristige Operationen wie das Training sind dann unter Umständen nicht mehr möglich. In diesem Fall besteht grundsätzlich die Möglichkeit, die notwendigen Operationen in den Hintergrund zu verschieben. Um eine Kompetetivität mit anderen Apps zu verhindern, sollte dies bei rechenintensitiven Operationen wie dem Training vermieden oder mit den entsprechenden Systemschnittstellen in Zeitbereiche mit geringer Auslastung verschoben werden\footnote{\url{https://developer.apple.com/documentation/backgroundtasks/choosing_background_strategies_for_your_app} (Abgerufen am 21.5.2021)}. Versucht eine App, sich dennoch selbst durch unsensibles und ressourcenverbrauchendes Hintergrundverhalten zu übervorteilen, so könnte sie bei der App-Prüfung (vor Veröffentlichung im jeweiligen App-Store) zurückgewiesen oder vom Betriebssystem aktiv benachteiligt werden. Je nach Betriebssystem und App-Store gelten hierfür unterschiedliche Entwicklerrichtlinien, wichtig ist bei der Entwicklung also auch die Beachtung der \textit{Compliance}.

\subsection{Modelloptimierung}

Um die Rechen-, Speicher- und Energieintensität von Machine-Learning-Modellen zu optimieren, können verschiedene Konzept angewandt werden. Da es sich hierbei um ein aktuelles Forschungsthema handelt und die Konzepte in verschiedenste Richtungen gehen, bietet sich ein Blick in spezifische Metaanalysen an. \cite{deng_model_2020} und \cite{choudhary_comprehensive_2020} vermitteln einen detaillierten Überblick über den Stand der Forschung in der Modelloptimierung, \cite[S. 315ff]{sosnovshchenko_machine_2018} gibt einen praxisorientierten Überblick. \cite{nan_deep_2019} und \cite{dai_toward_2021} analysieren verschiedene Optimierungsmethoden für mobile Plattformen unter anderem bezüglich ihrer konkreten Auswirkungen auf die Modellpräzision, die Inferenzzeit, die CPU-Last und die damit verbundenen thermischen Auswirkungen. Die zentralen Ideen der Modelloptimierung sollen in den nachfolgenden Abschnitten im Überblick vorgestellt werden.

\subsubsection{Limitierung des Anwendungsfalles}

Eine für alle Machine-Learning-Systeme anwendbare Methode zur Modelloptimierung ist durch die Limitierung des Anwendungsfalles realisierbar. Je mehr Aufgaben ein Modell erfüllen soll, beispielsweise durch die Unterscheidung von zahlreichen Mustern, desto komplexer wird normalerweise auch die interne Repräsentation. Neuronale Netzwerke müssen also zum Beispiel mehr Neuronen und Schichten erhalten, im Umkehrschluss wird das Modell größer und der Rechen-, Energie- und Speicheraufwand steigt. Außerdem steigt die Trainingsdauer und die Menge an benötigten Trainingsdaten. Es bietet sich also an, den Anwendungsfall des Modells soweit möglich zu reduzieren, beispielsweise könnte bei einer Verkehrsmittelklassifikation die Anzahl der Klassen auf \enquote{Fahrrad fahren} und \enquote{Nicht Fahrrad fahren} limitiert werden. Dasselbe Prinzip kann auf die Entropie der Trainingsdaten angewandt werden. Sind die Trainingsdaten variabler als es der Anwendungsfall erfordert, bedingt dies auch wieder eine größere interne Repräsentation des Modells. Bei einer Klassifikation von handgeschriebenen Buchstaben kann beispielsweise der fotografierte Winkel der Buchstaben beim Training variiert werden, um neue Trainingsdaten zu erzeugen (\textit{Data Augmentation}), jedoch nicht mehr als $\pm 30^\circ$, wenn das inferierende Smartphone später erwartungsgemäß später nicht mehr als $30^\circ$ geneigt wird \cites{otavio_good_how_2015}. Zusammenfassend kann also allein durch die gezielte Konfiguration der zu unterscheidenden Klassen sowie der Trainingsdaten die für eine hinreichend gute Klassifikation benötigte Modellgröße reduziert werden \cite[S. 319]{sosnovshchenko_machine_2018}.

\subsubsection{Architekturelle Restrukturierung des Modells}

Die Reduktion der Modellgröße durch Elimination von internen Parametern muss nicht wahllos geschehen. Auch hierfür können gezielte Strategien angewandt werden \cite[S. 317ff]{sosnovshchenko_machine_2018}. Speziell bei \acrshort{cnn}s bietet sich die Reduktion von einzelnen Convolution-Schichten an. Wie \cite{iandola_squeezenet_2016} zeigen, können allein durch die Reduktion einer im Ausgaben-nahen Teil des Netzwerkes liegenden $3\times 3$-Convolution-Schicht zu einer $1\times 1$-Convolution-Schicht die gesamten transitiven Parameter eines \acrshort{cnn}s um das Neunfache reduziert werden (\textit{SqueezeNet}), bei einem geringfügigen Informationsverlust in Relation zum nicht verkleinerten Modell. Eine weitere Idee bei der Optimierung der Convolution-Schichten wurde erstmals in \cite{howard_mobilenets_2017} (\textit{MobileNet}) vorgestellt. Im Forschungsbereich der Computergrafik und \textit{Computer Vision} ist es ein weit verbreitetes Verfahren, separierbare Filter wie den Gaußschen Weichzeichner in zwei oder mehrere Teilfilter zu zerlegen, um die absolut benötigte Anzahl an Rechenoperationen zu reduzieren. Beim Weichzeichner resultiert so ein horizontales und ein vertikales Filter. Die Convolution-Operationen eines \acrshort{cnn}s können analog hierzu, da sie lediglich spezielle Matrixoperationen sind, auch als mathematisches Filter betrachtet und deren Berechnung durch tiefenweise Separation optimiert werden. Bei $3\times 3$-Convolution-Schichten kann die Berechnung so neunfach effizenter (und schneller) ausgeführt werden\footnote{\url{https://machinethink.net/blog/googles-mobile-net-architecture-on-iphone/} (Abgerufen am 22.5.2021)}. Weitere Strategien für die architekturelle Optimierung von \acrshort{cnn}s sind die Verwendung von speziellen \textit{Shuffle}-Schichten \cite{zhang_shufflenet_2018} (\textit{ShuffleNet}) oder die gezielte Gruppierung von Convolution-Schichten in \textit{Grouped Convolutions} \cite{huang_condensenet_2018} (\textit{CondenseNet}). Auch für \acrshort{rnn}s bestehen Ideen zur architekturellen Optimierung, beispielsweise durch eine effizientere Repräsentation der Matrix-Vektor-Multiplikationen \cite{wang_accelerating_2017} oder der Ersetzung der typischen \acrshort{lstm}-Neuronen durch kompaktifizierbarere Alternativen \cite{kusupati_fastgrnn_2019, luo_neural_2019}. Neben \acrshort{cnn}s und \acrshort{rnn}s lassen sich auch FFNs hinsichtlich ihrer Architektur optimieren. Beispielsweise lassen sich spezielle \textit{Multiplexing}-Verbindungen in das Netzwerk einfügen, um (unter leichter Verschlechterung des Zeitverhaltens) bis zu $50\%$ der sonst benötigten Hidden Layers und den dazugehörigen Neuronen einzusparen \cite{khalil_efficient_2018}.

\subsubsection{Verlustfreie Kompressionsverfahren}

Neben einer architekturellen Restrukturierung von Machine-Learning-Modellen wie neuronalen Netzwerken ist es auch möglich, diese für den Netzwerktransport und die persistente Speicherung verlustfrei zu komprimieren. Diese Technologie ist altbewährt, verlustfreie Kompressionsprogramme wie \texttt{gzip}\footnote{\url{https://www.gzip.org/} (Abgerufen am 22.5.2021)} existieren bereits seit mehreren Jahrzehnten und basieren häufig auf der theoretischen Grundlage der \textit{Huffman-Kodierung}\footnote{\url{https://de.wikipedia.org/wiki/Huffman-Kodierung} (Abgerufen am 22.5.2021)}. Die Machine-Learning-Modelle müssen dabei nicht zwingend mit plattformspezifischen Kompressionsverfahren verkleinert werden, sowohl iOS\footnote{\url{https://developer.apple.com/documentation/compression/compression_zlib} (Abgerufen am 22.5.2021)} als auch Android\footnote{\url{https://developer.android.com/reference/java/util/zip/Deflater} (Abgerufen am 22.5.2021)} unterstützen zum Beispiel das weit verbreitete \texttt{zlib}-Verfahren \cite[S. 315]{sosnovshchenko_machine_2018}.

\subsubsection{Verlustbehaftete Kompressionsverfahren}

Während bei der verlustfreien Kompression alle Parameter des Machine-Learning-Modells erhalten bleiben, werden bei der verlustbehafteten Kompression gezielt Parameter verändert oder entfernt. Ziel ist es hierbei, so viele Parameter wie möglich zu reduzieren, bei einer möglichst gleichbleibenden (oder in manchen Fällen sogar verbesserten) Präzision.

\paragraph{Quantisierung:} Die Parameter eines herkömmlichen Machine-Learning-Modells liegen typischerweise in Fließkommazahlen nach dem IEEE 754\footnote{\url{https://de.wikipedia.org/wiki/IEEE_754} (Abgerufen am 23.5.2021)} $32 bit$-Format vor. Diese können zur Reduktion der Modellgröße in Ganzzahlen nach dem $8 bit$ Integer-Format überführt werden. Rechnerisch lässt sich hierüber die Größe des Modells auf bis zu ein Viertel der Ausgangsgröße reduzieren. Gleichzeitig müssen die Fließkommazahlen über eine skalare Multiplikation geeignet in den Integer-Wertebereich $[-128, 127]$ (signed) bzw. $[0, 255]$ (unsigned) überführt werden\footnote{\url{https://de.wikipedia.org/wiki/Integer_(Datentyp)} (Abgerufen am 23.5.2021)}. Hierzu können eine Min-Max-Skalierung oder verschiedene andere Skalierungsmethoden dienen. Da sich die Skalierung bei Änderung der Modellparameter mit ändert, bietet sich die Quantisierung in der Regel für eine Komprimierung \textit{nach} dem Training an. Es bestehen aber auch Möglichkeiten, bereits beim Training eine laufende Quantisierung durchzuführen. Trotz des signifikanten Verlustes der durch die binäre Repräsentation darstellbaren Werten verringert eine Quantisierung die Präzision eines Modells normalerweise nur geringfügig, daher eignet sie sich sehr gut für die verlustbehaftete Komprimierung \cite[S. 14ff]{liang_pruning_2021}.

\paragraph{Pruning:} Nicht alle Parameter eines Machine-Learning-Modells sind für die interne Wissensrepräsentation gleich wichtig. Es besteht die Möglichkeit, dass bestimmte Parameter nur einen geringen Einfluss auf die Ausgabe des Modells haben. Bei künstlichen neuronalen Netzwerken können so beispielsweise gezielt einzelne Neuronen, Gruppen von Neuronen oder ganze Schichten entfernt werden, wenn sie als unwichtig identifiziert werden können. Letzteres ist jedoch ein nichttriviales Problem, welches eine differenzierte Herangehensweise erfordert. Es ist möglich, über die zufällige Elimination von Neuronen und Verbindungen zu testen, ob diese die gesamte Präzision signifikant reduziert (\textit{Brute-Force}-Methode). Durch Heuristiken lässt sich dieses Verfahren jedoch weiter verbessern. Bei einer sehr einfachen Variante des \textit{Magnitude-based Pruning} werden beispielsweise Parameter innerhalb eines Machine-Learning-Modells entfernt, welche sehr kleine Werte nahe $0$ annehmen. Die Annahme hierhinter ist, dass Parameter mit sehr kleinen Werten in der Regel weniger wichtig für die Ausgabe des Modells sind, als Parameter mit größeren Werten. Ähnliche Methoden wie \textit{Optimal Brain Damage} \cite{lecun_optimal_1990} und \textit{Optimal Brain Surgeon} \cite{hassibi_second_1993} nutzen den Gradienten der Kostenfunktion für die Identifikation einzelner zu eliminierender Neuronen. Bei der Elimination von Neuronen ist auch die Regularisierung von Bedeutung, durch Regularisierungsmethoden (\Cref{sec:regularisierung}) wie Dropout oder $l_1$- und $l_2$-Regularisierung kann diese beschleunigt werden \cites[S. 7ff]{liang_pruning_2021}{chang_prune_2019}.

\paragraph{Knowledge Distillation:} Methoden wie Quantisierung und Pruning können auf einem bestehenden Modell direkt angewandt werden, um dieses durch die Elimination von potenziell unwichtigen Parametern zu komprimieren. Beim Verfahren \textit{Knowledge Distillation} wird ein bestehendes (trainiertes) Modell genutzt, um dessen Verhaltensweise durch ein kleineres Modell nachzustellen. Ziel ist es, eine effizientere Wissensrepräsentation durch das kleinere Modell zu erstellen, welche weniger Parameter bei einer annähernd gleichen Präzision besitzt.

Die diskutierten Methoden sind in dieser Form für die meisten Machine-Learning-Modelle flexibel anwendbar und dadurch auch weit verbreitet \cite{nan_deep_2019, choudhary_comprehensive_2020}. Die aufgezeigten Grundlagen erfassen jedoch nicht das gesamte Spektrum an potenziellen Ansätzen. Weitere Ansätze sind beispielsweise:

\begin{itemize}
\item Optimierung des Machine-Learning-Modells für die Inferenz durch die Elimination von Komponenten, die lediglich für das Training benötigt werden.
\item Approximation der Modellparameter durch \textit{Low Rank Tensor Approximation} und die daraus resultierende Verkleinerung der Matrixoperationen\footnote{\url{https://en.wikipedia.org/wiki/Low-rank_approximation} (Abgerufen am 23.5.2021)}.
\item Fusion von bestimmten Modell-Operationen (\textit{Layer/Tensor-Fusion})\footnote{\url{https://www.tensorflow.org/lite/convert/operation_fusion} (Abgerufen am 23.5.2021)} zur weiteren Optimierung und Komprimierung der Berechnungen.
\item Verschiedene hardwareorientierte Ansätze wie die Einführung einer dynamischen Speicherverwaltung, um größere Machine-Learning-Modelle effizient und parallelisiert auf Grafikeinheiten und dedizierten Koprozessoren ausführen zu können \cite{sb_dynamic_2019}.
\end{itemize}

Im Rahmen der Portierung bietet es sich in Anbetracht der verschiedensten Methoden an, eine konkrete Kompressionsstrategie zu entwickeln, welche einen oder mehrere dieser Schritte inkludiert. Dabei kann auch eine Selektion spezifischerer Varianten und Derivate der gezeigten Optimierungsmethoden gewählt werden.

\subsection{Profiling}

Die Ergebnisqualität eines Machine-Learning-Modells lässt sich anhand der in \Cref{sec:modellvalidierung} diskutierten Metriken beurteilen. Die Messung von Metriken wie dem $F1$-Score oder die Erstellung einer Konfusionsmatrix ist nicht zwingend direkt auf dem Smartphone notwendig, sofern das Modell in seiner Funktionsweise nicht zwischen Test und Deployment abgewandelt wird. Auch Parameter wie die statische Größe des Modells auf dem Festspeicher sind ohne ein Test-Smartphone messbar. Für andere Ressourcenparameter wiederum ist ein \textit{Profiling} des Modells direkt auf dem Test-Smartphone notwendig. Insbesondere sind dies der dynamische Speicherverbrauch im Arbeitsspeicher während der Ausführung, die Auslastung der CPU-Kerne und eventuell genutzter Koprozessoren bzw. GPU-Kerne und unmittelbar daraus entstehende Faktoren wie der Energieverbrauch und die Messung der Hardwaretemperatur. Um dies zu realisieren, können die in eine App eingebetteten Machine-Learning-Modelle in Kooperation mit entsprechenden Entwicklertools wie \textit{XCode}, \textit{Instruments} (iOS)\footnote{\url{https://developer.apple.com/library/archive/documentation/Performance/Conceptual/EnergyGuide-iOS/MonitorEnergyWithInstruments.html} (Abgerufen am 23.5.2021)} oder dem \textit{Android Profiler}\footnote{\url{https://developer.android.com/studio/profile/android-profiler} (Abgerufen am 23.5.2021)} ausgeführt und die genannten Ressourcenparameter ausgelesen werden. Insbesondere mit Hinblick auf den Vergleich zwischen mehreren Modellen ist es hierbei jedoch wichtig, dass die Testbedingungen möglichst konstant sind, dies muss bei der Evaluation des Tradeoffs zwischen Ressourcenbedarf und Ergebnisqualität später unbedingt einbezogen werden. Beispielsweise sollte eine konstante Temperatur und Energieversorgung sichergestellt werden, um eine zu starke Variation der Prozessorgeschwindigkeit im Smartphone-internen Regelsystem zu vermeiden. Über die diskutierten Parameter hinaus können noch zwei weitere Parameter evaluiert werden, genauer die Dauer und die Latenz der Inferenz. Dies kann über zeitlich messbare Unit-Tests\footnote{\url{https://developer.apple.com/documentation/xctest/xctestcase/1496290-measureblock} (Abgerufen am 23.5.2021)} des Machine-Learning-Modells geschehen, wobei diese Unit-Tests auch wieder unter möglichst konstanten Testbedingungen auf dem Smartphone ausgeführt werden sollten, um ein möglichst aussagekräftiges Ergebnis zu erhalten.


\section{Zusammenfassung}

Die zur Aufzeichnung von Smartphone-Aktivitätsdaten verwendeten Sensorsysteme sind durch ihre eigenen Fehlermuster geprägt. Bei der Positionsapproximation über \acrshort{gnss} ist zusätzlich der sehr hohe Energieverbrauch problematisch. Durch die kontextsensitive Konfiguration der Positionsapproximation kann auf energiesparendere Triangulationsmechanismen über Funknetze ausgewichen werden. Um, die inhärenten Fehlermuster zu mitigieren, können verschiedene mathematische Verfahren zur Reduktion von Rauschen, Rekonstruktion im Weltkoordinatensystem, Transformation der Zeitliniendaten in Frequenzdaten, Skalierung und Approximation eingesetzt werden. Für ein Alignment von parallel aufgezeichneten Aktivitätsdaten anhand eines festen Abtastrasters können spezielle Interpolationsverfahren genutzt werden.

Zur Klassifikation der vorverarbeiteten Daten bieten sich wegen der inhärenten Komplexität datenorientierte Machine-Learning-Modelle an, denn das Problem der Verkehrsmittelklassifikation stellt ein klassisches Problem der künstlichen Intelligenz dar. Dazu können die Daten zunächst segmentiert und dann in Features überführt werden, welche als Eingabe für das Machine-Learning-Modell dienen. Hierbei können Shallow Features und Non-Shallow Features berechnet und kombiniert werden. Das Supervised Learning ist ein grundlegendes Lernverfahren, welches zum Training eines Machine-Learning-Klassifikators auf gelabelten Daten verwendet werden kann. Während das Reinforcement Learning sich vermehrt auf das evolutionäre Training von \enquote{Agenten} fokussiert, kann das Unsupervised Learning mit dem Supervised Learning in erweiterten Lernverfahren kombiniert werden, um die Datengrundlage weiter zu erschließen. Der Lernprozess ist insbesondere beim Supervised Learning durch die Minimierung einer Kostenfunktion definiert. Das hieraus entstehende mathematische Optimierungsproblem kann durch Gradientenabstiegsverfahren mit diskreten Schritten und dem dazugehörigen Hyperparameter der Learning Rate näherungsweise gelöst werden. Wesentliche Probleme dessen sind das Overfitting und das Underfitting. Um diese Probleme zu vermeiden, kann das Machine-Learning-Modell während des Trainings validiert und regularisiert werden. Die konkrete Architektur des Modells ist hierbei flexibel ~- es bieten sich traditionelle Ansätze wie Decision Trees oder \acrshort{knn} an. Die in \cite{matusek_anwendung_2019}, \cite{werner_kontinuierliche_2020} und \cite{stojanov_continuous_2020} verwendeten Modelle sind Architekturen des Deep Learning. Hierbei kommen künstliche neuronale Netze mit unterschiedlichen Architekturen zum Einsatz. Während \acrshort{ffn}s bereits für grundlegende Klassifikationsprobleme geeignet sind, bieten \acrshort{cnn}s und \acrshort{rnn}s erweiterte Architekturen und charakteristische Anwendungsgebiete.

Mithilfe vorverarbeiteter Daten (Eingabe) und der gezeigten Machine-Learning-Modelle kann schließlich eine Verkehrsmittelklassifikation (Ausgabe) implementiert und als Client-Server-Applikation bereitgestellt werden. Gleichzeitig besteht jedoch auch die Möglichkeit, die Inferenz und das Training entweder komplett auf dem Smartphone oder durch Kollaboration mit anderen Edge-Geräten durchzuführen. Wird, wie im Kontext dieser Arbeit, die Datenverarbeitungspipeline inklusive Machine-Learning-Modell ohne In-Edge-Kollaboration auf das Smartphone portiert, müssen zunächst hardware- und softwaretechnischen Herausforderungen bewältigt werden. Die Modelloptimierung stellt somit ein wichtiges Werkzeug zur Portierung von Machine-Learning-Ansätzen dar. Neben einer Reduktion des Anwendungsfalls können Machine-Learning-Modelle zur Optimierung architekturell restrukturiert werden. Auch können verlustfreie und spezielle verlustbehaftete Kompressionsverfahren genutzt werden, um die Größe des Modells weiter zu reduzieren und schließlich ein Modell zu erhalten, welches eine möglichst destillierte Wissensrepräsentation ermöglicht. Um schließlich neben einer Modellvalidierung auch die Eignung für die Ausführung auf Smartphones zu überprüfen, kann das System dem Profiling unterzogen werden, bei dem wichtige Hardware- und Softwaremetriken aufgezeichnet werden können.


  \newpage

  % use lowercased roman page numbers for the appendix and the bibliography
  \pagenumbering{roman}

  \printbibliography[heading=bibintoc]\label{sec:bibliography}%

  % \begin{appendices}
  %   \tocless\chapter{Movebis-Modellarchitekturen}\label{appendix:a}

\begin{figure}[h]
  \centering
  \begin{subfigure}[t]{.25\textwidth}
    \centering
    \includegraphics[width=\linewidth, bb=0 0 368 491]{matusek-model.pdf}
    \caption{FFN aus \cite[S. 80]{matusek_anwendung_2019}.}
    \label{fig:matusek-ffn}
  \end{subfigure}%
  \begin{subfigure}[t]{.25\textwidth}
    \centering
    \includegraphics[width=\linewidth, bb=0 0 443 1139]{stojanov-model.pdf}
    \caption{\acrshort{cnn} aus \cite[S. 76]{stojanov_continuous_2020}.}
    \label{fig:stojanov-cnn}
  \end{subfigure}
  \begin{subfigure}[t]{.49\textwidth}
    \centering
    \includegraphics[width=\linewidth, bb=0 0 957 1031]{werner-model.pdf}
    \caption{\acrshort{rnn} aus \cite[S. 24]{werner_kontinuierliche_2020}.}
    \label{fig:werner-rnn}
  \end{subfigure}
  \caption[Konkrete Konfigurationen der Machine-Learning-Architekturen aus den Movebis-Ansätzen zur Analyse der Modellgrößen.]{Konkrete Konfigurationen der Machine-Learning-Architekturen aus den Movebis-Ansätzen zur Analyse der Modellgrößen. Das dreigeteilte \acrshort{rnn}-Netzwerk aus \cite{werner_kontinuierliche_2020} wird an der oberen Seite entlang der Zeitreihe eingebunden. \cite{stojanov_continuous_2020} stapelt die erstellten Feature-Fenster in drei Dimensionen (Zeitstempel, Anzahl Features, gestapelte Fenster) für den Input in das \acrshort{cnn}.}
  \label{fig:ml-movebis}
\end{figure}

\tocless\chapter{SHL-Datensatzanalyse}\label{appendix:b}

\begin{figure}[h]
\includegraphics[width=\linewidth, bb=0 0 1048 378]{shl/acceleration.pdf}
\caption{SHL Datensatz: Acceleration Boxplot.}
\end{figure}

\begin{figure}[h]
\includegraphics[width=\linewidth, bb=0 0 1057 378]{shl/gravity.pdf}
\caption{SHL Datensatz: Gravity Boxplot.}
\end{figure}

\begin{figure}[h]
\includegraphics[width=\linewidth, bb=0 0 1041 378]{shl/gyroscrope.pdf}
\caption{SHL Datensatz: Gyroscope Boxplot.}
\end{figure}

\begin{figure}[h]
\includegraphics[width=\linewidth, bb=0 0 1048 378]{shl/linear_acceleration.pdf}
\caption{SHL Datensatz: Linear Acceleration Boxplot.}
\end{figure}

\begin{figure}[h]
\includegraphics[width=\linewidth, bb=0 0 1054 378]{shl/magnetometer.pdf}
\caption{SHL Datensatz: Magnetometer Boxplot.}
\end{figure}

\begin{figure}[h]
\includegraphics[width=\linewidth, bb=0 0 1377 378]{shl/orientation.pdf}
\caption{SHL Datensatz: Orientation Boxplot.}
\end{figure}

\begin{figure}[h]
\includegraphics[width=\linewidth, bb=0 0 1052 378]{shl/other.pdf}
\caption{SHL Datensatz: Other Boxplot.}
\end{figure}

\begin{figure}[h]
\includegraphics[width=.5\linewidth, bb=0 0 423 297]{shl/shl-label-quantities.pdf}
\caption{SHL Datensatz: Label Quantities.}
\end{figure}

\begin{figure}[h]
\includegraphics[width=.5\linewidth, bb=0 0 416 297]{shl/shl-trip-lengths.pdf}
\caption{SHL Datensatz: Trip Lengths.}
\end{figure}

  % \end{appendices}

\end{document}
